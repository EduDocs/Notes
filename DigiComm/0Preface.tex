\chapter{Preface}

These notes provide an introduction to the fundamental concepts of digital communication systems.
The material emphasizes the unifying principles of communication theory, taking a mathematical approach to system design.
The main topics covered in these notes include sampling, quantization, data compression, channel coding, Shannon capacity and modulation theory.
Possessing some programming skills will also help in order to appreciate and use the computing material and examples contained in this document.

\section*{Major Goals}

\begin{enumerate}

\item Identify the various components of a digital communication system.
Discuss the purpose of source coding, channel coding, modulation, and equalization.
Become familiar with commonly encountered digital communication systems, and discuss how these systems can be decomposed into the same abstract constituent parts.
\item Review basic notions from Fourier analysis, including Fourier series and Fourier transforms.
Define the power spectrum of stochastic signals, and explore how it is affected by linear filtering.
\item Explore methods to convert an analog signal into a digital format through sampling and quantization.
Define the mean squared error and explain its role in assessing the performance of a digital communication system.
\item Discuss the purpose of information theory, and calculate the entropy of simple information sources.
Understand fundamental compression limits and survey efficient source coding algorithms.
\item Introduce the notions of channel capacity and error protection.
Understand how simple block codes work and compute the probability of decoding failure for simples codes.
\item Present simple modulation schemes, signal waveforms, and their vector space representations.
Characterize the structure of optimal receivers, and compute the probabilities of symbol and bit errors at the output of the demodulator.
\item Explore the properties of bandlimited channels.
Study the causes and implications of intersymbol interference, and derive the Nyquist criterion for no interference.
Review simple channel equalization schemes and go over the advantages of orthogonal frequency-division multiplexing.

\end{enumerate}

