\chapter{Sums}

\begin{equation*}
\begin{split}
\Expect [Y] &= \sum_{y \in g(X(\Omega))} y p_Y(y) \\
&= \sum_{y \in g(X(\Omega))} (ax + b) \sum_{x \in X(\Omega) | g(x) = y} p_X(x) \\
&= \sum_{x \in X(\Omega)} (ax + b) p_X(x)
\end{split}
\end{equation*}


\begin{proposition}
For any positive integer $n$, we have
\begin{equation*}
\sum_{k=1}^n k = \frac{n (n+1)}{2} .
\end{equation*}
\end{proposition}
\begin{proof}
This equation can be established through the following steps,
\begin{equation*}
\begin{split}
2 \sum_{k=1}^n k
&= \sum_{k=1}^n k + \sum_{k=1}^n (n + 1 - k) \\
&= \sum_{k=1}^n \left( k + (n + 1 - k) \right) \\
&= \sum_{k=1}^n (n + 1)
= n (n+1) .
\end{split}
\end{equation*}
The desired result is obtained by deviding both sides by two.
\end{proof}

\begin{proposition}
For any positive integer $n$, we have
\begin{equation*}
\sum_{k=1}^n k^2 = \frac{n (n+1) (2n+1)}{6} .
\end{equation*}
\end{proposition}

The sum of the first $n$ square numbers is equal to $\frac{n (n+1)(2n + 1)}{6}$.
The sum of the first $n$ cubic numbers is equal to $\frac{n^2 (n+1)^2}{4}$.

