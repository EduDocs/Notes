\chapter*{Preface}

These notes provide an introduction to probability.
It presents a treatment of probabilistic concepts and techniques necessary for a basic understanding of the subject.
The notes are designed to be used in a semester course totaling 45 hours.
The reader is expected to have a basic understanding of calculus including sequences, series, derivatives and integrals.

Possessing some programming skills will also help in order to appreciate and use the computing material and examples contained in the text.
Probability theory makes predictions about experiments whose outcomes depend upon chance.
Consequently, it lends itself beautifully to the use of computers as a tool to simulate and analyze experiments.
The computer is a powerful aid in understanding basic results of probability theory, especially through imaging and graphical representation of difficult concepts.
Finally, computer programs are useful in solving problems that do not lend themselves to close-form expressions.

This material is based upon work supported, in part, by the National Science Foundation (NSF).
Any opinions, findings, and conclusions or recommendations expressed in this material are those of the authors and do not necessarily reflect the views of the National Science Foundation.

