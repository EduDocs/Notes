\chapter{Linear Algebra}


\section{Fields}

Consider a set $F$ of objects and two operations on the elements of $F$, addition and multiplication.
For every pair of elements $s, t \in F$ then $(s + t) \in F$.
For every pair of elements $s, t \in F$ then $st \in F$.
Suppose that these two operations satisfy
\begin{enumerate}
\item addition is commutative: $s + t = t + s \; \forall s, t \in F$
\item addition is associative: $r + (s + t) = (r + s) + t \; \forall r, s, t \in F$
\item to each $s \in F$ there exists a unique element $(-s) \in F$ such that $s + (-s) = 0$
\item multiplication is commutative: $st = ts \; \forall s,t \in F$
\item multiplication is associative: $r(st) = (rs)t \; \forall r, s, t \in F$
\item there is a unique non-zero element $1 \in F$ such that $s1 = s \; \forall s \in F$
\item to each $s \in F - 0$ there exists a unique element $s^{-1} \in F$ such that $s s^{-1} = 1$
\item multiplication distributes over addition: $r (s + t) = rs + rt \; \forall r, s, t \in F$.
\end{enumerate}
Then, the set $F$ together with these two operations is a \emph{field}.

\begin{example}
The real numbers with the usual operations of addition and multiplication form a field.
The complex numbers with these two operations also form a field.
\end{example}

\begin{example}
The set of integers with addition and multiplication is not a field.
\end{example}

\begin{problem} \label{problem:RationalNumbers}
Is the set of rational numbers a subfield of the real numbers?
\end{problem}

\begin{example}
Is the set of all real numbers of the form $s + t \sqrt{2}$, where $s$ and $t$ are rational, a subfield of the complex numbers?

The set $F = \left\{ s + t \sqrt{2} : s, t \in \RationalNumbers \right\}$ together with the standard addition and multiplication is a field.
Let $s, t, u, v \in \RationalNumbers$,
\begin{gather*}
s + t \sqrt{2} + u + v \sqrt{2} = (s+u) + (t+v) \sqrt{2} \in F \\
\left( s + t \sqrt{2} \right) \left( u + v \sqrt{2} \right) = (su + 2 tv) + (sv + tu) \sqrt{2} \in F \\
\left( s + t \sqrt{2} \right)^{-1} = \frac{ s - t \sqrt{2} }{ s^2 + 2 t^2 }
= \frac{ s }{ s^2 + 2 t^2 } - \frac{ t }{ s^2 + 2 t^2 } \sqrt{2} \in F
\end{gather*}
Again, the remaining properties are straightforward to prove.
The field $s + t \sqrt{2}$, where $s$ and $t$ are rational, is a subfield of the complex numbers.
\end{example}


\section{Matrices}

Let $F$ be a field and consider the problem of finding $n$ scalars $x_1, \ldots, x_n$ which satisfy the conditions
\begin{equation} \label{equation:SystemOfEquations}
\begin{array}{ccccccccc}
a_{11} x_1 & + & a_{12} x_2 & + & \cdots & + & a_{1n} x_n & = & y_1 \\
a_{21} x_1 & + & a_{22} x_2 & + & \cdots & + & a_{2n} x_n & = & y_2 \\
\vdots & & \vdots  & & & & \vdots & & \vdots \\
a_{m1} x_1 & + & a_{m2} x_2 & + & \cdots & + & a_{mn} x_n & = & y_m
\end{array}
\end{equation}
where $\{ y_i : 1 \leq i \leq n \} \subset F$ and $\{ a_{ij} : 1 \leq i \leq m, 1 \leq j \leq n \} \subset F$.
These conditons form a \emph{system of $m$ linear equations in $n$ unknowns}.
A shorthand notation for~\eqref{equation:SystemOfEquations} is the matrix equation
\begin{equation*}
A \vecnot{x} = \vecnot{y},
\end{equation*}
where $A$ is the matrix representation given by
\begin{equation*}
A = \left[ \begin{array}{cccc}
a_{11} & a_{12} & \cdots & a_{1n} \\
a_{21} & a_{22} & \cdots & a_{2n} \\
\vdots & \vdots & \ddots & \vdots \\
a_{m1} & a_{m2} & \cdots & a_{mn}
\end{array} \right]
\end{equation*}
and $\vecnot{x}$, $\vecnot{y}$ denote
\begin{align*}
\vecnot{x} &= (x_1, \ldots, x_n)^T \\
\vecnot{y} &= (y_1, \ldots, y_m)^T .
\end{align*}

\begin{definition}
Let $A$ be an $m \times n$ matrix over $F$ and let $B$ be an $n \times p$ matrix over $F$.
the \emph{matrix product} $AB$ is the $m \times p$ matrix $C$ whose $i,j$ entry is
\begin{equation*}
c_{ij} = \sum_{r = 1}^n a_{ir} b_{rj}.
\end{equation*}
\end{definition}

\begin{definition}
Let $A$ be an $n \times n$ matrix over $F$.
An $n \times n$ matrix $B$ is called the \emph{inverse} of $A$ if
\begin{equation*}
AB = BA = I.
\end{equation*}
If $A$ is invertible then its inverse will be denoted by $A^{-1}$.
\end{definition}


\section{Vector Spaces}

\begin{definition} \label{definition:VectorSpace}
A \emph{vector space} consists of the following,
\begin{enumerate}
\item a field $F$ of scalars
\item a set $V$ of objects, called vectors
\item an operation called vector addition, which associates with each pair of vectors $\vecnot{v}, \vecnot{w} \in V$ a vector $\vecnot{v} + \vecnot{w} \in V$ such that
\begin{enumerate}
\item addition is commutative: $\vecnot{v} + \vecnot{w} = \vecnot{w} + \vecnot{v}$
\item addition is associative: $\vecnot{u} + (\vecnot{v} + \vecnot{w}) = (\vecnot{u} + \vecnot{v}) + \vecnot{w}$
\item there is a unique vector $\vecnot{0} \in V$ such that $\vecnot{v} + \vecnot{0} = \vecnot{v}$, $\forall \vecnot{v} \in V$
\item to each $\vecnot{v} \in V$ there is a unique vector $- \vecnot{v} \in V$ such that $\vecnot{v} + (- \vecnot{v}) = \vecnot{0}$
\end{enumerate}
\item an operation called scalar multiplication, which associates with each $s \in F$ and $\vecnot{v} \in V$ a vector $s \vecnot{v} \in V$ such that
\begin{enumerate}
\item $1 \vecnot{v} = \vecnot{v}$, $\forall \vecnot{v} \in V$
\item $(s_1 s_2) \vecnot{v} = s_1 ( s_2 \vecnot{v} )$
\item $s ( \vecnot{v} + \vecnot{w} ) = s \vecnot{v} + s \vecnot{w}$
\item $(s_1 + s_2) \vecnot{v} = s_1 \vecnot{v} + s_2 \vecnot{w}$.
\end{enumerate}
\end{enumerate}
\end{definition}

\begin{example}
Let $F$ be a field, and let $V$ be the set of all $n$-tuples $\vecnot{v} = (v_1, \ldots, v_n)$ of scalar $v_i \in F$.
If $\vecnot{w} = (w_1, \ldots, w_n)$ with $w_i \in F$, the sum of $\vecnot{v}$ and $\vecnot{w}$ is defined by
\begin{equation*}
\vecnot{v} + \vecnot{w} = (v_1 + w_1, \ldots, v_n + w_n).
\end{equation*}
The product of a scalar $s \in F$ and vector $\vecnot{v}$ is defined by
\begin{equation*}
s \vecnot{v} = (s v_1, \ldots, s v_n).
\end{equation*}
The set of $n$-tuples, denoted by $F^n$, together with the vector additon and scalar product defined above forms a vector space.
\end{example}

\begin{example}
Let $X$ be a non-empty set and let $F$ be a field.
Consider the set $V$ of all functions from $X$ into $F$.
The sum of two vectors $f,g \in V$ is the function from $X$ into $F$ defined by
\begin{equation*}
(f + g)(x) = f(x) + g(x) \quad \forall x \in X.
\end{equation*}
The product of scalar $s \in F$ and the function $f \in V$ is the function $sf$ defined by
\begin{equation*}
(sf)(x) = s f(x) \; \forall x \in X.
\end{equation*}
\end{example}

\begin{definition}
A vector $\vecnot{w} \in V$ is said to be a \emph{linear combination} of the vectors $\vecnot{v}_1, \ldots, \vecnot{v}_n \in V$ provided that there exist scalars $s_1, \ldots, s_n \in F$ such that
\begin{equation*}
\vecnot{w} = \sum_{i=1}^n s_i \vecnot{v}_i.
\end{equation*}
\end{definition}


\subsection{Subspaces}

\begin{definition}
Let $V$ be a vector space over $F$.
A \emph{subspace} of $V$ is a subset $W \subset V$ which is itself a vector space over $F$.
\end{definition}

A non-empty subset $W \subset V$ is a subspace of $V$ if and only if for every pair of vector $\vecnot{w}_1, \vecnot{w}_2 \in W$ and every scalar $s \in F$ the vector $s \vecnot{w}_1 + \vecnot{w}_2$ is again in $W$.
If $V$ is a vector space then the intersection of any collection of subspaces of $V$ is a subspace of $V$.

\begin{example}
Let $A$ be an $m \times n$ matrix over $F$.
The set of all $n \times 1$ column vectors $V$ such that
\begin{equation*}
\vecnot{v} \in V \implies A \vecnot{v} = \vecnot{0}
\end{equation*}
is a subspace of $F^{n \times 1}$.
\end{example}

\begin{definition}
Let $U$ be a set of vectors in a vector space $V$.
The \emph{subspace spanned} by $U$ is defined to be the intersection $W$ of all subspaces of $V$ which contain $U$.
\end{definition}

The subspace spanned by a non-empty subset $U \subset V$, where $V$ is a vector space, is the set of all linear combinations of vectors in $U$.


\subsection{Bases and Dimension}
\label{section:BasesAndDimension}

The dimension of a vector space can be defined using the concept of a basis for the space.

\begin{definition}
Let $V$ be a vector space over $F$.
A subset $U \subset V$ is \emph{linearly dependent} if there exist distinct vectors $\vecnot{u}_1, \ldots, \vecnot{u}_n \in U$ and scalars $s_1, \ldots, s_n \in F$, not all of which are $0$, such that
\begin{equation*}
\sum_{i=1}^n s_i \vecnot{u}_i = 0.
\end{equation*}
A set which is not linearly dependent is called \emph{linearly independent}.
\end{definition}

A few important consequences follow immediately from this definition.
Any subset of a linearly independent set is also linearly independent.
Any set which contains the $\vecnot{0}$ vector is linearly dependent.
A set $U \subset V$ is linearly independent if and only if each finite subset of $U$ is linearly independent.

\begin{definition}
Let $V$ be a vector space over $F$.
Let $\mathcal{B} = \left\{ \vecnot{v}_{\alpha} | \alpha \in A \right\}$ be a subset of linearly independent vectors from $V$, such that every $\vecnot{v} \in V$ has a decomposition as a finite linear combination of vectors from $\mathcal{B}$.
Then, the set $\mathcal{B}$ is a \emph{basis} for $V$.
The space $V$ is \emph{finite-dimensional} if it has a finite basis.
\end{definition}

\begin{theorem}
Every vector space has a basis.
\end{theorem}
\begin{proof}
Let $X$ be the set of linearly independent subsets of $V$.
Furthermore, for $x, y \in X$ consider the strict partial order defined by proper inclusion.
By the maximum principle, if $x$ is an element of $X$, then there exists a maximal simply ordered subset $Z$ of $X$ containing $x$.
This element is a basis for $V$.
\end{proof}

\begin{example}
Let $F$ be a field and let $U \subset F^n$ be the subset consisting of the vectors $\vecnot{e}_1, \ldots, \vecnot{e}_n$ defined by
\begin{equation*}
\begin{array}{ccc}
\vecnot{e}_1 & = & (1, 0, \ldots, 0) \\
\vecnot{e}_2 & = & (0, 1, \ldots, 0) \\
\vdots & = & \vdots \\
\vecnot{e}_n & = & (0, 0, \ldots, 1).
\end{array}
\end{equation*}
For any $\vecnot{v} = (v_1, \ldots, v_n) \in F^n$, we have
\begin{equation} \label{equation:StandartBasisSpan}
\vecnot{v} = \sum_{i=1}^n v_i \vecnot{e}_i.
\end{equation}
Thus, the collection $U = \left\{ \vecnot{e}_1, \ldots, \vecnot{e}_n \right\}$ spans $F^n$.
Since $\vecnot{v} = \vecnot{0}$ in~\eqref{equation:StandartBasisSpan} if and only if $v_1 = \cdots = v_n = 0$, $U$ is linearly independent.
Accordingly, the set $U$ is a basis for $F^{n \times 1}$.
This basis is termed the \emph{standard basis} of $F^n$.
\end{example}

\begin{example}
Let $A$ be an invertible matrix over $F$.
The columns of $A$, denoted by $A_1, \ldots, A_n$, form a basis for the space of column vectors $F^{n \times 1}$.
This can be seen as follows.
If $\vecnot{v} = (v_1, \ldots, v_n)^T$ is a column vector, then
\begin{equation*}
A \vecnot{v} = \sum_{i=1}^n v_i A_i.
\end{equation*}
Since $A$ is invertible,
\begin{equation*}
A \vecnot{v} = \vecnot{0} \implies I \vecnot{v} = A^{-1} \vecnot{0} \implies \vecnot{v} = \vecnot{0}.
\end{equation*}
That is, $\{ A_1, \ldots, A_n \}$ is a linearly independent set.
For any column vector $\vecnot{w} \in F^n$, let $\vecnot{v} = A^{-1} \vecnot{w}$.
It follows that $\vecnot{w} = A \vecnot{v}$ and, as a consequence, $\{ A_1, \ldots, A_n \}$ is a basis for $F^n$.
\end{example}

\begin{theorem}
Let $V$ be a vector space which is spanned by a finite set of vectors $W = \left\{ \vecnot{w}_1, \ldots, \vecnot{w}_n \right\}$.
Then any linearly independent set of vectors in $V$ is finite and contains no more than $n$ elements.
\end{theorem}
\begin{proof}
Assume that $U = \left\{ \vecnot{u}_1, \ldots, \vecnot{u}_m \right\} \subset V$ with $m > n$.
Since $W$ spans $V$, there exists scalars $a_{ij}$ such that
\begin{equation*}
\vecnot{u}_j = \sum_{i=1}^n a_{ij} \vecnot{w}_i.
\end{equation*}
For any $m$ scalars $s_1, \ldots, s_m$ we have
\begin{equation*}
\sum_{j=1}^m s_j \vecnot{u}_j
= \sum_{j=1}^m s_j \sum_{i=1}^n a_{ij} \vecnot{w}_i
= \sum_{j=1}^m \sum_{i=1}^n ( a_{ij} s_j ) \vecnot{w}_i
= \sum_{i=1}^n \left( \sum_{j=1}^m a_{ij} s_j \right) \vecnot{w}_i . 
\end{equation*}
Because $m > n$ there exist scalars $s_1, \ldots, s_n$, not all $0$, such that
\begin{equation*}
\sum_{j=1}^m a_{ij} s_j = 0, \quad 1 \leq i \leq n. 
\end{equation*}
For these scalars, $\sum_{j=1}^m s_j \vecnot{u}_j = 0$.
That is, the set $U$ is linearly dependent.
\end{proof}

If $V$ is a finite-dimensional vector space, then any two bases of $V$ have the same number of elements.
This allows us to define the dimension of a finite-dimensional vector space.

\begin{definition}
The \emph{dimension} of a finite-dimensional vector space is defined as the number of elements in a basis for $V$.
We denote the dimension of a finite-dimensional vector space $V$ by $\dim(V)$.
\end{definition}

The zero subspace of a vector space $V$ is the subspace spanned by the vector $\vecnot{0}$.
Since the set $\left\{ \vecnot{0} \right\}$ is linearly dependent and not a basis, we assign a dimension $0$ to the zero subspace.
Alternatively, it can be argue that the empty set $\emptyset$ spans $\left\{ \vecnot{0} \right\}$ because the intersection of all the subspaces containing the empty set is $\left\{ 0 \right\}$.
Though this is only a minor point.

\begin{theorem}
Let $A$ be an $n \times n$ matrix over $F$ and suppose that the columns of $A$, denoted by $A_1, \ldots, A_n$, form a linearly independent set of vectors in $F^{n \times 1}$.
Then $A$ is invertible.
\end{theorem}
\begin{proof}
Suppose that $W$ is the subspace of $F^{n \times 1}$ spanned by $A_1, \ldots, A_n$.
Since $A_1, \ldots, A_n$ are linearly independent, $\dim(W) = n = \dim(F^{n \times 1})$.
It follows that $W = V$ and, as a consequence, there exist scalars $b_{ij} \in F$ such that
\begin{equation*}
\vecnot{e}_j = \sum_{i=1}^n b_{ij} A_i, \quad 1 \leq j \leq n
\end{equation*}
where $\left\{ \vecnot{e}_1, \ldots, \vecnot{e}_n \right\}$ is the standard basis for $F^{n \times 1}$.
Then, for the matrix $B$ with entries $b_{ij}$, we have
\begin{equation*}
AB = I.
\end{equation*}
Note also that if the rows of $A$ form a linearly independent set of vectors in $F^{1 \times n}$ then $A$ is invertible.
\end{proof}


\subsection{Coordinate System}

Let $\left\{ \vecnot{v}_1, \ldots, \vecnot{v}_n \right\}$ be a basis for the $n$-dimensional vector space $V$.
Every vector $\vecnot{w} \in V$ can be expressed uniquely as
\begin{equation*}
\vecnot{w} = \sum_{i=1}^n s_i \vecnot{v}_i.
\end{equation*}
Recall that a set is an unordered collection of objects.
Yet it is possible to order the elements of $\left\{ \vecnot{v}_1, \ldots, \vecnot{v}_n \right\}$.
One of the useful features of an ordered basis is that it enables the introduction of a coordinate system in $V$.

\begin{definition}
If $V$ is a finite-dimensional vector space, an \emph{ordered basis} for $V$ is a finite sequence of vectors which is linearly independent and spans $V$.
\end{definition}

In particular, if the sequence $\vecnot{v}_1, \ldots, \vecnot{v}_n$ is an ordered basis for $V$, then the set $\left\{ \vecnot{v}_1, \ldots, \vecnot{v}_n \right\}$ is a basis for $V$.
The ordered basis $\mathcal{B}$ is the set $\left\{ \vecnot{v}_1, \ldots, \vecnot{v}_n \right\}$, together with the specific ordering of the vectors.
Based on this ordered basis, a vector $\vecnot{w} \in V$ can be unambiguously represented as an $n$-tuple,
\begin{equation*}
\vecnot{w} = (s_1, \ldots, s_n) = \sum_{i=1}^n s_i \vecnot{v}_i.
\end{equation*}
Equivalently, vector $\vecnot{w}$ can be described using the \emph{coordinate matrix of $\vecnot{w}$ relative to the ordered basis $\mathcal{B}$}:
\begin{equation*}
\vecnot{w} = \left[ \begin{array}{c} s_1 \\ \vdots \\ s_n \end{array} \right].
\end{equation*}
The dependence of this coordinate matrix on the basis $\mathcal{B}$ can be specified explicitly using the notation $\left[ \vecnot{w} \right]_{\mathcal{B}}$.
This will be particularly important when multiple coordinates systems are involved.

\begin{example}
The canonical example of an ordered basis is the standard basis for $F^n$ introduced in Section~\ref{section:BasesAndDimension}.
Note that the standard basis contains a natural ordering: $\vecnot{e}_1, \ldots, \vecnot{e}_n$.
Vectors in $F^n$ can therefore be unambiguously expressed as $n$-tuples.
\end{example}

Let $V$ be a finite-dimensional vector space and assume that
\begin{align*}
\mathcal{A} &= \vecnot{v}_1, \ldots, \vecnot{v}_n \\
\mathcal{B} &= \vecnot{w}_1, \ldots, \vecnot{w}_n
\end{align*}
are two ordered bases for $V$.
There are unique scalars $p_{ij}$ such that
\begin{equation*}
\vecnot{w}_j = \sum_{i=1}^n p_{ij} \vecnot{v}_i \quad 1 \leq j \leq n.
\end{equation*}
Let $\vecnot{u} \in \mathcal{B}$ and
\begin{equation*}
\vecnot{u} = \left[ \begin{array}{c} t_1 \\ \vdots \\ t_n \end{array} \right]_{\mathcal{B}}.
\end{equation*}
Then, we can write
\begin{equation*}
\vecnot{u} = \sum_{j=1}^n t_j \vecnot{w}_j
= \sum_{j=1}^n t_j \sum_{i=1}^n p_{ij} \vecnot{v}_i
= \sum_{j=1}^n \sum_{i=1}^n ( p_{ij} t_j ) \vecnot{v}_i
= \sum_{i=1}^n \left( \sum_{j=1}^n p_{ij} t_j \right) \vecnot{v}_i.
\end{equation*}
Since the coordinates $s_1, \ldots, s_n$ of $\vecnot{u}$ in the ordered basis $\mathcal{A}$ are uniquely determined, it follows that
\begin{equation*}
s_i = \sum_{j=1}^n p_{ij} t_j \quad 1 \leq i \leq n.
\end{equation*}
Let $P$ be the $n \times n$ matrix with entries $p_{ij}$, then $P$ is invertible and
\begin{align*}
\left[ \vecnot{u} \right]_{\mathcal{A}} &= P \left[ \vecnot{u} \right]_{\mathcal{B}} \\
\left[ \vecnot{u} \right]_{\mathcal{B}} &= P^{-1} \left[ \vecnot{u} \right]_{\mathcal{A}}
\end{align*}
for every $\vecnot{u} \in V$.
Furthermore, the columns of $P$ are given by
\begin{equation*}
P_j = \left[ \vecnot{w}_j \right]_{\mathcal{A}} \quad 1 \leq j \leq n.
\end{equation*}

\begin{problem} \label{problem:OrderBasisFromMatrix}
Suppose that $\mathcal{A} = \vecnot{v}_1, \ldots, \vecnot{v}_n$ is an ordered basis for $V$.
Let $P$ be an $n \times n$ invertible matrix.
Show that there exists an ordered basis $\mathcal{B} = \vecnot{w}_1, \ldots, \vecnot{w}_n$ for $V$ such that
\begin{align*}
\left[ \vecnot{u} \right]_{\mathcal{A}} &= P \left[ \vecnot{u} \right]_{\mathcal{B}} \\
\left[ \vecnot{u} \right]_{\mathcal{B}} &= P^{-1} \left[ \vecnot{u} \right]_{\mathcal{A}}
\end{align*}
for every $\vecnot{u} \in V$.
\end{problem}
\textbf{S~\ref{problem:OrderBasisFromMatrix}}.
Consider the ordered basis $\mathcal{A} = \vecnot{v}_1, \ldots, \vecnot{v}_n$.
If $\mathcal{B} = \vecnot{w}_1, \ldots, \vecnot{w}_n$ is an ordered basis for $V$ such that $\left[ \vecnot{u} \right]_{\mathcal{A}} = P \left[ \vecnot{u} \right]_{\mathcal{B}}$ then
\begin{equation*}
\vecnot{w}_j = \sum_{i = 1}^n p_{ij} \vecnot{v}_i.
\end{equation*}
We need only show that the vectors $\vecnot{w}_1, \ldots, \vecnot{w}_n$ defined by these equations form a basis for $V$.
Let $Q = P^{-1}$, then
\begin{equation*}
\begin{split}
\sum_{j = 1}^n q_{jk} \vecnot{w}_j
&= \sum_{j = 1}^n q_{jk} \sum_{i = 1}^n p_{ij} \vecnot{v}_i
= \sum_{j = 1}^n \sum_{i = 1}^n p_{ij} q_{jk} \vecnot{v}_i \\
&= \sum_{i = 1}^n \left( \sum_{j = 1}^n p_{ij} q_{jk} \right) \vecnot{v}_i
= \vecnot{v}_k .
\end{split}
\end{equation*}
The subspace spanned by the set $\mathcal{B} = \vecnot{w}_1, \ldots, \vecnot{w}_n$ containes $\mathcal{A}$.
Thus $\mathcal{V}$ is a basis for $V$.
By definition, it clear that $\left[ \vecnot{u} \right]_{\mathcal{A}} = P \left[ \vecnot{u} \right]_{\mathcal{B}}$ and $\left[ \vecnot{u} \right]_{\mathcal{B}} = P^{-1} \left[ \vecnot{u} \right]_{\mathcal{A}}$.



\section{Norms}
Let $V$ be a vector space over the real numbers or the complex numbers.

A \emph{norm} on vector space $V$ is a real-valued function $\left\| \cdot \right\| : V \rightarrow \RealNumbers$ that satisfies the following properties.
\begin{enumerate}
\item $\left\| \vecnot{v} \right\| \geq 0 \quad \forall \vecnot{v} \in V$;
equality holds if and only if $\vecnot{v} = \vecnot{0}$
\item $\left\| s \vecnot{v} \right\| = |s| \left\| \vecnot{v} \right\| \quad \forall \vecnot{v} \in V, s \in F$
\item $\left\| \vecnot{v} + \vecnot{w} \right\| \leq
\left\| \vecnot{v} \right\| + \left\| \vecnot{w} \right\| \quad \forall \vecnot{v}, \vecnot{w} \in V$.
\end{enumerate}

The concept of a norm is closely related to the concept of a metric.
For instance, a metric can be defined in terms of a norm.
Let $\left\| \vecnot{v} \right\|$ be a norm on vector space $V$, then
\begin{equation*}
d \left( \vecnot{v}, \vecnot{w} \right)
= \left\| \vecnot{v} - \vecnot{w} \right\|
\end{equation*}
is a metric.

\begin{example}
Consider vectors in $\RealNumbers^n$ with the euclidean metric
\begin{equation*}
d \left( \vecnot{v}, \vecnot{w} \right)
= \sqrt{ (v_1 - w_1)^2 + \cdots + (v_n - w_n)^2 }.
\end{equation*}
Recall that the standard bounded metric introduced in Problem~\ref{problem:StandardBoundedMetric} is given by
\begin{equation*}
\bar{d} \left( \vecnot{v}, \vecnot{w} \right)
= \min \left\{ d \left( \vecnot{v}, \vecnot{w} \right), 1 \right\}.
\end{equation*}
Define the function $f : \RealNumbers^n \rightarrow \RealNumbers$ by
\begin{equation*}
f \left( \vecnot{v} \right) = \bar{d} \left( \vecnot{v}, \vecnot{0} \right).
\end{equation*}
Is the function $f$ a norm?

By the properties of a metric, we have
\begin{enumerate}
\item $\bar{d} \left( \vecnot{v}, \vecnot{0} \right) \geq 0 \quad \forall \vecnot{v} \in V$; equality holds if and only if $\vecnot{v} = \vecnot{0}$
\item $\bar{d} \left( \vecnot{v}, \vecnot{0} \right) + \bar{d} \left( \vecnot{w}, \vecnot{0} \right) = \bar{d} \left( \vecnot{v}, \vecnot{0} \right) + \bar{d} \left( \vecnot{0}, \vecnot{w} \right) \geq \bar{d} \left( \vecnot{v}, \vecnot{w} \right) \quad \forall \vecnot{v}, \vecnot{w} \in V$.
\end{enumerate}
However, $\bar{d} \left( s \vecnot{v}, \vecnot{0} \right)$ is not necessarily equal to $s \bar{d} \left( \vecnot{v}, \vecnot{0} \right)$.
For instance,
$\bar{d} \left( 2 \vecnot{e}_1, \vecnot{0} \right) = 1 < 2 \bar{d} \left( \vecnot{e}_1, \vecnot{0} \right)$.
Thus, the function $f : \RealNumbers^n \rightarrow \RealNumbers$ defined by
\begin{equation*}
f \left( \vecnot{v} \right) = \bar{d} \left( \vecnot{v}, \vecnot{0} \right).
\end{equation*}
is not a norm.
\end{example}

\begin{example}
The following functions are examples of norms for $F^n$,
\begin{enumerate}
\item the $l_1$ norm: $\left\| \vecnot{v} \right\|_1 = \sum_{i=1}^n |x_i|$
\item the $l_p$ norm: $\left\| \vecnot{v} \right\|_p = \left( \sum_{i=1}^n |x_i|^p \right)^{\frac{1}{p}}, \quad p \in (1,\infty)$
\item the $l_{\infty}$ norm: $\left\| \vecnot{v} \right\|_{\infty} = \max_{1,\ldots, n} \{ |x_i| \}$.
\end{enumerate}
\end{example}

\begin{example}
Similarly, norms can be defined for the vector space of functions from $[a, b]$ to $F$,
\begin{enumerate}
\item the $L_1$ norm: $\left\| f(t) \right\|_1 = \int_a^b |f(t)| dt$
\item the $L_p$ norm: $\left\| f(t) \right\|_p = \left( \int_a^b |f(t)|^p dt \right)^{\frac{1}{p}}, \quad p \in (1,\infty)$
\item the $L_{\infty}$ norm: $\left\| f(t) \right\|_{\infty} = \esssup_{[a,b]} \{ | f(t) | \}$.
\end{enumerate}
\end{example}

\begin{definition}
A vector $\vecnot{v} \in V$ is said to be \emph{normalized} if $\left\| \vecnot{v} \right\| = 1$.
Any vector can be normalized, except the zero vector:
\begin{equation}
\vecnot{u} = \frac{\vecnot{v}}{\left\| \vecnot{v} \right\|}
\end{equation}
has norm $\left\| \vecnot{u} \right\| = 1$.
A normalized vector is also referred to as a \emph{unit vector}.
\end{definition}


\section{Inner Products}

\begin{definition} \label{definition:InnerProduct}
Let $F$ be the field of real numbers or the field of complex numbers, and assume $V$ is a vector space over $F$.
An \emph{inner product} on $V$ is a function which assigns to each ordered pair of vectors $\vecnot{v}, \vecnot{w} \in V$ a scalar $\left\langle \vecnot{v} | \vecnot{w} \right\rangle \in F$ in such a way that for all $\vecnot{u}, \vecnot{v}, \vecnot{w} \in V$ and any scalar $s \in F$
\begin{enumerate}
\item $\left\langle \vecnot{u} + \vecnot{v} | \vecnot{w} \right\rangle
= \left\langle \vecnot{u} | \vecnot{w} \right\rangle
+ \left\langle \vecnot{v} | \vecnot{w} \right\rangle$
\item $\left\langle s \vecnot{v} | \vecnot{w} \right\rangle
= s \left\langle \vecnot{v} | \vecnot{w} \right\rangle$
\item $\left\langle \vecnot{v} | \vecnot{w} \right\rangle
= \overline{ \left\langle \vecnot{w} | \vecnot{v} \right\rangle }$, where the overbar denotes complex conjugation;
\item $\left\langle \vecnot{v} | \vecnot{v} \right\rangle > 0$ if $\vecnot{v} \neq \vecnot{0}$.
\end{enumerate}
\end{definition}

Note that the conditions of Definition~\ref{definition:InnerProduct} imply that
\begin{equation*}
\left\langle \vecnot{u} | s \vecnot{v} + \vecnot{w} \right\rangle
= \overline{s} \left\langle \vecnot{u} | \vecnot{v} \right\rangle
+ \left\langle \vecnot{u} | \vecnot{w} \right\rangle .
\end{equation*}

\begin{definition}
A real or complex vector space equipped with an inner product is called an \emph{inner-product space}.
\end{definition}

\begin{example} \label{example:StandardInnerProduct}
Consider the inner product on $F^n$ defined by
\begin{equation*}
\left\langle \vecnot{v} | \vecnot{w} \right\rangle
= \left\langle (v_1, \ldots, v_n) | (w_1, \ldots, w_n) \right\rangle
= \sum_{j=1}^n v_j \overline{w}_j.
\end{equation*}
This inner product is called the \emph{standard inner product}.
When $F = \RealNumbers$, the standard inner product can also be written as
\begin{equation*}
\left\langle \vecnot{v} | \vecnot{w} \right\rangle
= \sum_{j=1}^n v_j w_j.
\end{equation*}
In this context it is often called the dot product, denoted by $\vecnot{v} \cdot \vecnot{w}$.
\end{example}

\begin{problem} \label{problem:InnerProductForm}
For $\vecnot{v} = (v_1, v_2)$ and $\vecnot{w} = (w_1, w_2)$  in $\RealNumbers^2$, show that
\begin{equation*}
\left\langle \vecnot{v} | \vecnot{w} \right\rangle
= v_1 w_1 - v_2 w_1 - v_1 w_2 + 4 v_2 w_2
\end{equation*}
is an inner product.
\end{problem}
\textbf{S~\ref{problem:InnerProductForm}}.
For all $\vecnot{u}, \vecnot{v}, \vecnot{w} \in V$ and all scalars $s$
\begin{equation*}
\begin{split}
\left\langle \vecnot{u} + \vecnot{v} | \vecnot{w} \right\rangle
&= (u_1 + v_1) w_1 - (u_2 + v_2) w_1 - (u_1 + v_1) w_2 + 4 (u_2 + v_2) w_2 \\
&= u_1 w_1 - u_2 w_1 - u_1 w_2 + 4 u_2 w_2
+ v_1 w_1 - v_2 w_1 - v_1 w_2 + 4 v_2 w_2 \\
&= \left\langle \vecnot{u} | \vecnot{w} \right\rangle
+ \left\langle \vecnot{v} | \vecnot{w} \right\rangle.
\end{split}
\end{equation*}
Also, we have
\begin{equation*}
\left\langle s \vecnot{v} | \vecnot{w} \right\rangle
= s v_1 w_1 - s v_2 w_1 - s v_1 w_2 + 4 s v_2 w_2
= s \left\langle \vecnot{v} | \vecnot{w} \right\rangle.
\end{equation*}
Since $V = \RealNumbers^2$, we have $\left\langle \vecnot{v} | \vecnot{w} \right\rangle = \overline{ \left\langle \vecnot{w} | \vecnot{v} \right\rangle }$.
Furthermore,
\begin{equation*}
\left\langle \vecnot{v} | \vecnot{v} \right\rangle
= v_1^2 - 2 v_1 v_2 + 4 v_2^2
= ( v_1 - v_2 )^2 + 3 v_2^2
> 0
\quad \text{if } \vecnot{v} \neq \vecnot{0}.
\end{equation*}
That is, $\left\langle \vecnot{v} | \vecnot{v} \right\rangle$ is an inner product.


\begin{example}
Let $V$ be the vector space of all continuous complex-valued functions on the unit interval $[0,1]$.
Then
\begin{equation*}
\left\langle f | g \right\rangle
= \int_0^1 f(t) \overline{g(t)} dt
\end{equation*}
is an inner product.
\end{example}

\begin{example}
Let $V$ and $W$ be two vector spaces over $F$ and suppose that $\langle \cdot | \cdot \rangle$ is an inner product on $W$.
If $T$ is a non-singular linear transformation from $V$ into $W$, then the equation
\begin{equation*}
p_T \left( \vecnot{v}_1, \vecnot{v}_2 \right)
= \left\langle T \vecnot{v}_1 | T \vecnot{v}_2 \right\rangle
\end{equation*}
defines an inner product $p_T$ on $V$.
\end{example}

\begin{theorem}
Let $V$ be a finite-dimensional space, and suppose that
\begin{equation*}
\mathcal{B} = \vecnot{w}_1, \ldots, \vecnot{w}_n
\end{equation*}
is an ordered basis for $V$.
Any inner product on $V$ is completely determined by the values
\begin{equation*}
h_{ji} = \left\langle \vecnot{w}_i | \vecnot{w}_j \right\rangle
\end{equation*}
it assumes on pairs of vectors in $\mathcal{B}$.
\end{theorem}
\begin{proof}
If $\vecnot{u} = \sum_{i} s_i \vecnot{w}_i$ and $\vecnot{v} = \sum_{j} t_j \vecnot{w}_j$, then
\begin{equation*}
\begin{split}
\left\langle \vecnot{u} | \vecnot{v} \right\rangle
&= \left\langle \sum_{i} s_i \vecnot{w}_i \Big| \vecnot{v} \right\rangle
= \sum_{i} s_i \left\langle \vecnot{w}_i | \vecnot{v} \right\rangle \\
&= \sum_{i} s_i \left\langle \vecnot{w}_i \Big| \sum_{j} t_j \vecnot{w}_j \right\rangle
= \sum_{i} \sum_{j} s_i \overline{t}_j \left\langle \vecnot{w}_i | \vecnot{w}_j \right\rangle \\
&= \sum_{i} \sum_{j} \overline{t}_j h_{ji} s_i
= \left[ \vecnot{v} \right]_{\mathcal{B}}^H H \left[ \vecnot{u} \right]_{\mathcal{B}}
\end{split}
\end{equation*}
where $\left[ \vecnot{u} \right]_{\mathcal{B}}$ and $\left[ \vecnot{v} \right]_{\mathcal{B}}$ are the coordinate matrices of $\vecnot{u}$, $\vecnot{v}$ in the ordered basis $\mathcal{B}$.
The matrix $H$ is called the \emph{matrix of the inner product in the ordered basis $\mathcal{B}$}.
\end{proof}

It is easily verified that $H$ is a hermitian matrix, i.e., $H = H^H$.
Furthermore, $H$ must satisfy the additional condition
\begin{equation} \label{equation:PositiveDefinite}
\vecnot{w}^H H \vecnot{w} > 0, \quad \forall \vecnot{w} \neq \vecnot{0}.
\end{equation}
In particualr, $H$ must be invertible.

Conversely if $H$ is an $n \times n$ hermitian matrix over $F$ which satisfies~\eqref{equation:PositiveDefinite}, then $H$ is the matrix in the ordered basis $\mathcal{B}$ of an inner product on $V$.
This inner product is given by
\begin{equation*}
\left\langle \vecnot{u} | \vecnot{v} \right\rangle
= \left[ \vecnot{v} \right]_{\mathcal{B}}^H H \left[ \vecnot{u} \right]_{\mathcal{B}}.
\end{equation*}

\begin{problem}
Let $V$ be a vector space over $F$.
Show that the sum of two inner products on $V$ is an inner product on $V$.
Show that a positive multiple of an inner product is also an inner product.
\end{problem}

\begin{example}
Let $X$ and $Y$ be random variables taking values in $\RealNumbers$.
Consider the inner product defined by
\begin{equation*}
\left\langle X | Y \right\rangle = \Expect \left[ XY \right] .
\end{equation*}
Note that in the vector space of random variables, $X = Y$ if $\left\| X - Y \right\| = 0$.
That is, if $X = Y$ almost surely then $X$ and $Y$ are different representation of the same vector.
\end{example}


\subsection{Induced Norms}

A finite-dimensional real inner-product space is often referred to as a \emph{Euclidean space}.
A complex inner-product space is sometimes called a \emph{unitary space}.

\begin{definition}
Let $V$ be an inner-product space with inner product $\langle \cdot | \cdot \rangle$.
This inner product can be used to define a norm, called the \emph{induced norm},
\begin{equation*}
\left\| \vecnot{v} \right\| = \left\langle \vecnot{v} | \vecnot{v} \right\rangle^{\frac{1}{2}}
\end{equation*}
for every $\vecnot{v} \in V$.
\end{definition}

\begin{theorem}
If $V$ is an inner-product space and $\| \cdot \|$ is its associated induced norm, then for any $\vecnot{v}, \vecnot{w} \in V$ and any scalar $s$
\begin{enumerate}
\item $\left\| s \vecnot{v} \right\| = |s| \left\| \vecnot{v} \right\|$
\item $\left\| \vecnot{v} \right\| > 0$ for $\vecnot{v} \neq \vecnot{0}$
\item $\left| \left\langle \vecnot{v} | \vecnot{w} \right\rangle \right| \leq \left\| \vecnot{v} \right\| \left\| \vecnot{w} \right\|$
\item $\left\| \vecnot{v} + \vecnot{w} \right\| \leq \left\| \vecnot{v} \right\| + \left\| \vecnot{w} \right\|$.
\end{enumerate}
\end{theorem}
\begin{proof}
The first two items follow immediately from the various definitions involved.
When $\vecnot{v} = \vecnot{0}$, then clearly $\left| \left\langle \vecnot{v} | \vecnot{w} \right\rangle \right| \leq \left\| \vecnot{v} \right\| \left\| \vecnot{w} \right\|$.
Assume $\vecnot{v} \neq \vecnot{0}$ and let
\begin{equation*}
\vecnot{u} = \vecnot{w} - \frac{ \left\langle \vecnot{w} | \vecnot{v} \right\rangle}{ \left\| \vecnot{v} \right\|^2 } \vecnot{v}.
\end{equation*}
Then $\left\langle \vecnot{u} | \vecnot{v} \right\rangle = 0$ and
\begin{equation*}
\begin{split}
0 &\leq \left\| \vecnot{u} \right\|^2
= \left\langle \vecnot{w} - \frac{ \left\langle \vecnot{w} | \vecnot{v} \right\rangle}{ \left\| \vecnot{v} \right\|^2 } \vecnot{v} \Big|
\vecnot{w} - \frac{ \left\langle \vecnot{w} | \vecnot{v} \right\rangle}{ \left\| \vecnot{v} \right\|^2 } \vecnot{v} \right\rangle \\
&= \left\langle \vecnot{w} | \vecnot{w} \right\rangle
- \frac{ \left\langle \vecnot{w} | \vecnot{v} \right\rangle
\left\langle \vecnot{v} | \vecnot{w} \right\rangle }
{ \left\| \vecnot{v} \right\|^2 }
= \left\| \vecnot{w} \right\|^2
- \frac{ \left| \left\langle \vecnot{v} | \vecnot{w} \right\rangle \right|^2 }
{ \left\| \vecnot{v} \right\|^2 } .
\end{split}
\end{equation*}
Hence
$\left| \left\langle \vecnot{v} | \vecnot{w} \right\rangle \right|^2 \leq \left\| \vecnot{v} \right\|^2 \left\| \vecnot{w} \right\|^2$.
Using this result, we get
\begin{equation*}
\begin{split}
\left\| \vecnot{v} + \vecnot{w} \right\|^2
&= \left\| \vecnot{v} \right\|^2 + \left\langle \vecnot{v} | \vecnot{w} \right\rangle + \left\langle \vecnot{w} | \vecnot{v} \right\rangle + \left\| \vecnot{w} \right\|^2 \\
&= \left\| \vecnot{v} \right\|^2 + 2 \Real \left\langle \vecnot{v} | \vecnot{w} \right\rangle + \left\| \vecnot{w} \right\|^2 \\
&\leq \left\| \vecnot{v} \right\|^2 + 2 \left\| \vecnot{v} \right\| \left\| \vecnot{w} \right\| + \left\| \vecnot{w} \right\|^2 .
\end{split}
\end{equation*}
Thus, $\left\| \vecnot{v} + \vecnot{w} \right\| \leq \left\| \vecnot{v} \right\| + \left\| \vecnot{w} \right\|$.
\end{proof}

The third inequality, $\left| \left\langle \vecnot{v} | \vecnot{w} \right\rangle \right| \leq \left\| \vecnot{v} \right\| \left\| \vecnot{w} \right\|$, is called the \emph{Cauchy-Schwarz inequality}.
Its proof shows that if $\vecnot{v}$ is non-zero, then $\left| \left\langle \vecnot{v} | \vecnot{w} \right\rangle \right| < \left\| \vecnot{v} \right\| \left\| \vecnot{w} \right\|$ unless
\begin{equation*}
\vecnot{w} = \frac{ \left\langle \vecnot{w} | \vecnot{v} \right\rangle }
{ \left\| \vecnot{v} \right\| } \vecnot{v} .
\end{equation*}
That is, equality holds if and only if $\vecnot{v}$ and $\vecnot{w}$ are linearly dependent.

\begin{theorem}
Consider the vector space $\RealNumbers^n$ with the standard inner product of Example~\ref{example:StandardInnerProduct}.
The function $f: V \rightarrow F$ defined by $f \left( \vecnot{w} \right) = \left\langle \vecnot{w} | \vecnot{v} \right\rangle$ is continuous.
\end{theorem}
\begin{proof}
Let $\vecnot{w}_1, \vecnot{w}_2, \ldots$ be a sequence in $V$ converging to $\vecnot{w}$.
Then,
\begin{equation*}
\left| \left\langle \vecnot{w}_n | \vecnot{v} \right\rangle
- \left\langle \vecnot{w} | \vecnot{v} \right\rangle \right|
= \left| \left\langle \vecnot{w}_n - \vecnot{w} | \vecnot{v} \right\rangle \right|
\leq \left\| \vecnot{w}_n - \vecnot{w} \right\| \left\| \vecnot{v} \right\|.
\end{equation*}
Since $\left\| \vecnot{w}_n - \vecnot{w} \right\| \rightarrow 0$, the convergence of $\left\langle \vecnot{w}_n, \vecnot{v} \right\rangle$ is established.
\end{proof}


\section{Orthogonal Projections}

\begin{definition}
Let $\vecnot{v}$ and $\vecnot{w}$ be vectors in an inner-product space $V$.
Then $\vecnot{v}$ is \emph{orthogonal} to $\vecnot{w}$ if $\left\langle \vecnot{v} | \vecnot{w} \right\rangle = 0$.
Note that this also implies that $\vecnot{w}$ is orthogonal to $\vecnot{v}$, so we can simply say that $\vecnot{v}$ and $\vecnot{w}$ are orthogonal.
A collection $W$ of vectors in $V$ is an \emph{orthogonal set} if all pairs of distinct vectors in $W$ are orthogonal.
An \emph{orthonormal set} $U$ is a collection of vectors in $V$ such that $U$ is an orthogonal set and $\left\| \vecnot{u} \right\| = 1$ for every $\vecnot{u} \in U$.
\end{definition}

\begin{example}
The standard basis of $\RealNumbers^n$ is an orthonormal set with respect to the standard inner product.
\end{example}

\begin{example}
Let $V$ be the space of continuous complex-valued functions on the interval $0 \leq x \leq 1$ with the inner product
\begin{equation*}
\left\langle f | g \right\rangle = \int_0^1 f(x) \overline{g(x)} dx.
\end{equation*}
Let $f_n(x) = \sqrt{2} \cos 2 \pi n x$ and $g_n (x) = \sqrt{2} \sin 2 \pi n x$.
Then $\{ 1, f_1, g_1, f_2, g_2, \ldots \}$ is an infinite orthonormal set.
\end{example}

\begin{theorem}
An orthogonal set of non-zero vectors is linearly independent.
\end{theorem}
\begin{proof}
Let $W$ be an orthogonal set of non-zero vectors in a given inner-product space $V$.
Suppose $\vecnot{w}_1, \ldots, \vecnot{w}_n$ are distinct vectors in $W$ and consider
\begin{equation*}
\vecnot{v} = s_1 \vecnot{w}_1 + \cdots + s_n \vecnot{w}_n.
\end{equation*}
The inner product $\left\langle \vecnot{v} | \vecnot{w}_i \right\rangle$ is given by
\begin{equation*}
\begin{split}
\left\langle \vecnot{v} | \vecnot{w}_i \right\rangle
&= \left\langle \sum_j s_j \vecnot{w}_j | \vecnot{w}_i \right\rangle
= \sum_j s_j \left\langle \vecnot{w}_j | \vecnot{w}_i \right\rangle
= s_i \left\langle \vecnot{w}_i | \vecnot{w}_i \right\rangle .
\end{split}
\end{equation*}
Since $\left\langle \vecnot{w}_i | \vecnot{w}_i \right\rangle \neq 0$, it follows that
\begin{equation*}
s_i = \frac{ \left\langle \vecnot{v} | \vecnot{w}_i \right\rangle }
{ \left\| \vecnot{w}_i \right\|^2 }
\quad 1 \leq i \leq n.
\end{equation*}
In particular, if $\vecnot{v} = 0$ then $s_j = 0$ for $1 \leq j \leq n$ and the vectors in $W$ are linearly independent.
\end{proof}

\begin{corollary}
If $\vecnot{v} \in V$ is a linear combination of an orthogonal sequence of distinct, non-zero vectors $\vecnot{w}_1, \ldots, \vecnot{w}_n$, then $\vecnot{v}$ is the particular linear combination
\begin{equation*}
\vecnot{v} = \sum_{i = 1}^n \frac{ \left\langle \vecnot{v} | \vecnot{w}_i \right\rangle } { \left\| \vecnot{w}_i \right\|^2 } \vecnot{w}_i.
\end{equation*}
\end{corollary}

\begin{theorem}
Let $V$ be an inner-product space and assume $\vecnot{v}_1, \ldots, \vecnot{v}_n$ are linearly independent vectors in $V$.
Then it is possible to construct an orthogonal sequence of vectors $\vecnot{w}_1, \ldots, \vecnot{w}_n \in V$ such that for each $k = 1, \ldots, n$ the set
\begin{equation*}
\left\{ \vecnot{w}_1, \ldots, \vecnot{w}_k \right\}
\end{equation*}
is a basis for the subspace spanned by $\vecnot{v}_1, \ldots, \vecnot{v}_k$.
\end{theorem}
\begin{proof}
First let $\vecnot{w}_1 = \vecnot{v}_1$.
Define the remaining vectors inductively as follows.
Suppose the vectors
\begin{equation*}
\vecnot{w}_1, \ldots, \vecnot{w}_m \quad (1 \leq m < n)
\end{equation*}
have been chosen so that for every $k$
\begin{equation*}
\left\{ \vecnot{w}_1, \ldots, \vecnot{w}_k \right\} \quad 1 \leq k \leq m
\end{equation*}
is an orthogonal basis for the subspace spanned by $\vecnot{v}_1, \ldots, \vecnot{v}_k$.
Let
\begin{equation*}
\vecnot{w}_{m+1} = \vecnot{v}_{m+1} - \sum_{i=1}^m \frac{ \left\langle \vecnot{v}_{m+1} | \vecnot{w}_i \right\rangle } { \left\| \vecnot{w}_i \right\|^2 } \vecnot{w}_i.
\end{equation*}
Then $\vecnot{w}_{m+1} \neq 0$, for otherwise $\vecnot{v}_{m+1}$ is a linear combination of $\vecnot{w}_1, \ldots, \vecnot{w}_m$ and hence a linear combination of $\vecnot{v}_1, \ldots, \vecnot{v}_m$.
Furthermore, for $j \in 1, \ldots, m$
\begin{equation*}
\begin{split}
\left\langle \vecnot{w}_{m+1} | \vecnot{w}_j \right\rangle
&= \left\langle \vecnot{v}_{m+1} | \vecnot{w}_j \right\rangle
- \sum_{i=1}^m \frac{ \left\langle \vecnot{v}_{m+1} | \vecnot{w}_i \right\rangle } { \left\| \vecnot{w}_i \right\|^2 }
\left\langle \vecnot{w}_i | \vecnot{w}_j \right\rangle \\
&= \left\langle \vecnot{v}_{m+1} | \vecnot{w}_j \right\rangle
- \frac{ \left\langle \vecnot{v}_{m+1} | \vecnot{w}_j \right\rangle } { \left\| \vecnot{w}_j \right\|^2 }
\left\langle \vecnot{w}_j | \vecnot{w}_j \right\rangle \\
&= 0.
\end{split}
\end{equation*}
Clearly, $\{ \vecnot{w}_1, \ldots, \vecnot{w}_{m+1} \}$ is an orthogonal set consisting of $m+1$ non-zero vectors in the subspace spanned by $\vecnot{v}_1, \ldots, \vecnot{v}_{m+1}$.
Since the dimension of the latter subspace is $m+1$, this set is a basis for the subspace.
\end{proof}

The inductive construction of the vectors $\vecnot{w}_1, \ldots, \vecnot{w}_n$ is known as the \emph{Gram-Schmidt orthogonalization process}.

\begin{corollary}
Every finite-dimensional inner-product space has a basis of orthonormal vectors.
\end{corollary}
\begin{proof}
Let $V$ be a finite-dimensional inner-product space.
Suppose that $\vecnot{v}_1, \ldots, \vecnot{v}_n$ is a basis for $V$.
Apply the Gram-Schmidt process to obtain a basis of orthogonal vectors $\vecnot{w}_1, \ldots, \vecnot{w}_n$.
Then, a basis of orthonormal vectors is given by
\begin{equation*}
\vecnot{u}_1 = \frac{ \vecnot{w}_1 }{ \left\| \vecnot{w}_1 \right\| }, \ldots,
\vecnot{u}_n = \frac{ \vecnot{w}_n }{ \left\| \vecnot{w}_n \right\| }.
\end{equation*}
\end{proof}

\begin{example}
Consider the vectors
\begin{align*}
\vecnot{v}_1 &= (2,2,1) \\
\vecnot{v}_2 &= (3,6,0) \\
\vecnot{v}_3 &= (6,3,9)
\end{align*}
in $\RealNumbers^3$ equipped with the standard inner product.
Apply the Gram-Schmidt process to $\vecnot{v}_1, \vecnot{v}_2, \vecnot{v}_3$ to obtain an orthogonal basis.

Applying the Gram-Schmidt process to $\vecnot{v}_1, \vecnot{v}_2, \vecnot{v}_3$, we get
\begin{align*}
\vecnot{w}_1 &= (2,2,1) \\
\vecnot{w}_2 &= (3,6,0)
- \frac{ \left\langle (3,6,0) | (2,2,1) \right\rangle }{ 9 } (2,2,1) \\
&= (3,6,0) - 2 (2,2,1) = (-1,2,-2) \\
\vecnot{w}_3 &= (6,3,9)
- \frac{ \left\langle (6,3,9) | (2,2,1) \right\rangle }{ 9 } (2,2,1)
- \frac{ \left\langle (6,3,9) | (-1,2,-2) \right\rangle }{ 9 } (-1,2,-2) \\
&= (6,3,9) - 3 (2,2,1) + 2 (-1,2,-2) = (-2,1,2) .
\end{align*}
It is easily verified that $\vecnot{w}_1, \vecnot{w}_2, \vecnot{w}_3$ is an orthogonal set of vectors.
\end{example}


\subsection{Orthogonal Subspace}

In essence, the Gram-Schmidt process is a sequence of orthogonal projections.
Orthogonal projections are very useful in many context.

Suppose $W$ is a subspace of an inner-product space $V$, and let $\vecnot{v}$ be an arbitrary vector in $V$.
Consider the problem of finding a vector $\vecnot{w} \in W$ such that $\left\| \vecnot{v} - \vecnot{w} \right\|$ is minimized.
The vector $\vecnot{w} \in W$ is a \emph{best approximation} to $\vecnot{v} \in V$ if
\begin{equation*}
\left\| \vecnot{v} - \vecnot{w} \right\| \leq \left\| \vecnot{v} - \vecnot{w}' \right\|
\end{equation*}
for every vector $\vecnot{w}' \in W$.

\begin{theorem} \label{theorem:OrthogonalProjection}
Suppose $W$ is a subspace of an inner-product space $V$, and let $\vecnot{v}$ be a vector in $V$.
\begin{enumerate}
\item The vector $\vecnot{w} \in W$ is a best approximation of $\vecnot{v} \in V$ by vectors in $W$ if and only if $\vecnot{v} - \vecnot{w}$ is orthogonal to every vector in $W$.
\item If a best approximation to $\vecnot{v} \in V$ by vectors in $W$ exists, it is unique.
\item If $W$ is finite-dimensional and $\vecnot{w}_1, \ldots, \vecnot{w}_n$ is an orthogonal basis for $W$, then
\begin{equation*}
\vecnot{w} = \sum_{i=1}^n \frac{ \left\langle \vecnot{v} | \vecnot{w}_i \right\rangle }{ \left\| \vecnot{w}_i \right\|^2 } \vecnot{w}_i
\end{equation*}
is the best approximation to $\vecnot{v}$ by vectors in $W$.
\end{enumerate}
\end{theorem}
\begin{proof}
Let $\vecnot{w} \in W$ and suppose $\vecnot{v} - \vecnot{w}$ is orthogonal to every vector in $W$.
Let $\vecnot{w}' \in W$ such that $\vecnot{w}' \neq \vecnot{w}$.
Then $\vecnot{v} - \vecnot{w}' = \left( \vecnot{v} - \vecnot{w} \right) + \left( \vecnot{w} - \vecnot{w}' \right)$ and
\begin{equation} \label{equation:OrthogonalVector}
\begin{split}
\left\| \vecnot{v} - \vecnot{w}' \right\|^2
&= \left\| \vecnot{v} - \vecnot{w} \right\|^2
+ 2 \Real \left\langle \vecnot{v} - \vecnot{w} | \vecnot{w} - \vecnot{w}' \right\rangle
+ \left\| \vecnot{w} - \vecnot{w}' \right\|^2 \\
&= \left\| \vecnot{v} - \vecnot{w} \right\|^2
+ \left\| \vecnot{w} - \vecnot{w}' \right\|^2 \\
&\geq \left\| \vecnot{v} - \vecnot{w} \right\|^2.
\end{split}
\end{equation}
Conversely, suppose that $\left\| \vecnot{v} - \vecnot{w}' \right\| \geq \left\| \vecnot{v} - \vecnot{w} \right\|$ for every $\vecnot{w}' \in W$.
From \eqref{equation:OrthogonalVector}, we get
\begin{equation*}
2 \Real \left\langle \vecnot{v} - \vecnot{w} | \vecnot{w} - \vecnot{w}' \right\rangle
+ \left\| \vecnot{w} - \vecnot{w}' \right\|^2 \geq 0
\end{equation*}
for all $\vecnot{w}' \in W$.
Note that every vector in $W$ can be expressed as $\vecnot{w} - \vecnot{w}'$ where $\vecnot{w}' \in W$, it follows that
\begin{equation} \label{equation:InequalityOrthogonalVectors}
2 \Real \left\langle \vecnot{v} - \vecnot{w} | \vecnot{w}'' \right\rangle
+ \left\| \vecnot{w}'' \right\|^2 \geq 0
\end{equation}
for every $\vecnot{w}'' \in W$.
If $\vecnot{w}'$ is in $W$ and $\vecnot{w}' \neq \vecnot{w}$ then we may take
\begin{equation*}
\vecnot{w}'' = - \frac{ \left\langle \vecnot{v} - \vecnot{w} | \vecnot{w} - \vecnot{w}' \right\rangle }{ \left\| \vecnot{w} - \vecnot{w}' \right\|^2 } \left( \vecnot{w} - \vecnot{w}' \right).
\end{equation*}
Inequality~\eqref{equation:InequalityOrthogonalVectors} then reduces to the statement
\begin{equation*}
- 2 \frac{ \left| \left\langle \vecnot{v} - \vecnot{w} | \vecnot{w} - \vecnot{w}' \right\rangle \right|^2 }{ \left\| \vecnot{w} - \vecnot{w}' \right\|^2 }
+ \frac{ \left| \left\langle \vecnot{v} - \vecnot{w} | \vecnot{w} - \vecnot{w}' \right\rangle \right|^2 }{ \left\| \vecnot{w} - \vecnot{w}' \right\|^2 }
\geq 0.
\end{equation*}
This inequality holds if and only if $\left\langle \vecnot{v} - \vecnot{w} | \vecnot{w} - \vecnot{w}' \right\rangle = 0$.
Therefore, $\vecnot{v} - \vecnot{w}$ is orthogonal to every vector in $W$.
Hence the vector $\vecnot{w} \in W$ is a best approximation of $\vecnot{v} \in V$ by vectors in $W$ if and only if $\vecnot{v} - \vecnot{w}$ is orthogonal to every vector in $W$.

Suppose $\vecnot{w}, \vecnot{w}' \in W$ are best approximations of $\vecnot{v}$ by vectors in $W$.
Then $\left\| \vecnot{v} - \vecnot{w} \right\| = \left\| \vecnot{v} - \vecnot{w}' \right\|$ and \eqref{equation:OrthogonalVector} implies that $\left\| \vecnot{w} - \vecnot{w}' \right\| = 0$.
That is, if a best approximation exists then it is unique.

Assume that $W$ is finite-dimensional and let $\vecnot{w}_1, \ldots, \vecnot{w}_n$ be an orthonormal basis for $W$.
Furthermore, let
\begin{equation*}
\vecnot{w} = \sum_{i=1}^n \frac{ \left\langle \vecnot{v} | \vecnot{w}_i \right\rangle }{ \left\| \vecnot{w}_i \right\|^2 } \vecnot{w}_i .
\end{equation*}
Then $\vecnot{v} -\vecnot{w}$ is orthogonal to $\vecnot{w}_j$ for $j = 1, \ldots, n$, i.e.,
\begin{equation*}
\begin{split}
\left\langle \vecnot{v} - \vecnot{w} | \vecnot{w}_j \right\rangle
&= \left\langle \vecnot{v} | \vecnot{w}_j \right\rangle
- \left\langle \sum_{i=1}^n \frac{ \left\langle \vecnot{v} | \vecnot{w}_i \right\rangle }{ \left\| \vecnot{w}_i \right\|^2 } \vecnot{w}_i \Big| \vecnot{w}_j \right\rangle \\
&= \left\langle \vecnot{v} | \vecnot{w}_j \right\rangle
- \frac{ \left\langle \vecnot{v} | \vecnot{w}_j \right\rangle }{ \left\| \vecnot{w}_i \right\|^2 } \left\langle \vecnot{w}_j | \vecnot{w}_j \right\rangle
= 0.
\end{split}
\end{equation*}
That is, $\vecnot{v} - \vecnot{w}$ is orthogonal to every vector in $W$ and therefore $\vecnot{w}$ is the best appoximation to $\vecnot{v}$ by vectors in $W$.
\end{proof}

\begin{definition}
Let $V$ be an inner-product space and $W$ be any set of vectors in $V$.
The \emph{orthogonal complement} of $W$ denoted by $W^{\bot}$ is the set of all vectors in $V$ which are orthogonal to every vector in $W$.
\end{definition}

\begin{problem} \label{problem:WbotSubspace}
Let $W$ be any subset of vector space $V$.
Show that $W^{\bot}$ is a subspace of $V$.
\end{problem}
\noindent
\textbf{S~\ref{problem:WbotSubspace}}.
Let $\vecnot{m}_1, \vecnot{m}_2 \in W^{\bot}$ and $s \in F$.
For any vector $\vecnot{w} \in W$, we have
\begin{equation*}
\left\langle \vecnot{m}_1 | \vecnot{w} \right\rangle
= \left\langle \vecnot{m}_2 | \vecnot{w} \right\rangle
= 0.
\end{equation*}
This implies
\begin{equation*}
\left\langle s \vecnot{m}_1 + \vecnot{m}_2 | \vecnot{w} \right\rangle
= s \left\langle \vecnot{m}_1 | \vecnot{w} \right\rangle
+ \left\langle \vecnot{m}_2 | \vecnot{w} \right\rangle
= 0.
\end{equation*}
That is, $s \vecnot{m}_1 + \vecnot{m}_2 \in W^{\bot}$.
Hence, $W^{\bot}$ is a subspace of $V$.

\begin{definition}
Whenever the vector $\vecnot{w}$ in Theorem~\ref{theorem:OrthogonalProjection} exists, it is called the \emph{orthogonal projection of $\vecnot{v}$ on $W$}.
If every vector in $V$ has an orthogonal projection on $W$, the mapping that assigns to each vector in $V$ its orthogonal projection on $W$ is called the \emph{orthogonal projection of $V$ on $W$}.
\end{definition}

\begin{problem}
Let $W$ be the subspace of $\RealNumbers^2$ spanned by the vector $(1,2)$.
Using the standard inner product, let $E$ be the orthogonal projection of $\RealNumbers^2$ onto $W$.
Find
\begin{enumerate}
\item a formula for $E(x_1, x_2)$
\item the matrix of $E$ in the standard ordered basis, i.e., $E(x_1, x_2) = E \vecnot{x}$
\item $W^{\bot}$
\item an orthonormal basis in which $E$ is represented by the matrix
\begin{equation*}
E = \left[ \begin{array}{cc} 1 & 0 \\ 0 & 0 \end{array} \right].
\end{equation*}
\end{enumerate}
\end{problem}

\begin{corollary}
Let $V$ be an inner-product space, $W$ be a finite-dimensional subspace, and $E$ be the orthogonal projection of $V$ on $W$.
Then the mapping
\begin{equation*}
\vecnot{v} \mapsto \vecnot{v} - E \vecnot{v}
\end{equation*}
is the orthogonal projection of $V$ on $W^{\bot}$.
\end{corollary}
\begin{proof}
Let $\vecnot{v}$ be any vector in $V$.
Then, $\vecnot{v} - E\vecnot{v}$ is in $W^{\bot}$, and for any $\vecnot{u}$ in $W^{\bot}$, $\vecnot{v} - \vecnot{u} = E \vecnot{v} + \left( \vecnot{v} - E \vecnot{v} - \vecnot{u} \right)$.
Since $E \vecnot{v} \in W$ and $\vecnot{v} - E \vecnot{v} - \vecnot{u} \in W^{\bot}$, it follows that
\begin{equation*}
\begin{split}
\left\| \vecnot{v} - \vecnot{u} \right\|^2 &= \left\| E \vecnot{v} \right\|^2 + \left\| \vecnot{v} - E \vecnot{v} - \vecnot{u} \right\|^2 \\
&\geq \left\| \vecnot{v} - \left( \vecnot{v} - E \vecnot{v} \right) \right\|^2
\end{split}
\end{equation*}
with strict inequality when $\vecnot{u} \neq \vecnot{v} - E \vecnot{v}$.
Thus, $\vecnot{v} - E \vecnot{v}$ is the best approximation to $\vecnot{v}$ by vectors in $W^{\bot}$.
\end{proof}

\begin{theorem} \label{theorem:OrthogonalSubspaceDirectSum}
Suppose $V$ is an inner-product space.
Let $W$ be a finite-dimensional subspace of $V$ and let $E$ denote the orthogonal projection of $V$ on $W$.
Then $E$ is an idempotent linear transformation of $V$ onto $W$, $W^{\bot}$ is the nullspace of $E$, and
\begin{equation*}
V = W \oplus W^{\bot}.
\end{equation*}
\end{theorem}
\begin{proof}
Let $\vecnot{v}$ be any vector in $V$.
Then $E \vecnot{v}$ is the best approximation of $\vecnot{v}$ by vectors in $W$.
If $\vecnot{v} \in W$ then $E \vecnot{v} = \vecnot{v}$.
It follows that $E \left( E \vecnot{v} \right) = E \vecnot{v}$ for any $\vecnot{v} \in V$ since $E \vecnot{v} \in W$.
That is, $E^2 = E$ and $E$ is idempotent.

To show that $E$ is a linear transformation, consider vectors $\vecnot{v}_1, \vecnot{v}_2 \in V$ and scalar $s \in F$.
Then $\vecnot{v}_1 - E \vecnot{v}_1$ and $\vecnot{v}_2 - E \vecnot{v}_2$ are each orthogonal to every vector in $W$.
The vector
\begin{equation*}
s \left( \vecnot{v}_1 - E \vecnot{v}_1 \right) + \left( \vecnot{v}_2 - E \vecnot{v}_2 \right) = \left( s \vecnot{v}_1 + \vecnot{v}_2 \right) - \left( s E \vecnot{v}_1 + E \vecnot{v}_2 \right)
\end{equation*}
is therefore also orthogonal to every vector in $W$.
Since $s E \vecnot{v}_1 + E \vecnot{v}_2$ is a vector in $W$, it follows from Theorem~\ref{theorem:OrthogonalProjection} that
\begin{equation*}
E \left( s \vecnot{v}_1 + \vecnot{v}_2 \right) = s E \vecnot{v}_1 + E \vecnot{v}_2.
\end{equation*}
That is, $E$ is a linear transformation.

Again, let $\vecnot{v} \in V$.
Then $E \vecnot{v}$ is the unique vector in $W$ such that $\vecnot{v} - E \vecnot{v}$ is in $W^{\bot}$.
In particular, $E \vecnot{v} = \vecnot{0}$ when $\vecnot{v} \in W^{\bot}$.
Conversely, if $E \vecnot{v} = \vecnot{0}$ then $\vecnot{v} \in W^{\bot}$.
Thus $W^{\bot}$ is the nullspace of $E$.
The equation
\begin{equation*}
\vecnot{v} = E \vecnot{v} + \vecnot{v} - E \vecnot{v}
\end{equation*}
shows that $V = W + W^{\bot}$.
Furthermore, $W \cap W^{\bot} = \left\{ \vecnot{0} \right\}$.
Hence $V$ is the direct sum of $W$ and $W^{\bot}$.
\end{proof}

\begin{corollary}
Let $W$ be a finite-dimensional subspace of an inner-product space $V$, and let $E$ be the orthogonal projection of $V$ on $W$.
Then $I - E$ is the orthogonal projection of $V$ on $W^{\bot}$.
It is an idempotent linear transformation of $V$ onto $W^{\bot}$ with nullspace $W$.
\end{corollary}

Theorem~\ref{theorem:OrthogonalSubspaceDirectSum} also implies the following result, known as Bessel's inequality.

\begin{corollary}
Let $\vecnot{v}_1, \ldots, \vecnot{v}_n$ be an orthogonal set of distinct, non-zero vectors in an inner-product space $V$.
If $\vecnot{v} \in V$ then
\begin{equation*}
\sum_{i=1}^n \frac{ \left| \left\langle \vecnot{v} | \vecnot{v}_i \right\rangle \right|^2 }{ \left\| \vecnot{v}_i \right\|^2 }
\leq \left\| \vecnot{v} \right\|^2.
\end{equation*}
Moreover, equality holds if and only if
\begin{equation*}
\vecnot{v} = \sum_{i=1}^n \frac{ \left\langle \vecnot{v} | \vecnot{v}_i \right\rangle }{ \left\| \vecnot{v}_i \right\|^2 } \vecnot{v}_i.
\end{equation*}
\end{corollary}
\begin{proof}
Define
\begin{equation*}
\vecnot{w} = \sum_{i=1}^n \frac{ \left\langle \vecnot{v} | \vecnot{v}_i \right\rangle }{ \left\| \vecnot{v}_i \right\|^2 } \vecnot{v}_i.
\end{equation*}
Then $\vecnot{v} = \vecnot{w} + \vecnot{u}$, where $\left\langle \vecnot{w} | \vecnot{u} \right\rangle = 0$ and $\left\| \vecnot{v} \right\|^2 = \left\| \vecnot{w} \right\|^2 + \left\| \vecnot{u} \right\|^2$.
Noting that
\begin{equation*}
\left\| \vecnot{w} \right\|^2
= \sum_{i=1}^n \frac{ \left| \left\langle \vecnot{v} | \vecnot{v}_i \right\rangle \right|^2 }{ \left\| \vecnot{v}_i \right\|^2 },
\end{equation*}
we obtain the desired result.
\end{proof}

If $\vecnot{v}_1, \ldots, \vecnot{v}_n$ is an orthonormal set, Bessel's inequality states that
\begin{equation*}
\sum_{i=1}^n \left| \left\langle \vecnot{v} | \vecnot{v}_i \right\rangle \right|^2 \leq \left\| \vecnot{v} \right\|^2.
\end{equation*}
It follows that the vector $\vecnot{v}$ is in the subspace spanned by $\vecnot{v}_1, \ldots, \vecnot{v}_n$ if an only if
\begin{equation*}
\vecnot{v} = \sum_{i=1}^n \left\langle \vecnot{v} | \vecnot{v}_i \right\rangle \vecnot{v}_i.
\end{equation*}


\section{Banach and Hilbert Spaces}

\begin{definition}
A complete normed vector space is called a \emph{Banach space}.
A complete inner-product space is called a \emph{Hilbert space}.
\end{definition}

\begin{definition}
Recall that a subset $\left\{ \vecnot{v}_{\alpha} | \alpha \in A \right\}$ of a Hilbert space $V$ is said to be orthonormal if $\left\| \vecnot{v}_{\alpha} \right\| = 1$ for every $\alpha \in A$ and $\left\langle \vecnot{v}_{\alpha} | \vecnot{v}_{\beta} \right\rangle = 0$ for all $\alpha \neq \beta$.
If the subspace spanned by the family $\left\{ \vecnot{v}_{\alpha} | \alpha \in A \right\}$ is dense in $V$, we call this set an \emph{orthonormal basis}.
\end{definition}

Note that, according to this definition, an orthonormal basis for a Hilbert space $V$ is not necessarily a basis for $V$.
However, it can be shown that any orthogonal basis is a subset of a basis.
In practice it is the orthonormal basis, not the basis itself, which is of most use.
None of these issues arise in finite-dimensional spaces, where an orthogonal basis is always a basis.

Let $\mathcal{B} = \left\{ \vecnot{v}_{\alpha} | \alpha \in A \right\}$ be an orthonormal basis for Hilbert space $V$.
Each element $\vecnot{v} \in V$ has the form
\begin{equation*}
\vecnot{v} = \sum_{\alpha \in A} s_{\alpha} \vecnot{v}_{\alpha},
\end{equation*}
where the sum converges in the (2)-norm.
The \emph{Parseval identity}
\begin{equation*}
\left\| \vecnot{v} \right\|^2 = \sum_{\alpha \in A} | s_{\alpha} |^2
\end{equation*}
is obtained by computig $\left\langle \vecnot{v} | \vecnot{v} \right\rangle$.

\begin{theorem}
Every orthogonal set in a Hilbert space $V$ can be enlarged to an orthonormal basis for $V$.
\end{theorem}
\begin{proof}
Let $X$ be the set of orthonormal subsets of $V$.
Furthermore, for $x, y \in X$ consider the strict partial order defined by proper inclusion.
If $x = \left\{ \vecnot{v}_{\alpha} | \alpha \in A_0 \right\}$ is an element of $X$, then by the maximum principle there exists a maximal simply ordered subset $Z$ of $X$ containing $x$.
This shows the existence of a maximal orthonormal set $\left\{ \vecnot{v}_{\alpha} | \alpha \in A \right\}$, where $A_0 \subset A$.

Let $W$ be the closed subspace of $V$ generated by $\left\{ \vecnot{v}_{\alpha} | \alpha \in A \right\}$.
If $W \neq V$, there is a unit vector $\vecnot{u} \in W^{\bot}$, contradicting the maximality of the system $\left\{ \vecnot{v}_{\alpha} | \alpha \in A \right\}$.
Thus, $W = V$ and we have an orthonormal basis.
\end{proof}
