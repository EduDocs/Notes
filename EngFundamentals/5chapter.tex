\chapter{Matrix Properties and Factorizations}

A useful analogy between matrices and complex numbers is as follows.
\begin{itemize}
\item \emph{Hermitian matrices} satisfying $A^H = A$ are analogous to real numbers, whose complex conjugates are equal to themselves.
\item \emph{Unitary matrices} satisfying $U^H U = I$ are analogous to complex numbers on the unit circle, satisfying $\bar{z}z = 1$.
\item \emph{Orthogonal matrices} satisfying $Q^TQ = I$ are analogous to the real numbers $z = \pm 1$, such that $z^2 = 1$.
\end{itemize}

The transformation
\begin{equation*}
z = \frac{1 + jr}{1 - jr}
\end{equation*}
maps real number $r$ into the unit circle $|z| = 1$.
Analogously, by \emph{Cayley's formula},
\begin{equation*}
U = (I + jR) (I - jR)^{-1},
\end{equation*}
a Hermitian matrix $R$ is mapped to a unitary matrix.


\section{Matrix Factorization}

\subsubsection{LU decomposition}

A square matrix $A \in F^{n \times n}$ can be factored as $A = LU$, where $L$ is a lower-triangular matrix with ones on the main diagonal and $U$ is an upper-triangular matrix.

\begin{gather*}
\left[ \begin{array}{ccc} 1 & 1 & 1 \\ 1 & 2 & 4 \\ 1 & 3 & 9 \end{array} \right] \\
\left[ \begin{array}{ccc} 1 & 0 & 0 \\ -1 & 1 & 0 \\ -1 & 0 & 1 \end{array} \right]
\left[ \begin{array}{ccc} 1 & 1 & 1 \\ 1 & 2 & 4 \\ 1 & 3 & 9 \end{array} \right]
= \left[ \begin{array}{ccc} 1 & 1 & 1 \\ 0 & 1 & 3 \\ 0 & 2 & 8 \end{array} \right] \\
\left[ \begin{array}{ccc} 1 & 0 & 0 \\ 0 & 1 & 0 \\ 0 & -2 & 1 \end{array} \right]
\left[ \begin{array}{ccc} 1 & 0 & 0 \\ -1 & 1 & 0 \\ -1 & 0 & 1 \end{array} \right]
\left[ \begin{array}{ccc} 1 & 1 & 1 \\ 1 & 2 & 4 \\ 1 & 3 & 9 \end{array} \right]
= \left[ \begin{array}{ccc} 1 & 1 & 1 \\ 0 & 1 & 3 \\ 0 & 0 & 2 \end{array} \right]
\end{gather*}

\begin{gather*}
\left[ \begin{array}{ccc} 1 & 0 & 0 \\ -1 & 1 & 0 \\ -1 & 0 & 1 \end{array} \right]
\left[ \begin{array}{ccc} 1 & 1 & 1 \\ 1 & 2 & 4 \\ 1 & 3 & 9 \end{array} \right]
= \left[ \begin{array}{ccc} 1 & 0 & 0 \\ 0 & 1 & 0 \\ 0 & 2 & 1 \end{array} \right]
\left[ \begin{array}{ccc} 1 & 1 & 1 \\ 0 & 1 & 3 \\ 0 & 0 & 2 \end{array} \right] \\
\left[ \begin{array}{ccc} 1 & 1 & 1 \\ 1 & 2 & 4 \\ 1 & 3 & 9 \end{array} \right]
= \left[ \begin{array}{ccc} 1 & 0 & 0 \\ 1 & 1 & 0 \\ 1 & 0 & 1 \end{array} \right]
\left[ \begin{array}{ccc} 1 & 0 & 0 \\ 0 & 1 & 0 \\ 0 & 2 & 1 \end{array} \right]
\left[ \begin{array}{ccc} 1 & 1 & 1 \\ 0 & 1 & 3 \\ 0 & 0 & 2 \end{array} \right] \\
\left[ \begin{array}{ccc} 1 & 1 & 1 \\ 1 & 2 & 4 \\ 1 & 3 & 9 \end{array} \right]
= \left[ \begin{array}{ccc} 1 & 0 & 0 \\ 1 & 1 & 0 \\ 1 & 2 & 1 \end{array} \right]
\left[ \begin{array}{ccc} 1 & 1 & 1 \\ 0 & 1 & 3 \\ 0 & 0 & 2 \end{array} \right] \\
\end{gather*}


\subsubsection{Cholesky decomposition}

A Hermitian positive-definite matrix $A \in F^{n \times n}$ can be factored as
\begin{equation*}
A = LL^H,
\end{equation*}
where $L$ is lower triangular.

\subsubsection{QR decomposition}

A matrix $A \in F^{m \times n}$ can be factored as
\begin{equation*}
A = Q R,
\end{equation*}
where $Q$ is a unitary matrix, $QQ^H = I$, and $R$ is an upper-triangular matrix.

