\chapter*{Preface}

This book is an outgrowth of a first year graduate course for engineers
that aims to develop intuition and analaysis in general linear spaces.
It is also intended to start graduate students on the road to mathematical
maturity.

The ideas are divided into three main parts:

Both engineering intuition and facility with writing proofs can be developed with practice.

Proofs are used to validate intuitive ideas about engineering systems.

Like music and art, theoretical engineering is learned by doing, not just by reading texts and listening to lectures.

Doing the exercises is the best way to get a feel for the material, test your understanding and see what needs further study.
It is important that you try to do the exercises by yourself prior to discussing with other students or your instructor.
Once solved, problems often appear much easier than they really are.
Testing your knowledge during exams is a dangerous proposition.

In this book we decided to take a good thing and make it better, adding the symbol $\triangle$ to the end of a definition, the symbol $\diamond$ for the end of an example.

Physical sloppiness is often a sign of either laziness or disrespect, and sloppiness in writing style is often a mask for sloppy thinking.


Functional knowledge,
but can you prove it?

A statement is something that can be verified, it has to be either true or false.
We will be making two assumptions when dealing with statements: every statement is either true or false, and no statement is both true and false.
One of the consequences of this law is that if a statement is not false, then it must be true.
Hence, to prove that something is true, it would suffice to show that it is not false.

"Write a computer program."
"Eat a pineapple"

What makes statement valuable for our purposes is that there are a number of ways of forming new statements from old ones.

In literary writing, some measure of ambiguity is often acceptable, and sometime valuable.
In mathematics, by contrast, precision is key; ambiguity is to be avoided at all cost.
When using a mathematical term, always stick to the precise mathematical definition, regardless of any other colloquial usage.



