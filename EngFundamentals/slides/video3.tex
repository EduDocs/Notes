\documentclass[10pt,english,aspectratio=169]{beamer}
% Use notes or hide notes or show only notes or handout


\usetheme{default}

\usepackage{xstring}
\usepackage{pgfpages}
%\makeatletter
%\IfSubStr{\@classoptionslist}{handout}
%  {\pgfpagesuselayout{2 on 1}[letterpaper,border shrink=5mm]}
%  {}
%\makeatother

\usepackage{amsmath,amssymb,amsthm}
\usepackage{stmaryrd}
\usepackage{enumerate}
\usepackage{stfloats}
\usepackage{bbm}
\usepackage{pdfpages}
\usepackage{framed}

\usepackage[most]{tcolorbox}
\tcbset{highlight math style={enhanced,
  colframe=white,colback=yellow!15,arc=8pt,boxrule=1pt,
  }}
  
\usepackage{tikz,pgf,pgfplots}
\usepackage{algorithm,algorithmic}
\usepgflibrary{shapes}
\usetikzlibrary{%
  arrows,%
  arrows.meta,
  shapes.misc,% wg. rounded rectangle
  shapes.arrows,%
  shapes,%
  calc,%
  chains,%
  matrix,%
  positioning,% wg. " of "
  scopes,%
  decorations.pathmorphing,% /pgf/decoration/random steps | erste Graphik
  shadows,%
  backgrounds,%
  fit,%
  petri,%
  quotes
}

\setbeamersize{text margin left=10mm,text margin right=35mm}

\pgfplotsset{compat=1.12}

%\usetheme{Frankfurt}
%\usecolortheme{ldpc}
\useinnertheme{rounded}
\usecolortheme{whale}
\usecolortheme{orchid}

\newcommand{\ul}[1]{\underline{#1}}
\renewcommand{\Pr}{\mathbb{P}}

%% Setup slides and notes
\makeatletter
\IfSubStr{\@classoptionslist}{notes} { \IfSubStr{\@classoptionslist}{hide} {}{\IfSubStr{\@classoptionslist}{only} {}{\setbeameroption{show notes on second screen=right}}} }{}
\makeatother
%\setbeamertemplate{note page}{\pagecolor{yellow!5}\vfill\insertnote\vfill}

\newcommand{\getpdfpages}[2]{\begingroup
  \setbeamercolor{background canvas}{bg=}
  \addtocounter{framenumber}{1}
  \includepdf[pages={#1},%
  pagecommand={%
    \expandafter\def\expandafter\insertshorttitle\expandafter{%
      \insertshorttitle\hfill\insertframenumber\,/\,\inserttotalframenumber}}%
  ]{#2}
  \endgroup}

\newcommand{\backupbegin}{
   \newcounter{finalframe}
   \setcounter{finalframe}{\value{framenumber}}
}
\newcommand{\backupend}{
   \setcounter{framenumber}{\value{finalframe}}
}

 \setbeamercolor{bibliography entry author}{fg=black}
 \setbeamercolor{bibliography entry title}{fg=black}
 \setbeamercolor{bibliography entry location}{fg=black}
 \setbeamercolor{bibliography entry note}{fg=black}
 
 \setbeamerfont{bibliography item}{size=\footnotesize}
 \setbeamerfont{bibliography entry author}{size=\footnotesize}
 \setbeamerfont{bibliography entry title}{size=\footnotesize}
 \setbeamerfont{bibliography entry location}{size=\footnotesize}
 \setbeamerfont{bibliography entry note}{size=\footnotesize}
 \setbeamertemplate{bibliography item}{\insertbiblabel}
 
\newlength\tikzwidth
\newlength\tikzheight


\newcommand{\mc}[1]{\mathcal{#1}}
\newcommand{\mbb}[1]{\mathbb{#1}}
\newcommand{\expt}{\mbb{E}}
\newcommand{\dd}{\mathrm{d}}

\def\checkmark{\tikz\fill[scale=0.4](0,.35) -- (.25,0) -- (1,.7) -- (.25,.15) -- cycle;}
\def\greencheck{{\color{green}\checkmark}}
\def\scalecheck{\resizebox{\widthof{\checkmark}*\ratio{\widthof{x}}{\widthof{\normalsize x}}}{!}{\checkmark}}
\def\xmark{\tikz [x=1.4ex,y=1.4ex,line width=.2ex, red] \draw (0,0) -- (1,1) (0,1) -- (1,0);}
\def\redx{{\color{red}\xmark}}

\renewcommand{\footnotesep}{-2pt}

\newif\ifslow
\slowtrue

\newcommand{\vecnot}[1]{#1}

\begin{document}

\ifslow

\title{ECE 586: Vector Space Methods \\ Lecture 3 Flip Video: Set Theory}
\author{Henry D. Pfister \\ Duke University}
\date{}
%\date{August 20th, 2020}
%\maketitle

\setbeamertemplate{navigation symbols}{}

\begin{frame}[plain]
	\titlepage
	
	\note{
		\vspace{8mm}
		\begin{enumerate}
			\setlength\itemsep{3mm}
			\color{red}
			\item Welcome to the third video lecture for ECE 586, Vector Space Methods. \\[2mm]
			Today, we'll discuss set theory.
		\end{enumerate}
	}
\end{frame}

\addtocounter{framenumber}{-1}
\setbeamertemplate{navigation symbols}{\textcolor{blue}{\footnotesize \insertframenumber ~/ \inserttotalframenumber}}


\begin{frame}<1-9> \frametitle{1.4: Set Theory}

\begin{itemize}
\setlength\itemsep{3mm}
\item<1-> is the \textcolor{blue}{foundation} (along with logic) of all modern mathematics \vspace{1mm}
\begin{itemize} 
  \setlength\itemsep{1.5mm}
  \item<2-> Numbers, relations, functions, \ldots are all defined using set theory
  \item<3-> But, it's not as simple as one would hope because \textcolor{blue}{naive approaches are inconsistent} (i.e., for some $P$, the definitions imply $P$ and $\neg P$)
  \item<4-> Axiomatic approaches avoid contradictions but are overly complicated
  \item<5-> Thus, we use \textcolor{blue}{naive set theory}, which defines the operations of set theory without worrying about paradoxes. This is sufficient for engineering math.
\end{itemize}

\item<6-> Naive Set Theory \vspace{1mm}
\begin{itemize} 
  \setlength\itemsep{1.5mm}
  \item<6-> \textcolor{blue}{Set defined as ``any collection of objects, mathematical or otherwise''}
  \item<6-> Ex. Consider ``the set of all books published in 2007''
  \item<7-> Objects in a set are called \textcolor{blue}{elements} or \textcolor{blue}{members} of the set
  \item<8-> Logical statement: ``$a$ is a member of the set $A$'' is written \textcolor{blue}{$a \in A$}
  \item<9-> Its negation: ``$a$ is not a member of the set $A$'' is written \textcolor{blue}{$a \notin A$}
\end{itemize}
\end{itemize}

\note{
	\vspace{2mm}
	\begin{enumerate}[<alert@+>]
	\footnotesize
	\setlength\itemsep{2mm}
	\item Read.
	\end{enumerate}
}

\end{frame}

\begin{frame}<1-7> \frametitle{1.4: Using Set Theory}

\begin{itemize}
\setlength\itemsep{2.5mm}
\item<1-> Defining Sets \vspace{1mm}
\begin{itemize} 
  \setlength\itemsep{1.5mm}
  \item<1-> One can \textcolor{blue}{present a set by listing elements}: standard English vowels \[A= \{ a,e,i,o,u \} \]
  \item<2-> Element \textcolor{blue}{order is irrelevant}: $\{ i,o,u,a,e \}$ is the same as $A$ \\ Repeated elements have no effect: $\{ a,e,i,o,u,e,o \}$ same as $A$
  \item<3-> A \textcolor{blue}{singleton} is a set containing exactly one element such as $\{a\}$
  \item<4-> Standard sets: Integers $\mathbb{Z}$, Real numbers $\mathbb{R}$, and Complex numbers $\mathbb{C}$
  %\item It is possible to construct each of these sets in a rigorous manner. Instead, we assume their meaning is intuitively clear
\end{itemize}

\item<5-> Building new sets from old \vspace{1mm}
\begin{itemize} 
  \setlength\itemsep{1mm}
  \item \textcolor{blue}{Set-builder notation}: For logical predicate $P(x)$ defined on $x\in X$,\\  ``$A$ is the set of elements in $X$ such that $P(x)$ is true'' is denoted by
\[ A = \{ x\in X | P(x) \} \]
  \item<6-> If no $x\in X$ satisfies $P(x)$, then result is the \textcolor{blue}{empty set $\emptyset$}
  \item<7-> Ex. natural numbers $\mathbb{N}$ and positive prime integers $P$:
\begin{align*}
  \mathbb{N} &= \{ x\in \mathbb{Z} | x\geq 1 \} = \{ 1,2,3,4,\ldots \} \\
  P &= \{ x\in \mathbb{Z} | x\geq 1 \textrm{ and ``}x\textrm{ is prime''} \} = \{ 2,3,5,7,11,\ldots \}
\end{align*}
\end{itemize}
\end{itemize}

\note{
	\vspace{2mm}
	\begin{enumerate}[<alert@+>]
	\footnotesize
	\setlength\itemsep{2mm}
	\item Read.
	\end{enumerate}
}

\end{frame}


\iffalse
\begin{frame}{Intervals?}

\begin{itemize}
\setlength\itemsep{3mm}
\item<1-> Defining Sets \vspace{1mm}
\begin{itemize} 
  \setlength\itemsep{1.5mm}


\end{itemize}
\end{itemize}
\end{frame}
\fi

\begin{frame}<1-5> \frametitle{1.4: Set Properties}

\begin{itemize}
\setlength\itemsep{3mm}
%\item<1-> Standard notation for real-interval subsets:
%\begin{align*}
%\textrm{Open interval:} \; &  (a,b) \triangleq \{ x\in \mathbb{R} | a<x<b \} \\
%\textrm{Closed interval:} \; & [a,b] \;\;\!\! \triangleq \{ x\in \mathbb{R} | a \leq x \leq b \} \\
%\textrm{Half-open intervals:} \; & (a,b] \;\! \triangleq \{ x\in \mathbb{R} | a < x \leq b \} \\
%& [a,b) \;\! \triangleq \{ x\in \mathbb{R} | a \leq x < b \}
%\end{align*}

\item<1-> Cardinality \vspace{1mm}
\begin{itemize} 
  \setlength\itemsep{2mm}
  \item<1-> For set $A$, the \textcolor{blue}{cardinality} $|A|$ is the number of elements in $A$
  \[ |\{ a,e,i,o,u \}| = 5 \]
  \item<2-> If there is a one-to-one correspondence between $A$ and the natural numbers $\mathbb{N}$, then $A$ is called \textcolor{blue}{countably infinite} and  $|A| = \infty$
  \item<3-> Ex. Set of \textcolor{blue}{rational numbers $\mathbb{Q} \triangleq \{ q \in \mathbb{R} \,|\, \exists n \in \mathbb{N}, nq \in \mathbb{Z} \}$} is countably infinite (e.g., list rationals $m/n$ with $|m|\leq n^2$)
  \item<4-> If $|A| = \infty$ but not countably infinite, then $A$ is \textcolor{blue}{uncountably infinite}
  \item<5-> Ex. Real numbers are uncountably infinite by Cantor's diagonal argument \\ (not covered in this class but worth googling if you haven't seen it)
  
\end{itemize}
\end{itemize}

\note{
	\vspace{2mm}
	\begin{enumerate}[<alert@+>]
	\footnotesize
	\setlength\itemsep{2mm}
	\item Read.
	\end{enumerate}
}

\end{frame}

% Definition of circles
\def\firstcircle{(0,0) circle (1.5cm)}
\def\secondcircle{(0:2cm) circle (1.5cm)}

\colorlet{circle edge}{blue!50}
\colorlet{circle area}{blue!20}

\tikzset{filled/.style={fill=circle area, draw=circle edge, thick},outline/.style={draw=circle edge, thick}}

\setlength{\parskip}{5mm}

\begin{frame} \frametitle{1.4: Venn Diagrams}

% Set A or B
\begin{tikzpicture}
    \draw[filled] \firstcircle node {$A$}
                  \secondcircle node {$B$};
    \node[anchor=south] at (current bounding box.north) {$A \cup B$};
\end{tikzpicture}
\hfill
% Set A and B
\begin{tikzpicture}
    \begin{scope}
        \clip \firstcircle;
        \fill[filled] \secondcircle;
    \end{scope}
    \draw[outline] \firstcircle node {$A$};
    \draw[outline] \secondcircle node {$B$};
    \node[anchor=south] at (current bounding box.north) {$A \cap B$};
\end{tikzpicture}

% Set A^c = U - A
\begin{tikzpicture}
	\draw[filled, even odd rule] (-1.55,-1.75) rectangle (3.25,1.75) \firstcircle node {$A$}; 
    \node at (2.1,1.3) {$A^c = U-A$};
    \node[blue] at (2.75,-1.2) {$U$};
\end{tikzpicture}
\hfill
% Set A but not B
\begin{tikzpicture}
    \begin{scope}
        \clip \firstcircle;
        \draw[filled, even odd rule] \firstcircle node {$A$}
                                     \secondcircle;
    \end{scope}
    \draw[outline] \firstcircle
                   \secondcircle node {$B$};
    \node[anchor=south] at (current bounding box.north) {$A - B$};
\end{tikzpicture}

\note{
	\vspace{2mm}
	\begin{enumerate}[<alert@+>]
	\footnotesize
	\setlength\itemsep{2mm}
	\item Here, we see Venn diagrams for some standard set operations.  The sets $A$ and $B$ are represented by overlapping circles.  Starting at the top left graphic, we see the union of $A$ and $B$ which contains all elements in either $A$ or $B$.  \\ [2mm] Looking at the top right graphic, we see the intersection of $A$ and $B$ which contains elements that are in both $A$ and $B$.   \\ [2mm]  Moving to the bottom right graphic, we see the set difference $A-B$ which contains all elements that are in $A$ and not in $B$.   \\ [2mm]   Finally, in the bottom left graphic, we see the complement of $A$ which contains all elements in the universal set $U$ except those in $A$.
	\end{enumerate}
}
    
\end{frame}

\begin{frame}<1-4> \frametitle{1.4: From Logic to Set Theory}

\begin{itemize}
\setlength\itemsep{3mm}
\item<1-> Operations on sets $A,B$ \vspace{1mm}
\begin{itemize} 
  \setlength\itemsep{1.5mm}
  \item<1-> \textcolor{blue}{Union} of $A$ and $B$ ($A\cup B$): set of elements in either $A$ or $B$
%This means that $A \cup B = \{ x\in A \textrm{ or } x\in B \}$ is also defined by
\[ x\in A\cup B \Leftrightarrow (x\in A)\vee (x\in B) \]
  \item<2-> \textcolor{blue}{Intersection} of $A$ and $B$ ($A\cap B$): set of elements in both $A$ and $B$
%This means that $A\cap B = \{x\in A | x\in B \}$ is also defined by
\[ x\in A\cap B \Leftrightarrow (x\in A)\wedge (x\in B) \]
  \item<3-> \textcolor{blue}{Set difference} $A-B$ (or $A \setminus \!B$): set of elements in $A$ but not in $B$
%This means that
\[ x\in A-B \Leftrightarrow (x\in A)\wedge (x\notin B) \]
  \item<4-> \textcolor{blue}{Complement} $A^c$ for implied universal set $U$ is defined by $A^c = U-A$
  \[ x\in A^c \Leftrightarrow x \notin A \]
\end{itemize}


\end{itemize}

\note{
	\vspace{2mm}
	\begin{enumerate}[<alert@+>]
	\footnotesize
	\setlength\itemsep{2mm}
	\item Read.  Logically, that means (read).  Here and below, we use the notation for equivalence because this biconditional holds for all $x$, $A$, and $B$.
	\item Read.  Logically, that means (read).
	\item Read.  Logically, that means (read).
	\item Read.  Logically, that means (read).
	
	\end{enumerate}
}

\end{frame}

\begin{frame}<1-4> \frametitle{1.4: Relationships Between Sets}

\begin{itemize}
\setlength\itemsep{3mm}
\item<1-> Relationships between sets $A,B$ \vspace{1mm}
\begin{itemize} 
  \setlength\itemsep{1.5mm}
  \item<1-> $A$ \textcolor{blue}{equals} $B$ (denoted $A=B$) if both sets have the same elements
\[ ``A=B" \Leftrightarrow \forall x \left( (x\in A)\leftrightarrow (x\in B) \right) \]
  \item<2-> $A$ is a \textcolor{blue}{subset} of $B$ (denoted $A\subseteq B$) if all elements in $A$ are also in $B$
\[ ``A\subseteq B"  \Leftrightarrow \forall x \left( (x\in A)\rightarrow (x\in B) \right) \]
  \item<3-> $A$ is a \textcolor{blue}{proper subset} of $B$ (denoted $A\subset B$) if $A\subseteq B$ and $A\neq B$ \vspace{2mm}
  \item<4-> Two sets are called \textcolor{blue}{disjoint} if $A\cap B = \emptyset$
\end{itemize}

\end{itemize}

\note{
	\vspace{2mm}
	\begin{enumerate}[<alert@+>]
	\footnotesize
	\setlength\itemsep{2mm}
	\item Read.  The ``for all $x$'' refers to all $x$ in the implied universal set.  However, for this operation, we could also write `for all $x\in A$'' without changing the meaning.
	\item Read.  The ``for all $x$'' refers to all $x$ in the implied universal set.  Again, writing ``for all $x\in A$'' wouldn't change the meaning.
	\item Read.  In this case, there must be some element in $B$ that is not in $A$.
	\item Read.
	
	\end{enumerate}
}

\end{frame}


\begin{frame}<1-3> \frametitle{1.4: De Morgan, Infinite Operations, and Negation}

\begin{itemize}
\setlength\itemsep{3mm}
\item<1-> De Morgan's $\neg (P \vee Q) = \neg P \wedge \neg Q$ with $P/Q=``x\in A/B"$ implies: $\!\!\!\!\!\!\!\!\!\!\!\!$
\begin{align*}
(A \cup B)^c & = A^c \cap B^c
%\\ (A \cap B)^c & = A^c \cup B^c,
\end{align*}

\item<2-> Infinite unions and intersections are defined by:
  \begin{align*}
  \bigcup_{\alpha \in I} S_{\alpha}
  &\triangleq \{ x | x \in S_{\alpha} \text{ for some } \alpha \in I \} \\
  \bigcap_{\alpha \in I} S_{\alpha}
  &\triangleq \{ x | x \in S_{\alpha} \text{ for all } \alpha \in I \} 
  \end{align*}  
  
\item<3-> De Morgan's identity still applies:
  \begin{align*}
  ``x \in \bigcup_{\alpha \in I} S_{\alpha}" &\Leftrightarrow ``\exists \alpha \in I, x\in S_\alpha" \\
  ``x \in \left(\bigcup_{\alpha \in I} S_{\alpha}\right)^c" &\Leftrightarrow ``\forall \alpha \in I, x \notin S_\alpha" \Leftrightarrow ``x \in \bigcap_{\alpha \in I} S^c_{\alpha}"
  \end{align*}  
  
\end{itemize}

\note{
	\vspace{0mm}
	\begin{enumerate}[<alert@+>]
	\footnotesize
	\setlength\itemsep{2mm}
	\item Read.  This is because membership in the complement of a set is logically the same as negation of membership in that set.
	\item Now, we see that infinite unions and intersections have rather intuitive and direct definitions. \\ [2mm] Let $S_\alpha$ denote a collection of sets indexed by $\alpha \in I$.  The first equation says the infinite union of the collection contains $x$ if there is some $\alpha \in I$ such that $x\in S_\alpha$.  This matches our intuition that the union contains all elements in all the sets. \\ [2mm] Similarly, the second equation says the infinite intersection of the collection contains $x$ only if $x \in S_\alpha$ for all $\alpha \in I$.  This matches our intuition that the intersection contains an element iff it is in all the sets.
	\item Now, we can apply De Morgan's identity by translating to logic, negating, and mapping back to set theory.  First, we see that $x$ is in the infinite union iff $\exists \alpha \in I, x\in S_\alpha$.  Since the logical negation of this statement is $\forall \alpha \in I, x \notin S_\alpha$, it follows that $x$ is in the complement of the infinite union iff it is in the infinite intersection of the complements of the sets in the collection.
	
	\end{enumerate}
}

\end{frame}


\begin{frame}{1.4: Foundations of Set Theory}

\begin{example}[Russell's Paradox]
Let $R = \{S|S\notin S\}$ be the set of all sets that do not contain themselves.
This set exists in naive set theory (though it may empty) simply because it is described by the above sentence.
The paradox arises from the fact that this definition leads to the logical contradiction: $R\notin R \leftrightarrow R\in R$.
\end{example}

\vspace{-3mm}

\begin{itemize}
\setlength\itemsep{3mm}
\item<2-> Sets that contain themselves in naive set theory? \vspace{2mm}
 
\item<3-> What does Russell's paradox show? \vspace{1mm}
\begin{itemize} 
  \setlength\itemsep{1.5mm}
  \item It shows that \textcolor{blue}{naive set theory is not consistent} because it allows constructions leading to contradictions
  \item It is avoided in axiomatic formulations by restricting constructions
  \item It also implies that $R$ cannot exist in any consistent set theory
  %\item Are their sets that contains themselves? In naive set theory, the answer is yes.  Some examples are the ``set of all sets'' and the ``set of all abstract ideas''
  %\item ZF axiomatic formulation has theorem: ``no set contains itself''
\end{itemize}

\item<4-> This class will use naive set theory but avoid problematic statements

\end{itemize}

\note{
	\vspace{4mm}
	\begin{enumerate}[<alert@+>]
		\footnotesize
		\setlength\itemsep{3mm}
		\item Naive set theory allows one to define any set described by a sentence.  Russell's paradox is a famous example that exposed a major problem with naive set theory.   (read) . In particular, the set-builder construction implies that, if a set does not contain itself, then it must be in $R$.  Alternatively, it must be in $R$.
		\item In naive set theory, there are sets that contains themselves.  For example, consider the ``set of all sets'' or the ``set of abstract ideas''.  But, this is not allowed in axiomatic set theory.
		\item Read.  
		\item Read.  
	\end{enumerate}
}

\end{frame}


\begin{frame} \frametitle{Next Steps}

\begin{itemize}
\setlength\itemsep{5mm}
\item To continue studying after this video -- \vspace{2mm}

\begin{itemize}
 \setlength\itemsep{3mm}
 \item Try the suggested reading: Course Notes EF 1.4
 \item Or the optional reading: PAF 3.1-3.5
 \item Also, look at the problems in Assignment 2
\end{itemize}
\end{itemize}

\note{
	\vspace{8mm}
	\begin{enumerate}
		\setlength\itemsep{3mm}
		\color{red}
		\item Here are some options to continue learning this material. (read) \\ [2mm]  That's it for today.  So, I'll see you next time.
	\end{enumerate}
}

\end{frame}

  
\backupbegin

%\begin{frame}
%\frametitle{Backup Slides}
%\begin{itemize}
%\item Slide numbers not included in denominator!
%\end{itemize}
%\end{frame}

%\begin{frame}[allowframebreaks]
%\frametitle{References}
%\bibliographystyle{alpha}
%\footnotesize
%\bibliography{IEEEabrv,WCLabrv,WCLbib,WCLnewbib}
%\end{frame}

\backupend

\end{document}
