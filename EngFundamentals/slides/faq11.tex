\documentclass[10pt,english]{article}

\usepackage{amsfonts,url}
\usepackage{amsmath}
\usepackage{amssymb}
\usepackage{hyperref}
% Paper setup
\evensidemargin=0in
\oddsidemargin=0in
\textwidth=6.25in
\topmargin=-0.5in
\headheight=0.0in
\headsep=0.5in
\textheight=9.0in
\footskip=0.5in

\newcommand{\Interior}[1]{\ensuremath{{#1}^{\circ}}}
\newcommand{\Closure}[1]{\ensuremath{\overline{#1}}}
\newcommand{\Complement}[1]{\ensuremath{{#1}^{c}}}

\newcommand{\Expect}{\ensuremath{\mathrm{E}}}
\newcommand{\vecnot}{\underline}
\newcommand{\RealNumbers}{\ensuremath{\mathbb{R}}}
\newcommand{\RationalNumbers}{\mathbb{Q}}
\newcommand{\ComplexNumbers}{\mathbb{C}}
\newcommand{\Real}{\mathrm{Re}}
\newcommand{\Span}{\mathrm{span}}
\newcommand{\Rank}{\mathrm{rank}}
\newcommand{\Nullity}{\mathrm{nullity}}
\newcommand{\Trace}{\mathrm{tr}}
\newcommand{\Diag}{\mathrm{diag}}
\newcommand{\dd}{\mathrm{d}}
\DeclareMathOperator*{\esssup}{ess\,sup}

% Use < , > inner product
\newcommand{\inner}[2]{{\left\langle #1 \mskip2mu , #2 \right\rangle}}
\newcommand{\tinner}[2]{{\langle #1 \mskip1mu , #2 \rangle}}

% Use < | > inner product
%\newcommand{\inner}[2]{{\left\langle #1 \mskip2mu \middle| \mskip2mu #2 \right\rangle}}
%\newcommand{\tinner}[2]{{\langle #1 \mskip1mu | \mskip1mu  #2 \rangle}}




\begin{document}

\title{ECE 586: Vector Space Methods \\ Lecture 11: Frequently Asked Questions}
\author{Henry D. Pfister \\ Duke University}
\date{September 25, 2020}

\maketitle

Here we give a list of questions, and their answers, that were submitted by students after watching the flip video.

\section{Linear Transformations}

\paragraph{For addition of vector spaces, if my understanding is correct, they need to have the same dimension and number of vectors right?}

To add 2 subsets $U, W$ of a vector space $V$, their elements must have the same dimension (since $U+W \triangleq \{\vecnot{u}+\vecnot{w}| \vecnot{u}\in U, \vecnot{w}\in W\}$, so you need to be able to perform $\vecnot{u}+\vecnot{w}$).
But, $U$ and $W$ need not have the same number of elements -- the definition of the addition of $U$ and $W$ says that you need to perform the sum for all possible pairs $(\vecnot{u},\vecnot{w})$ in $U\times W$.

Also, the slides don't actually consider adding vector spaces.
Instead, they define the sum of \emph{subsets of vector spaces}.
In this case, the vectors are compatible because they live in the same vector space and vector addition is already defined by that space.




\paragraph{If we use a matrix to represent null space, is it a matrix filled with zeros?}
There seems to be some confusion about what the nullspace really means.
Importantly, for a transformation $T: V \rightarrow W$, \textbf{the nullspace is a subset of V.} It is precisely the set of things in $V$ that get ``killed" by T (i.e. $N(T) \triangleq \{\vecnot{x}\in V| T(\vecnot{x}) = 0\}$).

Keeping this in mind, a matrix $B$ representing the null space of a matrix $A$ would probably contain the basis vectors of the null space as rows or columns.

\paragraph{Are vectors in ordered basis orthogonal to each other?}

Not necessarily, no.
Also, we don't yet have a notion of the dot product or orthogonality.

\paragraph{Can you please point out again what is the essential properties to prove the subspace?}

A subspace $X$ has 2 properties that you should keep in mind:
\begin{itemize}
    \item It is closed under scalar multiplication, i.e. if $\vecnot{x}\in X$, then $\forall s\in F, s\vecnot{x} \in X$ (where $F$ is a field).
    \item It is closed under vector addition, i.e. if $\vecnot{x},\vecnot{y} \in X $, then $\vecnot{x}+\vecnot{y} \in X$.
\end{itemize}

\paragraph{In slide 5, the rank and nullity are all defined on T and V. What do they do with the space W?}
Before understanding rank and nullity, it is important to understand range and nullspace.
Specifically, for a transformation $T:V \rightarrow W$, 
\begin{itemize}
    \item The range $R(T)$ \textbf{is a subset of $W$}, and rank($T$) = $\dim(R(T))$
    \item The nullspace $N(T)$ \textbf{is a subset of $V$}, and nullity($T$) = $\dim(N(T))$
\end{itemize}

\paragraph{On slide 1, does the term disjoint mean every vector in the two subspaces are linearly independent?}
Disjoint just means that the intersection of the sets contains only the \vecnot{0} vector. You can check that this implies that, if we have 2 disjoint subspaces $X,Y$, then any $\vecnot{x}\in X$ cannot be a linear combination of elements in $Y$ (else, $\vecnot{x}$ would be in $Y$, which is a contradiction since $X$ and $Y$ are disjoint by assumption).

\paragraph{I am used to the term "orthonormal basis" which means, to my knowledge, essentially the same thing as the definition given for ordered basis.  Is there a key difference here I am missing?}

``Orthonormal" requires you to define an inner product $\tinner{ u }{ v }$.
A basis $\{v_1,...,v_n\}$ is orthonormal if \[\tinner{ v_i }{ v_j} = \begin{cases} 0,~~ \text{if}~ i\neq j\\ 1, ~~ \text{if}~i=j.\end{cases} \]

\paragraph{Is there a unique coordinate vector for any vector within the space, or are there multiple solutions?}

An ordered basis $\mathcal{A}=\{v_1,...,v_n\}$ is linearly independent by definition. Now let's assume there are 2 distinct coordinate vectors for $\vecnot{x}$ relative to $\mathcal{A}$, $(a_1,...,a_n)$ and $(b_1,...,b_n)$.
Then, we can write:
\begin{align}
    a_1\vecnot{v}_1 + ... + a_n\vecnot{v}_n = b_1\vecnot{v}_1 + ... + b_n \vecnot{v}_n \Rightarrow (a_1-b_1)\vecnot{v}_1 + ... + (a_n-b_n)\vecnot{v}_n = \vecnot{0}
\end{align}
By linear independence of $\{v_1,...,v_n\}$, we get $a_i-b_i = 0 \Rightarrow a_i = b_i,~\forall i \in \{1,...,n\}$. Hence the coordinate vector of $\vecnot{x}$ relative to $\mathcal{A}$ is unique.

\paragraph{What applications are there for linear transformations?}
Common applications include linear regression, neural networks (a sequence of linear operations interspersed with nonlinearities), and Markov chains (e.g., described by a transition probability matrix $P$ which transforms state probability vector after $n$ steps to the state probability vector after $n+1$ steps).

\paragraph{Other than orthogonal complement, are there any other typical examples of useful direct sum decompositions?}

``Orthogonality" requires you to define an inner product $\tinner{ u }{ v }$, which we don't have yet.
Also, disjoint subspaces need not be orthogonal.

\paragraph{Does the vector space have to be a direct sum in order for the unique decomposition of a vector to work?}

Yes, if you want unique decomposition for all vectors.
If two subspaces contain the same vector $\vecnot{x}$, then that vector can be represented either as $\vecnot{x} + \vecnot{0}$ or as $\vecnot{0} + \vecnot{x}$.
Thus, we cannot uniquely determine which parts are in which subspace.

\end{document}
