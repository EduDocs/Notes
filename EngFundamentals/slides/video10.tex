\documentclass[10pt,english,aspectratio=169]{beamer}
% Use notes or hide notes or show only notes or handout

\usetheme{default}

\usepackage{xstring}
\usepackage{pgfpages}
%\makeatletter
%\IfSubStr{\@classoptionslist}{handout}
%  {\pgfpagesuselayout{2 on 1}[letterpaper,border shrink=5mm]}
%  {}
%\makeatother

\usepackage{amsmath,amssymb,amsthm}
\usepackage{stmaryrd}
\usepackage{enumerate}
\usepackage{stfloats}
\usepackage{bbm}
\usepackage{pdfpages}
\usepackage{framed}

\usepackage[most]{tcolorbox}
\tcbset{highlight math style={enhanced,
  colframe=white,colback=yellow!15,arc=8pt,boxrule=1pt,
  }}
  
\usepackage{tikz,pgf,pgfplots}
\usepackage{algorithm,algorithmic}
\usepgflibrary{shapes}
\usetikzlibrary{%
  arrows,%
  arrows.meta,
  backgrounds,
  shapes.misc,% wg. rounded rectangle
  shapes.arrows,%
  shapes,%
  calc,%
  chains,%
  matrix,%
  positioning,% wg. " of "
  scopes,%
  decorations.pathmorphing,% /pgf/decoration/random steps | erste Graphik
  shadows,%
  backgrounds,%
  fit,%
  petri,%
  quotes
}

\tikzset{background rectangle/.style={
    fill=white,
  },
  use background/.style={    
    show background rectangle
  }
}

\setbeamersize{text margin left=10mm,text margin right=35mm}

\pgfplotsset{compat=1.12}

%\usetheme{Frankfurt}
%\usecolortheme{ldpc}
\useinnertheme{rounded}
\usecolortheme{whale}
\usecolortheme{orchid}

\newcommand{\ul}[1]{\underline{#1}}
\renewcommand{\Pr}{\mathbb{P}}

%% Setup slides and notes
\makeatletter
\IfSubStr{\@classoptionslist}{notes} { \IfSubStr{\@classoptionslist}{hide} {}{\IfSubStr{\@classoptionslist}{only} {}{\setbeameroption{show notes on second screen=right}}} }{}
\makeatother
%\setbeamertemplate{note page}{\pagecolor{yellow!5}\vfill\insertnote\vfill}

\newcommand{\getpdfpages}[2]{\begingroup
  \setbeamercolor{background canvas}{bg=}
  \addtocounter{framenumber}{1}
  \includepdf[pages={#1},%
  pagecommand={%
    \expandafter\def\expandafter\insertshorttitle\expandafter{%
      \insertshorttitle\hfill\insertframenumber\,/\,\inserttotalframenumber}}%
  ]{#2}
  \endgroup}

\newcommand{\backupbegin}{
   \newcounter{finalframe}
   \setcounter{finalframe}{\value{framenumber}}
}
\newcommand{\backupend}{
   \setcounter{framenumber}{\value{finalframe}}
}

 \setbeamercolor{bibliography entry author}{fg=black}
 \setbeamercolor{bibliography entry title}{fg=black}
 \setbeamercolor{bibliography entry location}{fg=black}
 \setbeamercolor{bibliography entry note}{fg=black}
 
 \setbeamerfont{bibliography item}{size=\footnotesize}
 \setbeamerfont{bibliography entry author}{size=\footnotesize}
 \setbeamerfont{bibliography entry title}{size=\footnotesize}
 \setbeamerfont{bibliography entry location}{size=\footnotesize}
 \setbeamerfont{bibliography entry note}{size=\footnotesize}
 \setbeamertemplate{bibliography item}{\insertbiblabel}
 
\newlength\tikzwidth
\newlength\tikzheight


\newcommand{\mc}[1]{\mathcal{#1}}
\newcommand{\mbb}[1]{\mathbb{#1}}
%\newcommand{\expt}{\mbb{E}}
%\newcommand{\dd}{\mathrm{d}}
%\usepackage{fixltx2e}

\newcommand{\Interior}[1]{\ensuremath{{#1}^{\circ}}}
\newcommand{\Closure}[1]{\ensuremath{\overline{#1}}}
\newcommand{\Complement}[1]{\ensuremath{{#1}^{c}}}

\newcommand{\Expect}{\ensuremath{\mathrm{E}}}
\newcommand{\vecnot}{\underline}
\newcommand{\RealNumbers}{\mathbb{R}}
\newcommand{\RationalNumbers}{\mathbb{Q}}
\newcommand{\ComplexNumbers}{\mathbb{C}}
\newcommand{\Real}{\mathrm{Re}}
\newcommand{\Span}{\mathrm{span}}
\newcommand{\Rank}{\mathrm{rank}}
\newcommand{\Nullity}{\mathrm{nullity}}
\newcommand{\Trace}{\mathrm{tr}}
\newcommand{\Diag}{\mathrm{diag}}
\DeclareMathOperator*{\esssup}{ess\,sup}
\newcommand{\dd}{\mathrm{d}}



\def\checkmark{\tikz\fill[scale=0.4](0,.35) -- (.25,0) -- (1,.7) -- (.25,.15) -- cycle;}
\def\greencheck{{\color{green}\checkmark}}
\def\scalecheck{\resizebox{\widthof{\checkmark}*\ratio{\widthof{x}}{\widthof{\normalsize x}}}{!}{\checkmark}}
\def\xmark{\tikz [x=1.4ex,y=1.4ex,line width=.2ex, red] \draw (0,0) -- (1,1) (0,1) -- (1,0);}
\def\redx{{\color{red}\xmark}}

\renewcommand{\footnotesep}{-2pt}


\begin{document}

\title{ECE 586: Vector Space Methods \\ Lecture 10 Flip Video: Vector Spaces}
\author{Henry D. Pfister \\ Duke University}
\date{}
%\date{August 20th, 2020}
%\maketitle

\setbeamertemplate{navigation symbols}{}

\begin{frame}[plain]
	\titlepage
	
	\note{
		\vspace{8mm}
		\begin{enumerate}
			\setlength\itemsep{3mm}
			\color{red}
			\item Welcome to the 10th video lecture for ECE 586, Vector Space Methods. \\[2mm]
			Today, we'll discuss vector spaces.
		\end{enumerate}
	}
\end{frame}

\addtocounter{framenumber}{-1}
\setbeamertemplate{navigation symbols}{\textcolor{blue}{\footnotesize \insertframenumber ~/ \inserttotalframenumber}}


\begin{frame}<1-4> \frametitle{3.3: Vector Spaces}

\textbf{Definition:} A \textcolor{blue}{vector space} consists of the following,
\begin{enumerate}
\item<1-> a {\bf \boldmath field $F$} of \textcolor{blue}{scalars} (where $0,1 \in F$ denote the add. and mult. identities)
\item<2-> a {\bf \boldmath set $V$} of \textcolor{blue}{vectors} (which are decorated by an underline in these notes)
\item<3-> a \textbf{binary operation called vector addition}, which maps any pair of vectors $\vecnot{v}, \vecnot{w} \in V$ to a vector $\vecnot{v} + \vecnot{w} \in V$ satisfying four conditions:
\begin{enumerate}
\item vector addition is commutative: $\vecnot{v} + \vecnot{w} = \vecnot{w} + \vecnot{v}$
\item vector addition is associative: $\vecnot{u} + (\vecnot{v} + \vecnot{w}) = (\vecnot{u} + \vecnot{v}) + \vecnot{w}$
\item there is a unique vector $\vecnot{0} \in V$ such that $\vecnot{v} + \vecnot{0} = \vecnot{v}$, $\forall \vecnot{v} \in V$
\item to each $\vecnot{v} \in V$ there is a unique vector $- \vecnot{v} \in V$ such that $\vecnot{v} + (- \vecnot{v}) = \vecnot{0}$
\end{enumerate}
\item<4-> a \textbf{binary operation called scalar multiplication}, which maps any $s \in F$ and $\vecnot{v} \in V$ to a vector $s \vecnot{v} \in V$ satisfying four conditions:
\begin{enumerate}
\item the identity is multiplicative identity of the field: $1 \vecnot{v} = \vecnot{v}$, $\forall \vecnot{v} \in V$
\item scalar multiplication is associative: $(s_1 s_2) \vecnot{v} = s_1 ( s_2 \vecnot{v} )$
\item scalar multiplication distributes over vector addition: $s ( \vecnot{v} + \vecnot{w} ) = s \vecnot{v} + s \vecnot{w}$
\item scalar addition distributes scalar multiplication: $(s_1 + s_2) \vecnot{v} = s_1 \vecnot{v} + s_2 \vecnot{v}$.
\end{enumerate}
\end{enumerate}

\note{
	\vspace{2mm}
	\begin{enumerate}[<alert@+>]
		\footnotesize
		\setlength\itemsep{3mm}
		\item Let's start by defining a vector space. Read.
		\item Read.
		\item Read. 
		\item Read.  Please review these rules once carefully.
	\end{enumerate}
}

\end{frame}

\begin{frame}<1-2> \frametitle{3.3: Vector Space Examples}

\begin{example}<1-> [Standard vector space for $F^n$]
Let $F$ be a field, and let $V=F^n$ be the set of $n$-tuples $\vecnot{v} = (v_1, \ldots, v_n)$.
For $\vecnot{w} = (w_1, \ldots, w_n) \in F^n$, the sum of $\vecnot{v}$ and $\vecnot{w}$ is defined by \vspace{-1mm}
\begin{equation*}
\vecnot{v} + \vecnot{w} = (v_1 + w_1, \ldots, v_n + w_n).
\end{equation*}
The scalar product of $s \in F$ and $\vecnot{v} \in V$ is defined $s \vecnot{v} = (s v_1, \ldots, s v_n)$.
\end{example}

\begin{example}<2-> [General vector space of functions]
Let $X$ be a non-empty set and let $Y$ be a vector space over $F$.
Consider the set $V$ of all functions mapping $X$ to $Y$.
The vector addition of two functions $f,g \in V$ is the function $(f+g)\colon X \to Y$ defined by  \vspace{-1mm}
\begin{equation*}
(f + g)(x) = f(x) + g(x) \quad \forall x \in X,
\end{equation*}
where the RHS is defined by vector addition in $Y$.
The scalar product of $s \in F$ and the function $f \in V$ is the function $sf$ defined by
$(sf)(x) = s \, f(x)$ for all $x \in X$,
where the RHS is defined by scalar multiplication in $Y$.
\vspace{-1mm}
\end{example}

\note{
	\vspace{2mm}
	\begin{enumerate}[<alert@+>]
		\footnotesize
		\setlength\itemsep{3mm}
		\item A vector space is defined by choosing a set and rules for vector addition and scalar multiplication which satisfy the requirements. \\ [2mm] Read. "element-wise addition".  where each element of $\vecnot{v}$ is multiuplied by the scalar $s$.
		\item Read. "point-wise addition"
	\end{enumerate}
}

\end{frame}



\begin{frame}<1-3> \frametitle{3.3.1: Subspaces}

\begin{definition}
Let $V$ be a vector space over $F$.
A \textcolor{blue}{subspace} of $V$ is a subset $W \subset V$ which is itself a vector space over $F$.
\end{definition}

\begin{lemma}<2->
A non-empty subset $W \subset V$ is a subspace of $V$ if and only if, for every pair $\vecnot{w}_1, \vecnot{w}_2 \in W$ and every scalar $s \in F$, the vector $s \vecnot{w}_1 + \vecnot{w}_2 \in W$.
\end{lemma}

\uncover<2->{Sketch proof in live session (via inheritance from $V$).}
\vspace{2mm}

\begin{example}<3->
Let $A$ be an $m \times n$ matrix over $F$.
Then, the subset $V\subseteq F^{n \times 1}$ of vectors satisfying \textcolor{blue}{$A \vecnot{v} = \vecnot{0}$} forms a subspace.
\end{example}

\note{
	\vspace{2mm}
	\begin{enumerate}[<alert@+>]
		\footnotesize
		\setlength\itemsep{3mm}
		\item Read.  Subspaces play an important role in vector spaces.
		\item A subset of a vector space is a subspace if and only it is closed under vector addition and scalar multiplication. \\ [2mm] This lemma states this principle formally. Read.  This lemma also makes it easy to determine when a subset is a subspace.
		\item Read. Try using the Lemma to prove this is a subspace.
	\end{enumerate}
}

\end{frame}


\begin{frame}<1-3> \frametitle{3.3: Linear Combinations}

\begin{definition}<1->
A vector $\vecnot{w} \in V$ is said to be a \textcolor{blue}{linear combination} of the vectors $\vecnot{v}_1, \ldots, \vecnot{v}_n \in V$ provided that there exist scalars $s_1, \ldots, s_n \in F$ such that
\begin{equation*}
\vecnot{w} = \sum_{i=1}^n s_i \vecnot{v}_i.
\end{equation*}
\end{definition}

\begin{definition}<2->
Let $U$ be a list (or set) of vectors in $V$.
The \textcolor{blue}{span} of $U$, denoted $\Span(U)$, is defined to be the set of all finite linear combinations of vectors in $U$.
%The \defn{vector space}{subspace spanned} by $U$ is defined to be the intersection $W$ of all subspaces of $V$ which contain $U$.
\end{definition}

\begin{example}<3->
For a vector space $V$, the span of any list (or set) of vectors in $V$ forms a subspace.
\end{example}

\note{
	\vspace{2mm}
	\begin{enumerate}[<alert@+>]
		\footnotesize
		\setlength\itemsep{3mm}
		\item Read. It's likely you've all heard this term before for real vectors.
		\item Read. Note that we only consider finite linear combinations at this point because we don't yet have a topology that allows us to discuss convergence of an infinite linear combination.
		\item Read. This also follows easily from the lemma on the previous slide.
	\end{enumerate}
}

\end{frame}

\begin{frame}<1-2> \frametitle{3.3.2: Linear Dependence and Independence}

\begin{definition}<1->
Let $V$ be a vector space over $F$.
A list of vectors $\vecnot{u}_1, \ldots, \vecnot{u}_n \in V $ is called \textcolor{blue}{linearly dependent} if there are scalars $s_1, \ldots, s_n \in F$, not all equal to $0$, such that
\begin{equation*}
\sum_{i=1}^n s_i \vecnot{u}_i = \vecnot{0}.
\end{equation*}
A list that is not linearly dependent is called \textcolor{blue}{linearly independent}. \\[1.5mm]
Similarly, a subset $U \subset V$ is called linearly dependent if there is a finite list $\vecnot{u}_1, \ldots, \vecnot{u}_n \in U$ of distinct vectors that is linearly dependent.
Otherwise, it is called linearly independent.
\end{definition}

\begin{example}<2->
For $V=\RealNumbers^4$, vectors $\vecnot{v}_1=(1,1,0,0)$, $\vecnot{v}_2=(0,1,1,0)$, $\vecnot{v}_3=(0,0,1,1)$ are linearly independent because $\vecnot{u} = s_1 \vecnot{v}_1 + s_2  \vecnot{v}_2 + s_3  \vecnot{v}_3 \neq \vecnot{0}$ unless $s_1 \!=\! s_2 \!=\! s_3 \!=\! 0$ (e.g., $u_1 \neq 0$ if $s_1 \neq 0$, $u_4 \neq 0$ if $s_3 \neq 0$, and $u_2 \neq 0$ if $s_2 \neq 0 \, \wedge \, s_1 = 0$).
\end{example}

\note{
	\vspace{2mm}
	\begin{enumerate}[<alert@+>]
		\footnotesize
		\setlength\itemsep{3mm}
		\item Read.
		\item Read. Thus, these three vectors are linearly independent.
	\end{enumerate}
}

\end{frame}

\begin{frame}<1-2> \frametitle{3.3.2: Basis}

\begin{definition}
Let $V$ be a vector space over $F$.
Let $\mathcal{B} = \left\{ \vecnot{v}_{\alpha} | \alpha \in A \right\}$ be a subset of linearly independent vectors from $V$ such that every $\vecnot{v} \in V$ can be written as a finite linear combination of vectors from $\mathcal{B}$.
Then, the set $\mathcal{B}$ is a \textcolor{blue}{Hamel basis} for $V$.
If $V$ has a finite basis, it is called \textcolor{blue}{finite-dimensional}.
\end{definition}

\vspace{3mm}

From this, a basis decomposition $\vecnot{v} = \sum_{i=1}^n s_i \vecnot{v}_{\alpha_i}$ must be unique: \\[1mm]
\hspace{5mm} The difference between any two distinct decompositions produces a finite linear dependency in the basis and, hence, a contradiction.

\vspace{3mm}

\begin{theorem}<2->
Every vector space has a Hamel basis.
\end{theorem}

\note{
	\vspace{2mm}
	\begin{enumerate}[<alert@+>]
		\footnotesize
		\setlength\itemsep{3mm}
		\item Read. Thus, in vector spaces, uniqueness of representation follows from the linear independence of the basis vectors.   
		\item Read. In finite dimensions, a Hamel basis is exactly what you've seen called a basis in past linear algebra courses.  For infinite dimensions, things will be different once we have a topology defined.
	\end{enumerate}
}

\end{frame}

\begin{frame}<1-1> \frametitle{3.3.2: Standard Basis}
\begin{example}
Let $F$ be a field and let $U \subset F^n$ be the list of vectors $\vecnot{e}_1, \ldots, \vecnot{e}_n$ defined by
\begin{equation*}
\begin{array}{ccc}
\vecnot{e}_1 & = & (1, 0, \ldots, 0) \\
\vecnot{e}_2 & = & (0, 1, \ldots, 0) \\
\vdots & = & \vdots \\
\vecnot{e}_n & = & (0, 0, \ldots, 1).
\end{array}
\end{equation*}
For any $\vecnot{v} = (v_1, \ldots, v_n) \in F^n$, we have
\begin{equation} \label{equation:StandardBasisSpan}
\vecnot{v} = \sum_{i=1}^n v_i \vecnot{e}_i.
\end{equation}
Thus, the collection $U = \left\{ \vecnot{e}_1, \ldots, \vecnot{e}_n \right\}$ spans $F^n$.
Since $\vecnot{v} = \vecnot{0}$ in~\eqref{equation:StandardBasisSpan} if and only if $v_1 = \cdots = v_n = 0$, $U$ is linearly independent.
Accordingly, the set $U$ is a basis for $F^{n}$.
This basis is termed the \textcolor{blue}{standard basis} of $F^n$.
\end{example}

\note{
	\vspace{2mm}
	\begin{enumerate}[<alert@+>]
		\footnotesize
		\setlength\itemsep{3mm}
		\item Read.
	\end{enumerate}
}

\end{frame}

\begin{frame}<1-2> \frametitle{3.3.2: Dimension}

\begin{theorem}<1-> 
Let $V$ be a finite-dimensional  vector space that is spanned by a finite set of vectors $W = \left\{ \vecnot{w}_1, \ldots, \vecnot{w}_n \right\}$.
If $U = \left\{ \vecnot{u}_1, \ldots, \vecnot{u}_m \right\} \subset V$ is a linearly independent set of vectors, then $m\leq n$.
\end{theorem}

\visible<1-> {Proof in live session.}

\vspace{6mm}

\uncover<2->{Thus, all bases have the same number of vectors.}
\begin{definition}<2->
The \textcolor{blue}{dimension} of a finite-dimensional vector space is the number of elements in any basis for $V$.
It is denoted by $\dim(V)$.
\end{definition}

\note{
	\vspace{2mm}
	\begin{enumerate}[<alert@+>]
		\footnotesize
		\setlength\itemsep{3mm}
		\item The following theorem says that, if a space is spanned by $n$ vectors then it can contain at most $n$ linearly indepedent vectors.  Specifically, (read). \\ [2mm]
		We can prove this theorem using our lemma from the last lecture based on the row-reduced echelon form.
		\item Read. To see this, suppose two bases have a different number of vectors.  Then, if we let $W$ be the smaller basis and $U$ be the larger basis, we can apply the Theorem to get a contradiction.
	\end{enumerate}
}

\end{frame}

\begin{frame}<1-2> \frametitle{3.3.2: Invertibility}

\begin{lemma}<1->
Let $A\in F^{n \times n}$ be an invertible matrix.
Then, the columns of $A$ form a basis for $F^n$.
Similarly, the rows of $A$ will also form a basis for $F^n$.
\end{lemma}

\visible<1->{Proof in live session.}

\vspace{3mm}

\begin{theorem}<2->
Let $A$ be an $n \times n$ matrix over $F$ whose columns, denoted by $\vecnot{a}_1, \ldots, \vecnot{a}_n$, form a linearly independent set of vectors in $F^{n}$.
Then $A$ is invertible.
\end{theorem}

\visible<2->{Proof in live session.}

\note{
	\vspace{2mm}
	\begin{enumerate}[<alert@+>]
		\footnotesize
		\setlength\itemsep{3mm}
		\item Read. This shows that invertible matrices has linearly independent rows and columns.
		\item Read. This shows a matrix is invertible if and only if its rows and columns are linearly independent.
	\end{enumerate}
}

\end{frame}


\begin{frame} \frametitle{Next Steps}

\begin{itemize}
\setlength\itemsep{5mm}
\item To continue studying after this video -- \vspace{2mm}

\begin{itemize}
 \setlength\itemsep{3mm}
 \item Try the required reading: Course Notes EF 3.1 - 3.3
 \item Or the recommended reading: LADR Ch. 1, Ch. 2
 \item Also, look at the problems in Assignment 4
\end{itemize}
\end{itemize}

\note{
	\vspace{8mm}
	\begin{enumerate}
		\setlength\itemsep{3mm}
		\color{red}
		\item Here are some options to continue learning this material. (read) \\ [2mm]  That's it for today.  So, I'll see you next time.
	\end{enumerate}
}

\end{frame}



\end{document}


