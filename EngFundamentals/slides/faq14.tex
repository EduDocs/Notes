\documentclass[10pt,english]{article}

\usepackage{amsfonts,url}
\usepackage{amsmath}
\usepackage{amssymb}
\usepackage{hyperref}
% Paper setup
\evensidemargin=0in
\oddsidemargin=0in
\textwidth=6.25in
\topmargin=-0.5in
\headheight=0.0in
\headsep=0.5in
\textheight=9.0in
\footskip=0.5in

%\usepackage{fixltx2e}

\newcommand{\Interior}[1]{\ensuremath{{#1}^{\circ}}}
\newcommand{\Closure}[1]{\ensuremath{\overline{#1}}}
\newcommand{\Complement}[1]{\ensuremath{{#1}^{c}}}

\newcommand{\Expect}{\ensuremath{\mathrm{E}}}
\newcommand{\vecnot}{\underline}
\newcommand{\RealNumbers}{\mathbb{R}}
\newcommand{\RationalNumbers}{\mathbb{Q}}
\newcommand{\ComplexNumbers}{\mathbb{C}}
\newcommand{\Real}{\mathrm{Re}}
\newcommand{\Span}{\mathrm{span}}
\newcommand{\Rank}{\mathrm{rank}}
\newcommand{\Nullity}{\mathrm{nullity}}
\newcommand{\Trace}{\mathrm{tr}}
\newcommand{\Diag}{\mathrm{diag}}
\DeclareMathOperator*{\esssup}{ess\,sup}
\newcommand{\dd}{\mathrm{d}}



\begin{document}

\title{ECE 586: Vector Space Methods \\ Lecture 14: Frequently Asked Questions}
\author{Henry D. Pfister \\ Duke University}
\date{\today}

\maketitle

Here we give a list of questions, and their answers, that were submitted by students after watching the flip video or reviewing the lecture slides.

\section{Matrix and Operator Norms}

\paragraph{For the second point of proof on page 5, why does this convergence hold?}
This holds from a proof that, in a complete metric space, absolute convergence (i.e. the convergence of the absolute sum $\sum_{i=0}^\infty ||\vecnot{v}_i|| = c <\infty $) implies convergence of the sum $ \sum_{i=0}^\infty \vecnot{v}_i$. The gist of the proof is that convergence of the real sum $b_n \triangleq \sum_{i=0}^n ||\vecnot{v}_i|| $ implies that $b_n$ is a Cauchy sequence. You then do a bit of manipulation to show that the sequence $\vecnot{u}_n = \sum_{i=0}^n \vecnot{v}_i$ is also Cauchy.
Thus, by completeness, $\vecnot{u}_n$ also converges.
For more details, see the section
``Proof that any absolutely convergent series in a Banach space is convergent" in 
\href{https://en.wikipedia.org/wiki/Absolute_convergence}{\textbf{web link}}.

\paragraph{For the matrix norm, how is $\| \cdot \|_p$ defined?}
Suppose $\vecnot{v}$ is a $n$-dimensional vector, then its vector $p$-norm is defined by
$$\| \vecnot{v} \|_p \triangleq \left[\sum_{i=1}^n |v_i|^p \right]^{\frac1p}.$$
The matrix $p$-norm is defined by 
$$\| A \|_p \triangleq \max_{\|\vecnot{v}\|_p = 1} \|A \vecnot{v} \|_p \, , $$
where the norms on the RHS are vector $p$-norms.

\paragraph{In course notes EF 6.3 (the induced operator norm definition), the norm pipes/brackets ($||\cdot||$) surrounding $Tv$ and $v$ do not have a subscript. However, the prelecture slide 1 uses the vector space symbols as subscripts to the norm pipes/brackets (i.e. $|| . ||_{W}$ or $|| . ||_{V}$). What does that subscript signify and is it necessary for clarifying the expression?}
The subscript identifies which norm to use.
When there is no subscript, it means it should be obvious to the reader what norm is being used. For example we know that the vector $\vecnot{v}$ is in $V$, so when we write $\| \vecnot{v} \|$ we implicitly mean $\|\vecnot{v}\|_V$.
We also know $T \vecnot{v} \in W$, so when we write $\|T\vecnot{v}\|$, we implicitly mean $\|T\vecnot{v}\|_W$.

\paragraph{Could you explain the proof for the Neumann expansion with slightly more detail again?}
See Section 6.3.2 of the lecture notes.

\paragraph{Why the $\ell^2$ norm for matrices is defined based on the max eigenvalues?}
Because this gives a simple analytic expression.  See proof in section 6.3.3 of the lecture notes.

\paragraph{In Slide 3, how to derive these matrix norms? Why the $\ell^{\infty}$ norm is the same as $\ell^1$ norm swapping i and j indices?}

For proofs of these equalities, see \href{https://www.cs.utexas.edu/users/flame/Notes/NotesOnNorms.pdf}{\textbf{web link}}, in the section starting with ``Specifically, the matrix p-norm $\vert\vert{\bf A}\vert\vert _p$ can be based on the vector p-norm $\vert\vert{\bf x}\vert\vert _p$, as defined in the following for  $p=1, 2, \infty$."

\paragraph{What is the relation of metric space and vector space, could you maybe give a Venn diagram of them?}
A metric space defines the topology of a set and a vector space defines the linear structure of a set.
%The Venn diagram you mention is pretty straightforward, see \href{https://cacoo.com/wp-app/uploads/2018/09/Cacoo-Venn-Diagram-Blog-680x450.png}{\textbf{this picture}}.
In general, metric spaces are not automatically vector spaces, and vector spaces are not automatically metric spaces.
But, a normed vector space also defines a canonical metric space structure on the same set via the induced metric.

In general, a metric space $(X,d)$ is simply a space of elements $X$ which is endowed with a metric $d(\cdot,\cdot)$ (you are already familiar with the properties of $d$).  Vector spaces need not have a metric in general, but they do need to have operations of scalar multiplication (over a field $F$), and vector addition, and these operations need to satisfy a list of properties.

\paragraph{Why is the induced operator norm considered induced?}
It is considered induced because it is defined in terms of the vector norms over the spaces $V$ and $W$ -- if we change which vector norms we are using over $V$ and $W$, then the induced norm of the operator will change as well.


\paragraph{Can one conceive of valid operator norms that are not the standard induced operator norm? And if so, do they all end up being equivalent in some manner?}

Yes, the Frobenius norm $\| A \|_F \triangleq \left( \sum_{i,j} |a_{ij}|^2 \right)^{1/2}$ is used fairly often but it is not induced by any vector norm.

\paragraph{What would be a real-world application for operator norm? I guess I heard about Neumann series. Are Neumann series and Neumann expansion related?}

Yes, they are two different names for the same thing.

\paragraph{What is the significance of the Neumann expansion? Is this just a tool used in proofs? Could you demonstrate an example of Neumann expansion?}

It is a fundamental relationship that is useful both in theory an practice.
For example, it was recently used to motivate a new class of neural networks that solve linear inverse problems: \href{https://arxiv.org/pdf/1901.03707.pdf}{\textbf{web link}}.

\paragraph{In the different $p$-norms, can $p$ be some fractional real value such as 1.5? Is there any point in doing so?}

Yes, any $p\geq 1$ works in general and thus $p$ does not have to be an integer.
There are various reasons for using arbitrary $p$ values.
Consider a problem where you have a long vector $\vecnot{v}$ and you know the $p$-norm for all $p\in [1,\infty]$ but you don't know the vector.
Then, the number of entries $v_i$ satisfying $|v_i| \geq t$ satisfies the upper bound $$\left| \{i\in [n]\,|\, |v_i|\geq t \} \right| \leq \frac{1}{t^p} \sum_{i=1}^n |v_i|^p = \frac{1}{t^p} \| \vecnot{v} \|_p^p . $$
Moreover, this upper bound can sometimes give a good estimate when minimized over $p\in [1,\infty]$.

\paragraph{Will we review eigenvalues and eigenvectors and how to calculate them with linear algebra instead?}

We will discuss eigenvalues and eigenvectors in more detail later in the class but we will not discuss numerical methods for computing them.
See EF Chapter 8 for more information.

\paragraph{Are matrix norms classified as a subset of operator norms?}
Yes, a matrix norm is an operator norm.
But, if the vector space is infinite dimensional, then the induced operator norm is usually not called a matrix norm.
This is true even if the operator can be represented as an infinite-dimensional matrix.

\paragraph{Is the idea of a contraction on a vector space related to the operator norm?}

A contraction on a closed subset $A$ of a normed vector space $V$ can be defined using the induced metric $d(\vecnot{u},\vecnot{w}) = \| \vecnot{u} - \vecnot{w} \|$.
In particular, $f \colon V \to V$ is a contraction on $A$ if $f(A) \subseteq A$ and $\| f(\vecnot{u}) - f(\vecnot{v}) \| \leq \gamma \| \vecnot{u} - \vecnot{v} \|$ for some $\gamma \in [0,1)$.

If $f$ is differentiable with Jacobian $J(\vecnot{v})$, then the second condition is essentially equivalent to the condition that the induced operator norm $\|J(\vecnot{v}) \|_{\mathrm{op}} \leq \gamma < 1$ for all $\vecnot{v} \in A$.

\end{document}
