\documentclass[10pt,english,aspectratio=169]{beamer}
% Use notes or hide notes or show only notes or handout

\usetheme{default}

\usepackage{xstring}
\usepackage{pgfpages}
%\makeatletter
%\IfSubStr{\@classoptionslist}{handout}
%  {\pgfpagesuselayout{2 on 1}[letterpaper,border shrink=5mm]}
%  {}
%\makeatother

\usepackage{amsmath,amssymb,amsthm}
\usepackage{stmaryrd}
\usepackage{enumerate}
\usepackage{stfloats}
\usepackage{bbm}
\usepackage{pdfpages}
\usepackage{framed}

\usepackage[most]{tcolorbox}
\tcbset{highlight math style={enhanced,
  colframe=white,colback=yellow!15,arc=8pt,boxrule=1pt,
  }}
  
\usepackage{tikz,pgf,pgfplots}
\usepackage{algorithm,algorithmic}
\usepgflibrary{shapes}
\usetikzlibrary{%
  arrows,%
  arrows.meta,
  backgrounds,
  shapes.misc,% wg. rounded rectangle
  shapes.arrows,%
  shapes,%
  calc,%
  chains,%
  matrix,%
  positioning,% wg. " of "
  scopes,%
  decorations.pathmorphing,% /pgf/decoration/random steps | erste Graphik
  shadows,%
  backgrounds,%
  fit,%
  petri,%
  quotes
}

\tikzset{background rectangle/.style={
    fill=white,
  },
  use background/.style={    
    show background rectangle
  }
}

\setbeamersize{text margin left=10mm,text margin right=35mm}

\pgfplotsset{compat=1.12}

%\usetheme{Frankfurt}
%\usecolortheme{ldpc}
\useinnertheme{rounded}
\usecolortheme{whale}
\usecolortheme{orchid}

\newcommand{\ul}[1]{\underline{#1}}
\renewcommand{\Pr}{\mathbb{P}}

%% Setup slides and notes
\makeatletter
\IfSubStr{\@classoptionslist}{notes} { \IfSubStr{\@classoptionslist}{hide} {}{\IfSubStr{\@classoptionslist}{only} {}{\setbeameroption{show notes on second screen=right}}} }{}
\makeatother
%\setbeamertemplate{note page}{\pagecolor{yellow!5}\vfill\insertnote\vfill}

\newcommand{\getpdfpages}[2]{\begingroup
  \setbeamercolor{background canvas}{bg=}
  \addtocounter{framenumber}{1}
  \includepdf[pages={#1},%
  pagecommand={%
    \expandafter\def\expandafter\insertshorttitle\expandafter{%
      \insertshorttitle\hfill\insertframenumber\,/\,\inserttotalframenumber}}%
  ]{#2}
  \endgroup}

\newcommand{\backupbegin}{
   \newcounter{finalframe}
   \setcounter{finalframe}{\value{framenumber}}
}
\newcommand{\backupend}{
   \setcounter{framenumber}{\value{finalframe}}
}

 \setbeamercolor{bibliography entry author}{fg=black}
 \setbeamercolor{bibliography entry title}{fg=black}
 \setbeamercolor{bibliography entry location}{fg=black}
 \setbeamercolor{bibliography entry note}{fg=black}
 
 \setbeamerfont{bibliography item}{size=\footnotesize}
 \setbeamerfont{bibliography entry author}{size=\footnotesize}
 \setbeamerfont{bibliography entry title}{size=\footnotesize}
 \setbeamerfont{bibliography entry location}{size=\footnotesize}
 \setbeamerfont{bibliography entry note}{size=\footnotesize}
 \setbeamertemplate{bibliography item}{\insertbiblabel}
 
\newlength\tikzwidth
\newlength\tikzheight


\newcommand{\mc}[1]{\mathcal{#1}}
\newcommand{\mbb}[1]{\mathbb{#1}}
%\newcommand{\expt}{\mbb{E}}
%\newcommand{\dd}{\mathrm{d}}
\newcommand{\Interior}[1]{\ensuremath{{#1}^{\circ}}}
\newcommand{\Closure}[1]{\ensuremath{\overline{#1}}}
\newcommand{\Complement}[1]{\ensuremath{{#1}^{c}}}

\newcommand{\Expect}{\ensuremath{\mathrm{E}}}
\newcommand{\vecnot}{\underline}
\newcommand{\RealNumbers}{\ensuremath{\mathbb{R}}}
\newcommand{\RationalNumbers}{\mathbb{Q}}
\newcommand{\ComplexNumbers}{\mathbb{C}}
\newcommand{\Real}{\mathrm{Re}}
\newcommand{\Span}{\mathrm{span}}
\newcommand{\Rank}{\mathrm{rank}}
\newcommand{\Nullity}{\mathrm{nullity}}
\newcommand{\Trace}{\mathrm{tr}}
\newcommand{\Diag}{\mathrm{diag}}
\newcommand{\dd}{\mathrm{d}}
\DeclareMathOperator*{\esssup}{ess\,sup}

% Use < , > inner product
\newcommand{\inner}[2]{{\left\langle #1 \mskip2mu , #2 \right\rangle}}
\newcommand{\tinner}[2]{{\langle #1 \mskip1mu , #2 \rangle}}

% Use < | > inner product
%\newcommand{\inner}[2]{{\left\langle #1 \mskip2mu \middle| \mskip2mu #2 \right\rangle}}
%\newcommand{\tinner}[2]{{\langle #1 \mskip1mu | \mskip1mu  #2 \rangle}}




\def\checkmark{\tikz\fill[scale=0.4](0,.35) -- (.25,0) -- (1,.7) -- (.25,.15) -- cycle;}
\def\greencheck{{\color{green}\checkmark}}
\def\scalecheck{\resizebox{\widthof{\checkmark}*\ratio{\widthof{x}}{\widthof{\normalsize x}}}{!}{\checkmark}}
\def\xmark{\tikz [x=1.4ex,y=1.4ex,line width=.2ex, red] \draw (0,0) -- (1,1) (0,1) -- (1,0);}
\def\redx{{\color{red}\xmark}}

\renewcommand{\footnotesep}{-2pt}


\begin{document}

\title{ECE 586: Vector Space Methods \\ Lecture 11 Flip Video: Linear Transforms}
\author{Henry D. Pfister \\ Duke University}
\date{}
%\date{August 20th, 2020}
%\maketitle

\setbeamertemplate{navigation symbols}{}

\begin{frame}[plain]
	\titlepage
	
	\note{
		\vspace{8mm}
		\begin{enumerate}
			\setlength\itemsep{3mm}
			\color{red}
			\item Welcome to the 11th video lecture for ECE 586, Vector Space Methods. \\[2mm]
			Today, we'll finish our discussion of subspaces and bases and then move on to linear transforms.
		\end{enumerate}
	}
\end{frame}

\addtocounter{framenumber}{-1}
\setbeamertemplate{navigation symbols}{\textcolor{blue}{\footnotesize \insertframenumber ~/ \inserttotalframenumber}}


\begin{frame}<1-4> \frametitle{Sums and Direct Sums}

\vspace{-1mm}

\begin{definition}<1->
Let $U,W$ be subsets of a vector space $V$.
The \textcolor{blue}{sum} of $U$ and $W$ is defined by \vspace{-2mm}
\begin{equation*}
U + W \triangleq \left\{ \vecnot{v} \in V \, \middle| \, \exists \vecnot{u} \in U, \exists \vecnot{w} \in W, \vecnot{v} = \vecnot{u} + \vecnot{w} \right\}.
\end{equation*}
\end{definition}

\begin{definition}<2->
For a vector space $V$, subspaces $U$ and $W$ are called \textcolor{blue}{disjoint} if $U \cap W = \{\vecnot{0}\}$.
\end{definition}

\begin{definition}<3->
For disjoint subspaces $U$ and $W$ in a vector space, their \textcolor{blue}{direct sum}  equals their sum but is denoted by $U \oplus W$ to emphasize that $U$ and $W$ are disjoint.
\end{definition}

\vspace{1mm}

\uncover<4->{%
An important property of a direct sum is that any vector $\vecnot{v} \in U \oplus W$ has a unique decomposition $\vecnot{v} = \vecnot{u} + \vecnot{w}$ where $\vecnot{u} \in U$ and $\vecnot{w} \in W$.
}

\vspace{-1mm}

\note{
	\vspace{2mm}
	\begin{enumerate}[<alert@+>]
		\footnotesize
		\setlength\itemsep{3mm}
		\item Sums and Direct Sums allow one to combine subspaces.  Read.
		\item Read.  This is the smallest intersection possible because all subspaces contain the zero vector.
		\item Read.
		\item Read.
	\end{enumerate}
}

\end{frame}



\begin{frame}<1-4> \frametitle{3.3.3 Coordinate Systems and Vectors}

\vspace{-1mm}

\begin{definition}<1->
If $V$ is a finite-dimensional vector space, an \textcolor{blue}{ordered basis} for $V$ is a finite list of vectors that is linearly independent and spans $V$.
\end{definition}

\begin{exampleblock}{Remark}<2->
If $\mathcal{B} = (\vecnot{v}_1, \ldots, \vecnot{v}_n )$ is an ordered basis for $V$, then the set $\left\{ \vecnot{v}_1, \ldots, \vecnot{v}_n \right\}$ is a basis for $V$.
But, $\mathcal{B}$ defines the set and a specific ordering for the vectors.
\end{exampleblock}

\begin{definition}<3->
For a finite-dimensional vector space $V$ with ordered basis $\mathcal{B}=(\vecnot{v}_1, \ldots, \vecnot{v}_n )$, the \textcolor{blue}{coordinate vector} of $\vecnot{v} \in V$ is denoted by $\left[ \vecnot{v} \right]_{\mathcal{B}}$ and equals the unique vector $\vecnot{s} = F^n$ such that \vspace{-1mm}
\[ \vecnot{v} = \sum_{i=1}^n s_i \vecnot{v}_i. \]  
\end{definition}

\vspace{-0.5mm}

\visible<4->{Try computing $[(1,2,3,4)]_\mathcal{B}$ for $\mathcal{B} = ( (1,1,1,1),(0,1,1,1),(0,0,1,1),(0,0,0,1) )$.\hspace*{-10mm}}

\note{
	\vspace{2mm}
	\begin{enumerate}[<alert@+>]
		\footnotesize
		\setlength\itemsep{3mm}
		\item Coordinate systems allow multiple representations of the same vector in different bases. Read.
		\item Read. 
		\item Read.
		\item Read.  We know such a such exists and is unique because $\mathcal{B}$ is a basis.
	\end{enumerate}
}

\end{frame}



\begin{frame}<1-3> \frametitle{3.4: Linear Transforms}

\vspace{-1mm}

\begin{definition}
Let $V$ and $W$ be vector spaces over a field $F$.
A \textcolor{blue}{linear transform} from $V$ to $W$ is a function $T$ from $V$ into $W$ such that \vspace{-2mm}
\begin{equation*}
T \left( s \vecnot{v}_1 + \vecnot{v}_2 \right)
= s T \vecnot{v}_1 + T \vecnot{v}_2  \vspace{-2mm}
\end{equation*}
for all vectors $\vecnot{v}_1, \vecnot{v}_2 \in V$ and all scalars $s \in F$ (i.e., $T$ is linear).
\end{definition}

\vspace{-1mm}

\begin{example}<2->
Let $A$ be a fixed $m \times n$ matrix over $F$.
The function $T$ defined by $T \left( \vecnot{v} \right) = A \vecnot{v}$ is a linear transformation from $F^{n \times 1}$ into $F^{m \times 1}$.
\end{example}

\vspace{-1mm}

\begin{example}<3->
Let $V$ be the space of continuous functions from $[0,1]$ to $\RealNumbers$. Define $T$ by  \vspace{-2mm}
\begin{equation*}
(Tf)(x) = \int_{0}^x f(t) dt .  \vspace{-2mm}
\end{equation*}
Then, $T$ is a linear transform from $V$ to $V$ because $Tf$ is continuous.
\end{example}

\note{
	\vspace{2mm}
	\begin{enumerate}[<alert@+>]
		\footnotesize
		\setlength\itemsep{3mm}
		\item Linear transformations play an important role in linear algebra. Read.
		\item Read. It is good exercise to identify which properties of which objects imply this?
		\item Read.  Again, it is good to think how you would show this.
	\end{enumerate}
}

\end{frame}

\begin{frame}<1-3> \frametitle{3.4.2: Properties of Linear Transforms}

\vspace{-1mm}

\begin{definition}[Range]<1->
For a linear transformation $T\colon V \to W$, the \textcolor{blue}{range} of $T$ is the subspace of vectors $\vecnot{w} \in W$ such that $\vecnot{w} = T \vecnot{v}$ for some $\vecnot{v} \in V$.
It is denoted by \vspace{-2mm}
\[ \mathcal{R}(T) \triangleq \{ \vecnot{w}\in W | \exists \vecnot{v}\in V \textrm{ s.t. } T\vecnot{v} = \vecnot{w} \} = \{ T \vecnot{v} | \vecnot{v} \in V\}. \]
\end{definition}

\vspace{-1mm}

\begin{definition}[Nullspace]<2->
For a linear transformation $T\colon V \to W$, the \textcolor{blue}{nullspace} of $T$ is the subspace of vectors $\vecnot{v} \in V$ such that $T \vecnot{v} = \vecnot{0}$.
We denote the nullspace of $T$ by \vspace{-2mm}
\[ \mathcal{N}(T) \triangleq \{ \vecnot{v}\in V | T \vecnot{v} = \vecnot{0} \} .\]
\end{definition}

\vspace{-1mm}

\begin{theorem}<3->
Let $V,W$ be vector spaces over $F$ and $\mathcal{B} = \{ \vecnot{v}_{\alpha} | \alpha \in A \}$ be a Hamel basis for $V$.
For each mapping $G \colon \mathcal{B} \rightarrow W$, there is a unique linear transformation $T \colon V \rightarrow W$ such that $T \vecnot{v}_{\alpha} = G \left( \vecnot{v}_{\alpha} \right)$ for all $\alpha \in A$.
\end{theorem}

\vspace{-0.5mm}

\visible<3->{Proof in live session.}

\note{
	\vspace{2mm}
	\begin{enumerate}[<alert@+>]
		\footnotesize
		\setlength\itemsep{3mm}
		\item Each linear transform has a few natural subspaces associated with it. \\ [2mm] Read. This is just the standard range of the function $T$ but linearity implies that set is also subspace.  Can you identify exactly why?
		\item Read. Again, how would you prove this is a subspace?
		\item This theorem describes the most important property of a linear transform.  It says that a linear transform is uniquely defined by where it maps a basis.  Read.
	\end{enumerate}
}

\end{frame}


\begin{frame}<1-3> \frametitle{3.4.2: Rank and Nullity}

\vspace{-1mm}

\begin{definition}[Rank and Nullity]<1->
Let $V$ and $W$ be vector spaces over a field $F$, and let $T$ be a linear transformation from $V$ into $W$.
The \textcolor{blue}{rank} of $T$ is the dimension of the range of $T$ and the \textcolor{blue}{nullity} of $T$ is the dimension of the nullspace of $T$.
\end{definition}

\begin{theorem}[Rank-Nullity]<2->
Let $V$ and $W$ be vector spaces over the field $F$ and let $T$ be a linear transformation from $V$ into $W$.
If $V$ is finite-dimensional, then \vspace{-2mm}
\begin{equation*}
\Rank (T) + \Nullity (T) = \dim (V)
\end{equation*}
\end{theorem}

\vspace{-0.5mm}
\visible<2->{Proof in live session.}



\begin{theorem}<3->
If $A$ is an $m \times n$ matrix with entries in the field $F$, then \vspace{-2mm}
\begin{equation*} 
\mathrm{row~rank} (A) \triangleq \dim( \mathcal{R}(A^T)) =  \dim( \mathcal{R}(A)) \triangleq \Rank (A).
\end{equation*}
\end{theorem}

\vspace{-0.5mm}

\visible<3->{Proof in live session.}

\note{
	\vspace{2mm}
	\begin{enumerate}[<alert@+>]
		\footnotesize
		\setlength\itemsep{3mm}
		\item The dimensions of the range and nullspace also have special names. Read.
		\item Read. It is worth noting here that the range is a subspace of $W$ while the nullspace is a subspace of $V$.
		\item This theorem shows that number of linearly independent rows in a matrix equals the number of linearly independent columns. Read. 
	\end{enumerate}
}


\end{frame}


\begin{frame} \frametitle{Next Steps}

\begin{itemize}
\setlength\itemsep{5mm}
\item To continue studying after this video -- \vspace{2mm}

\begin{itemize}
 \setlength\itemsep{3mm}
 \item Try the required reading: Course Notes EF 3.3 - 3.4
 \item Or the recommended reading: LADR Ch. 3ABC
 \item Also, look at the problems in Assignment 4
\end{itemize}
\end{itemize}

\note{
	\vspace{8mm}
	\begin{enumerate}
		\setlength\itemsep{3mm}
		\color{red}
		\item Here are some options to continue learning this material. (read) \\ [2mm]  That's it for today.  So, I'll see you next time.
	\end{enumerate}
}

\end{frame}


\end{document}

