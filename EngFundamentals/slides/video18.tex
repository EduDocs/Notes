\documentclass[10pt,english,aspectratio=169]{beamer}
% Use notes or hide notes or show only notes or handout


\usetheme{default}

\usepackage{xstring}
\usepackage{pgfpages}
%\makeatletter
%\IfSubStr{\@classoptionslist}{handout}
%  {\pgfpagesuselayout{2 on 1}[letterpaper,border shrink=5mm]}
%  {}
%\makeatother

\usepackage{amsmath,amssymb,amsthm}
\usepackage{stmaryrd}
\usepackage{enumerate}
\usepackage{stfloats}
\usepackage{bbm}
\usepackage{pdfpages}
\usepackage{framed}
\usepackage{tabularx}
\usepackage{scalerel}

\usepackage[most]{tcolorbox}
\tcbset{highlight math style={enhanced,
  colframe=white,colback=yellow!15,arc=8pt,boxrule=1pt,
  }}
  
\usepackage{tikz,pgf,pgfplots}
\usepackage{algorithm,algorithmic}
\usepgflibrary{shapes}
\usetikzlibrary{%
  arrows,%
  arrows.meta,
  backgrounds,
  shapes.misc,% wg. rounded rectangle
  shapes.arrows,%
  shapes,%
  calc,%
  chains,%
  matrix,%
  positioning,% wg. " of "
  scopes,%
  decorations.pathmorphing,% /pgf/decoration/random steps | erste Graphik
  shadows,%
  backgrounds,%
  fit,%
  petri,%
  quotes
}

\tikzset{background rectangle/.style={
    fill=white,
  },
  use background/.style={    
    show background rectangle
  }
}

\setbeamersize{text margin left=10mm,text margin right=35mm}

\pgfplotsset{compat=1.12}

%\usetheme{Frankfurt}
%\usecolortheme{ldpc}
\useinnertheme{rounded}
\usecolortheme{whale}
\usecolortheme{orchid}

\newcommand{\ul}[1]{\underline{#1}}
\renewcommand{\Pr}{\mathbb{P}}

%% Setup slides and notes
\makeatletter
\IfSubStr{\@classoptionslist}{notes} { \IfSubStr{\@classoptionslist}{hide} {}{\IfSubStr{\@classoptionslist}{only} {}{\setbeameroption{show notes on second screen=right}}} }{}
\makeatother
%\setbeamertemplate{note page}{\pagecolor{yellow!5}\vfill\insertnote\vfill}

\newcommand{\getpdfpages}[2]{\begingroup
  \setbeamercolor{background canvas}{bg=}
  \addtocounter{framenumber}{1}
  \includepdf[pages={#1},%
  pagecommand={%
    \expandafter\def\expandafter\insertshorttitle\expandafter{%
      \insertshorttitle\hfill\insertframenumber\,/\,\inserttotalframenumber}}%
  ]{#2}
  \endgroup}

\newcommand{\backupbegin}{
   \newcounter{finalframe}
   \setcounter{finalframe}{\value{framenumber}}
}
\newcommand{\backupend}{
   \setcounter{framenumber}{\value{finalframe}}
}

 \setbeamercolor{bibliography entry author}{fg=black}
 \setbeamercolor{bibliography entry title}{fg=black}
 \setbeamercolor{bibliography entry location}{fg=black}
 \setbeamercolor{bibliography entry note}{fg=black}
 
 \setbeamerfont{bibliography item}{size=\footnotesize}
 \setbeamerfont{bibliography entry author}{size=\footnotesize}
 \setbeamerfont{bibliography entry title}{size=\footnotesize}
 \setbeamerfont{bibliography entry location}{size=\footnotesize}
 \setbeamerfont{bibliography entry note}{size=\footnotesize}
 \setbeamertemplate{bibliography item}{\insertbiblabel}
 
\newlength\tikzwidth
\newlength\tikzheight


\newcommand{\mc}[1]{\mathcal{#1}}
\newcommand{\mbb}[1]{\mathbb{#1}}
%\newcommand{\expt}{\mbb{E}}
%\newcommand{\dd}{\mathrm{d}}
%\usepackage{fixltx2e}

\newcommand{\Interior}[1]{\ensuremath{{#1}^{\circ}}}
\newcommand{\Closure}[1]{\ensuremath{\overline{#1}}}
\newcommand{\Complement}[1]{\ensuremath{{#1}^{c}}}

\newcommand{\Expect}{\ensuremath{\mathrm{E}}}
\newcommand{\vecnot}{\underline}
\newcommand{\RealNumbers}{\mathbb{R}}
\newcommand{\RationalNumbers}{\mathbb{Q}}
\newcommand{\ComplexNumbers}{\mathbb{C}}
\newcommand{\Real}{\mathrm{Re}}
\newcommand{\Span}{\mathrm{span}}
\newcommand{\Rank}{\mathrm{rank}}
\newcommand{\Nullity}{\mathrm{nullity}}
\newcommand{\Trace}{\mathrm{tr}}
\newcommand{\Diag}{\mathrm{diag}}
\DeclareMathOperator*{\esssup}{ess\,sup}
\newcommand{\dd}{\mathrm{d}}



\def\checkmark{\tikz\fill[scale=0.4](0,.35) -- (.25,0) -- (1,.7) -- (.25,.15) -- cycle;}
\def\greencheck{{\color{green}\checkmark}}
\def\scalecheck{\resizebox{\widthof{\checkmark}*\ratio{\widthof{x}}{\widthof{\normalsize x}}}{!}{\checkmark}}
\def\xmark{\tikz [x=1.4ex,y=1.4ex,line width=.2ex, red] \draw (0,0) -- (1,1) (0,1) -- (1,0);}
\def\redx{{\color{red}\xmark}}

\renewcommand{\footnotesep}{-2pt}


\begin{document}

\title{ECE 586: Vector Space Methods \\ Lecture 18: Best Approximation Formulas}
\author{Henry D. Pfister \\ Duke University}
\date{}
%\date{August 20th, 2020}
%\maketitle

\setbeamertemplate{navigation symbols}{}

\begin{frame}[plain]
	\titlepage
	
	\note{
		\vspace{8mm}
		\begin{enumerate}
			\setlength\itemsep{3mm}
			\color{red}
			\item Welcome to the 11th video lecture for ECE 586, Vector Space Methods. \\[2mm]
			Today, we'll finish our discussion of subspaces and bases and then move on to linear transforms.
		\end{enumerate}
	}
\end{frame}

\addtocounter{framenumber}{-1}
\setbeamertemplate{navigation symbols}{\textcolor{blue}{\footnotesize \insertframenumber ~/ \inserttotalframenumber}}




\begin{frame}{4.2: Normal Equations}

Let $W$ be a subspace of a Hilbert space $V$ that is spanned by the linearly independent (but not orthogonal) set of vectors  $\vecnot{w}_1, \ldots, \vecnot{w}_n \in V$.

\vspace{2mm}

The projection theorem shows that $\hat{\vecnot{v}} \in W$ is the best approximation of $\vecnot{v} \in V$ if and only if $(\vecnot{v} - \hat{\vecnot{v}}) \bot \vecnot{w}_j$ for $j=1,\ldots,n$.
This implies that \vspace{-1mm}
\begin{equation*}
\left\langle \vecnot{v} - \hat{\vecnot{v}} | \vecnot{w}_j \right\rangle
= \left\langle \vecnot{v} - \sum_{i=1}^n s_i \vecnot{w}_i \Big| \vecnot{w}_j \right\rangle
= 0 \vspace{-1mm}
\end{equation*}
or, equivalently, the \textcolor{blue}{normal equations} \vspace{-1mm}
\begin{equation*}
\sum_{i=1}^n s_i \left\langle \vecnot{w}_i | \vecnot{w}_j \right\rangle
= \left\langle \vecnot{v} | \vecnot{w}_j \right\rangle. \vspace{-1mm}
\end{equation*}

The gives a system of $n$ linear equations in $n$ unknowns defined by \vspace{-1mm}
\begin{equation*}
\underbrace{\left[ \begin{array}{cccc}
\left\langle \vecnot{w}_1 | \vecnot{w}_1 \right\rangle
& \left\langle \vecnot{w}_2 | \vecnot{w}_1 \right\rangle & \cdots
& \left\langle \vecnot{w}_n | \vecnot{w}_1 \right\rangle \\
\left\langle \vecnot{w}_1 | \vecnot{w}_2 \right\rangle
& \left\langle \vecnot{w}_2 | \vecnot{w}_2 \right\rangle & \cdots
& \left\langle \vecnot{w}_n | \vecnot{w}_2 \right\rangle \\
\vdots & \vdots & \ddots & \vdots \\
\left\langle \vecnot{w}_1 | \vecnot{w}_n \right\rangle
& \left\langle \vecnot{w}_2 | \vecnot{w}_n \right\rangle & \cdots
& \left\langle \vecnot{w}_n | \vecnot{w}_n \right\rangle
\end{array} \right]}_G
\underbrace{\left[ \begin{array}{c}
s_1 \\ s_2 \\ \vdots \\ s_n \end{array} \right]}_\vecnot{s}
= \underbrace{\left[ \begin{array}{c}
\left\langle \vecnot{v} | \vecnot{w}_1 \right\rangle \\
\left\langle \vecnot{v} | \vecnot{w}_2 \right\rangle \\ \vdots \\
\left\langle \vecnot{v} | \vecnot{w}_n \right\rangle \end{array} \right]}_\vecnot{t} .
\end{equation*}

\end{frame}

\begin{frame}{4.2: The Gramian}

\vspace{-1.5mm}

\begin{definition}<1->
For $\vecnot{w}_1,\ldots,\vecnot{w}_n$, the $n \times n$ \textcolor{blue}{Gramian matrix} is defined to be \vspace{-2mm}
\begin{equation*}
G = \left[ \begin{array}{cccc}
\left\langle \vecnot{w}_1 | \vecnot{w}_1 \right\rangle
& \left\langle \vecnot{w}_2 | \vecnot{w}_1 \right\rangle & \cdots
& \left\langle \vecnot{w}_n | \vecnot{w}_1 \right\rangle \\
\left\langle \vecnot{w}_1 | \vecnot{w}_2 \right\rangle
& \left\langle \vecnot{w}_2 | \vecnot{w}_2 \right\rangle & \cdots
& \left\langle \vecnot{w}_n | \vecnot{w}_2 \right\rangle \\
\vdots & \vdots & \ddots & \vdots \\
\left\langle \vecnot{w}_1 | \vecnot{w}_n \right\rangle
& \left\langle \vecnot{w}_2 | \vecnot{w}_n \right\rangle & \cdots
& \left\langle \vecnot{w}_n | \vecnot{w}_n \right\rangle
\end{array} \right] \vspace{-1mm}
\end{equation*}
Since $g_{ij} = \smash{\left\langle \vecnot{w}_j | \vecnot{w}_i \right\rangle}$, we see $G$ is Hermitian symmetric (i.e. $G^H = G$).
\end{definition}

\vspace{-1mm}

\begin{definition}<2->
A matrix $M\in F^{n \times n}$ is \textcolor{blue}{positive-semidefinite} if $M=M^H$ and $\vecnot{v}^H M \vecnot{v} \geq 0$ for all $\vecnot{v} \in F^n - \left\{ \vecnot{0} \right\}$.
If the inequality is strict, then it is \textcolor{blue}{positive-definite}.
\end{definition}

\vspace{-1mm}

\begin{theorem}<2->
A Gramian matrix $G$ is always positive-semidefinite.
It is positive-definite if and only if the vectors $\vecnot{w}_1, \ldots, \vecnot{w}_n$ is linearly independent.
\end{theorem}
\vspace{-1mm}

\visible<2->{Proof in live session.}

\end{frame}

\begin{frame}{4.3: Least-Squares Solution of a Linear System}

\visible<1->{%
For $V = F^m$, let $A \in F^{m \times n}$ be a matrix whose $i$-th column is $\vecnot{w}_i \in V$.
Then, a vector $\hat{\vecnot{v}} \in W = \text{colspace}(A)$ can be written as \vspace{-1mm}
\[ \hat{\vecnot{v}} = A \vecnot{s} = \sum_{i=1}^n s_i \vecnot{w}_i. \vspace{-1mm} \]

Also, the best approximation of $\vecnot{v}$ by vectors in $W$ is found by solving
\[ \min_{\hat{\vecnot{v}}\in W} \| \vecnot{v} - \hat{ \vecnot{v}} \| = \min_\vecnot{s} \| \vecnot{v} - A \vecnot{s} \| . \]
}

\vspace{-2mm}

\visible<2->{%
For the induced norm, any solution must satisfy the normal equations
\[ \left\langle \vecnot{v} - \hat{\vecnot{v}} | \vecnot{w}_j \right\rangle
= \left\langle \vecnot{v} - A \vecnot{s} | \vecnot{w}_j \right\rangle
= 0, \quad j \in [n]. \]
}

\vspace{-4mm}

\visible<3->{%
For the standard inner product, these equations can be expressed as
\begin{equation*}
\vecnot{0} = \left[ \begin{array}{c} \vecnot{w}_1^H \\ \vdots \\ \vecnot{w}_n^H \end{array} \right] \left( \vecnot{v} - A \vecnot{s} \right) = A^H \vecnot{v} - A^H A \vecnot{s} = \vecnot{t} - G \vecnot{s},
\end{equation*}
where \textcolor{blue}{$G = A^H A$ is the Gramian} and \textcolor{blue}{$\vecnot{t}$ is the cross-correlation vector.}}
\end{frame}

\begin{frame}{4.3.2: Pseudo-Inverse and Projection}

\visible<1->{%
When the vectors $\vecnot{w}_1, \ldots, \vecnot{w}_n$ are linearly independent, the Gramian matrix is positive definite and hence invertible.
Thus, the optimal solution for the least-squares problem is given by
\begin{equation*}
\vecnot{s} = G^{-1} \vecnot{t} = \left( A^H A \right)^{-1} A^H \vecnot{v},
\end{equation*}
where the matrix \textcolor{blue}{$\left( A^H A \right)^{-1} A^H$ is the pseudoinverse} of $A$ in this case.}

\vspace{7mm}

\visible<2->{%
Using this, the best approximation of $\vecnot{v} \in V$ by vectors in $W$ is equal to
\begin{equation*}
\hat{\vecnot{v}} = A \vecnot{s} = A \left( A^H A \right)^{-1} A^H \vecnot{v} .
\end{equation*}
The matrix \textcolor{blue}{$P = A \left( A^H A \right)^{-1} A^H$ is the projection matrix} for the range of $A$.
It defines an orthogonal projection onto the range of $A$ (i.e., the subspace spanned by the columns of $A$).}

\end{frame}

\begin{frame}{4.3.3: Weighted Least-Squares Solution of a Linear System}

For the standard Euclidean norm $\| \vecnot{v} \|_E = \sqrt{\vecnot{v}^H \vecnot{v}}$ and any invertible $B$, consider the weighted least-squares problem
\[ \min_{\hat{\vecnot{v}}\in W} \| B(\vecnot{v} - \hat{ \vecnot{v}}) \|_E = \min_\vecnot{s} \| B( \vecnot{v} - A \vecnot{s}) \|_E \]
But, $\| B \vecnot{v} \|_E$ equals the induced norm $\| \vecnot{v} \|$ for the weighted inner product
\[ \left\langle \vecnot{u} | \vecnot{v} \right\rangle  \triangleq \vecnot{v}^H B^H B \vecnot{u}. \]

For the weighted inner product, the normal equations look the same
\[ \left\langle \vecnot{v} - \hat{\vecnot{v}} | \vecnot{w}_j \right\rangle
= \left\langle \vecnot{v} - A \vecnot{s} | \vecnot{w}_j \right\rangle
= 0, \quad j \in [n]. \]
but they solve a different problem and they reduce to
\begin{equation*}
\vecnot{0} = \left[ \begin{array}{c} \vecnot{w}_1^H \\ \vdots \\ \vecnot{w}_n^H \end{array} \right] B^H B \left( \vecnot{v} - A \vecnot{s} \right) = A^H B^H B \vecnot{v} - A^H B^H B A \vecnot{s}
\end{equation*}

\end{frame}

\iffalse
\begin{frame}{4.3.4: Expression for Minimum Approximation Error}

Let $\hat{\vecnot{v}} \in W$ be the best approximation of $\vecnot{v}$ by vectors in $W$. Then,
\begin{equation*}
\vecnot{v} = \hat{\vecnot{v}} + \vecnot{e},
\end{equation*}
where $\vecnot{e} \in W^{\bot}$ is the minimum achievable error.
The squared norm of the minimum error is given implicitly by
\begin{equation*}
\left\| \vecnot{v} \right\|^2
= \left\| \hat{\vecnot{v}} + \vecnot{e} \right\|^2
= \left\langle \hat{\vecnot{v}} + \vecnot{e} | \hat{\vecnot{v}} + \vecnot{e} \right\rangle
= \left\langle \hat{\vecnot{v}} | \hat{\vecnot{v}} \right\rangle
+ \left\langle \vecnot{e} | \vecnot{e} \right\rangle
= \left\| \hat{\vecnot{v}} \right\|^2 + \left\| \vecnot{e} \right\|^2 .
\end{equation*}
For the weighted problem, let $H=B^H B$ and write
\begin{equation*}
\begin{split}
\left\| \vecnot{e} \right\|^2
&= \left\| \vecnot{v} \right\|^2
- \left\| \hat{\vecnot{v}} \right\|^2
= \vecnot{v}^H H \vecnot{v} - \hat{\vecnot{v}}^H H \hat{\vecnot{v}} \\
&= \vecnot{v}^H H \vecnot{v} - \vecnot{s}^H A^H H A \vecnot{s} \\
&= \vecnot{v}^H H \vecnot{v}
- \vecnot{v}^H H A \left( A^H H A \right)^{-1} A^H H \vecnot{v} \\
&= \vecnot{v}^H
\left( H -  H A \left( A^H H A \right)^{-1} A^H H \right)
\vecnot{v}.
\end{split}
\end{equation*}
\end{frame}
\fi

\begin{frame}{4.4.2: Linear Minimum Mean-Squared Error Estimation}

Let $Y, X_1, \ldots, X_n$ be zero-mean random variables.
Linear minimum mean-squared error (LMMSE) estimation finds $s_1, \ldots, s_n$ such that
\[ \hat{Y} = s_1 X_1 + \cdots + s_n X_n \]
minimizes the mean squared-error $\Expect[|Y-\hat{Y}|^2]$.
Using the inner product
\begin{equation*}
\left\langle X | Y \right\rangle = \Expect \left[ X \overline{Y} \right],
\end{equation*}
the normal equations for the LMMSE estimate $\hat{Y}$ are $G \vecnot{s} = \vecnot{t}$, where
\begin{equation*}
G = \left[ \begin{array}{cccc}
\Expect \left[ X_1 \overline{X}_1 \right]
& \Expect \left[ X_2 \overline{X}_1 \right] & \cdots
& \Expect \left[ X_n \overline{X}_1 \right] \\
\Expect \left[ X_1 \overline{X}_2 \right]
& \Expect \left[ X_2 \overline{X}_2 \right] & \cdots
& \Expect \left[ X_n \overline{X}_2 \right] \\
\vdots & \vdots & \ddots & \vdots \\
\Expect \left[ X_1 \overline{X}_n \right]
& \Expect \left[ X_2 \overline{X}_n \right] & \cdots
& \Expect \left[ X_n \overline{X}_n \right] \\
\end{array} \right], \quad \vecnot{t} = \left[ \begin{array}{c}
\Expect \left[ Y \overline{X}_1 \right] \\
\Expect \left[ Y \overline{X}_2 \right] \\ \vdots \\
\Expect \left[ Y \overline{X}_n \right] \end{array} \right].
\end{equation*}

If $G$ is invertible, then $\vecnot{s} = G^{-1} \vecnot{t}$ implies $\Expect [|\hat{Y}|^2] = \vecnot{s}^H G \vecnot{s} = \vecnot{s}^H \vecnot{t}$ and
\begin{equation*}
%\left\| Y - \hat{Y} \right\|^2 =
\left\| Y \right\|^2 - \left\| \hat{Y} \right\|^2 
= \Expect \left[ |Y|^2 \right]
- \Expect \left[ |\hat{Y}|^2 \right] = \Expect \left[ |Y|^2 \right] - \vecnot{s}^H \vecnot{t}.
\end{equation*}

\end{frame}

\begin{frame}{4.5.1: Dual Approximation and Minimum-Norm Solutions}

An underdetermined system of linear equations has an infinite number of solutions.
It often makes sense to prefer the \textcolor{blue}{minimum-norm solution}.

\vspace{1.5mm}

Let $V$ be a Hilbert space and $\vecnot{w}_1 ,\vecnot{w}_2 , \ldots, \vecnot{w}_n$ be a basis for subspace $W$.
For any $\vecnot{v} \in V$, the best approximation of $\vecnot{v}$ in $W$ can be found by solving
\begin{equation} \label{eqn:DualNormalEquations}
\left[ \begin{array}{cccc}
\left\langle \vecnot{w}_1 | \vecnot{w}_1 \right\rangle
& \left\langle \vecnot{w}_2 | \vecnot{w}_1 \right\rangle & \cdots
& \left\langle \vecnot{w}_n | \vecnot{w}_1 \right\rangle \\
\left\langle \vecnot{w}_1 | \vecnot{w}_2 \right\rangle
& \left\langle \vecnot{w}_2 | \vecnot{w}_2 \right\rangle & \cdots
& \left\langle \vecnot{w}_n | \vecnot{w}_2 \right\rangle \\
\vdots & \vdots & \ddots & \vdots \\
\left\langle \vecnot{w}_1 | \vecnot{w}_n \right\rangle
& \left\langle \vecnot{w}_2 | \vecnot{w}_n \right\rangle & \cdots
& \left\langle \vecnot{w}_n | \vecnot{w}_n \right\rangle
\end{array} \right]
\left[ \begin{array}{c}
s_1 \\ s_2 \\ \vdots \\ s_n \end{array} \right]
= \left[ \begin{array}{c}
\left\langle \vecnot{v} | \vecnot{w}_1 \right\rangle \\
\left\langle \vecnot{v} | \vecnot{w}_2 \right\rangle \\ \vdots \\
\left\langle \vecnot{v} | \vecnot{w}_n \right\rangle \end{array} \right] .
\end{equation}

\vspace{-1mm}

\begin{theorem}
Let $V$ be a Hilbert space and $\vecnot{w}_1 ,\vecnot{w}_2 , \ldots, \vecnot{w}_n$ be a basis for $W\subseteq V$.
The \textcolor{blue}{dual approximation} problem is to find the minimum-norm vector $\vecnot{w}\in V$ satisfying $\left\langle \vecnot{w} | \vecnot{w}_i \right\rangle = c_i$ for $i=1,\ldots,n$.
Then, the solution $\vecnot{w}$ satisfies \vspace{-3mm}
\[ \vecnot{w} = \sum_{i=1}^n s_i \vecnot{w}_i \in W, \vspace{-2.75mm} \]
where $s_1,s_2,\ldots,s_n$ can be found by solving \eqref{eqn:DualNormalEquations} with $\left\langle \vecnot{v} | \vecnot{w}_i \right\rangle = c_i$.
\end{theorem}

\end{frame}

\begin{frame}{4.5.2: Minimum-Norm Solutions}

\visible<1->{%
For $A \in \ComplexNumbers^{m\times n}$ with $m<n$ and $\vecnot{v} \in \ComplexNumbers^m$, consider the underdetermined linear system $A \vecnot{s} = \vecnot{v}$.
Then, the dual approximation theorem can be applied to solve the minimum-norm problem
\[ \min_{\vecnot{s} : A\vecnot{s} = \vecnot{v}} \| \vecnot{s} \| . \]
}

\vspace{1mm}

\visible<2->{%
To see this as a dual approximation, we can rewrite the constraint $A \vecnot{s} = \vecnot{v}$ as $B^H \vecnot{s} = \vecnot{v}$ where $B = A^H$.
Then, the theorem concludes that the minimum-norm solution lies in  the column space of $B=A^H$.}

\vspace{4mm}

\visible<3->{%
Using $\vecnot{s} \in \mathcal{R}(A^H)$, there is a $\vecnot{t}$ such that $\hat{\vecnot{s}} = A^H \vecnot{t}$ and the constraint gives $A (A^H \vecnot{t}) = \vecnot{v}$.
If the rows of $A$ are linearly independent, then the columns of $B=A^H$ are linearly independent and $(B^H B)^{-1} = (A A^H)^{-1}$ exists.

\vspace{4mm}

Thus, the solution $\hat{\vecnot{s}}$ can be obtained in closed form and is given by \vspace{-1mm}
\[ \color{blue} \hat{\vecnot{s}} = A^H \left( A A^H \right)^{-1} \vecnot{v}. \]
}

\end{frame}

\begin{frame} \frametitle{Next Steps}

\begin{itemize}
\setlength\itemsep{5mm}
\item To continue studying after this video -- \vspace{2mm}

\begin{itemize}
 \setlength\itemsep{3mm}
 \item Try the required reading: Course Notes EF 4.2 - 4.3.4
 \item Or the recommended reading: LADR 6C
 \item Also, look at the problems in Assignment 7
\end{itemize}
\end{itemize}

\note{
	\vspace{8mm}
	\begin{enumerate}
		\setlength\itemsep{3mm}
		\color{red}
		\item Here are some options to continue learning this material. (read) \\ [2mm]  That's it for today.  So, I'll see you next time.
	\end{enumerate}
}

\end{frame}


\end{document}


\begin{frame}{6.5: The Four Fundamental Subspaces}

Consider a linear transform mapping $\RealNumbers^n \to \RealNumbers^m$ represented by $A\in \RealNumbers^{m\times n}$

\begin{itemize}
\setlength\itemsep{1mm}

\item<1-> The four fundamental subspaces are: $\mathcal{R}(A)$, $\mathcal{N}(A)$, $\mathcal{R}(A^T)$, $\mathcal{N}(A^T)$

\item<1-> Recall $A^T \in \RealNumbers^{n\times m}$ maps $\RealNumbers^m \to \RealNumbers^n$ and \textcolor{blue}{$\mathcal{R}(A^T)$ is the row space of $A$}

\vspace{3mm}

\item<2-> Notice that $\vecnot{x} \in \RealNumbers^n$ is in the \textcolor{blue}{null space of $A$} \emph{if and only if} \vspace{-1mm}
\[ A \vecnot{x} = \begin{bmatrix} -- \text{row 1} -- \\ -- \text{row 2} -- \\ \vdots \\ -- \text{row m} -- \end{bmatrix} \vecnot{x} = \begin{bmatrix} 0 \\ 0 \\ \vdots \\ 0 \end{bmatrix} \]
\emph{if and only if} \textcolor{blue}{all rows orthogonal to $\vecnot{x}$} under standard inner product

\vspace{3mm}

\item<3-> Thus, the \textcolor{blue}{null space of $A$} is \textcolor{red}{orthogonal} to the \textcolor{blue}{column space of $A^T$} 

\item<4-> Symmetry: \textcolor{blue}{null space of $A^T$} is \textcolor{red}{orthogonal} to the \textcolor{blue}{column space of $A$} 

\vspace{2mm}

\item<5-> In our notation, this means that $\mathcal{N}(A) \bot \mathcal{R}(A^T)$ and $\mathcal{N}(A^T) \bot \mathcal{R}(A)$

\end{itemize}
\end{frame}

\begin{frame}{6.5: The Four Fundamental Subspaces: Linear Equations}

\hspace*{-1mm}
\includegraphics[width=108mm]{strang_fig1.png}

\vspace{-3mm}
$r \triangleq \dim(\mathcal{R}(A))$ implies \textcolor{blue}{$\dim ( \mathcal{N}(A)) = n\!-\!r$} and \textcolor{blue}{$\dim ( \mathcal{N}(A^T)) = m\!-\!r$}

\let\thefootnote\relax\footnotetext{\hspace*{-4mm} {\tiny Figure from ``The Fundamental Theorem of Linear Algebra'' by Gilbert Strang, The American Mathematical Monthly, Nov. 1993 }}

\end{frame}

\begin{frame}{6.5: The Four Fundamental Subspaces: Least Squares}

\hspace*{-4mm}
\includegraphics[width=114mm]{strang_fig2}

Observe $A^T A \colon \RealNumbers^n \to \RealNumbers^n$ is invertible if non-singular (i.e., if $n=r$)

\let\thefootnote\relax\footnotetext{\hspace*{-4mm} {\tiny Figure from ``The Fundamental Theorem of Linear Algebra'' by Gilbert Strang, The American Mathematical Monthly, Nov. 1993 }}

\end{frame}

\begin{frame}{Interlude: Alternative for Linear Systems}

A ``Theorem of the Alternative'' asserts that exactly one of two logical statements is true. The following is a famous example.

\begin{theorem}
For a matrix $A \in \mathbb{R}^{m\times n}$ and a vector $\vecnot{b}\in \mathbb{R}^m$, exactly one of the following statements is true:
\begin{itemize}
\item There exists an $\vecnot{x}\in \mathbb{R}^n$ such that $A\vecnot{x} = \vecnot{b}$
\item There exists a $\vecnot{y} \in \mathbb{R}^m$ such that $A^T \vecnot{y} = \vecnot{0}$ and $\vecnot{y}^T \vecnot{b} \neq 0$.
\end{itemize} 
\end{theorem}

\begin{proof}<2->
\begin{itemize}
\item In the first case, $\vecnot{b} \in \mathcal{R}(A)$ and the four fundamental subspaces imply that $\vecnot{y}^T \vecnot{b} = 0$ (i.e., $\vecnot{y} \perp \vecnot{b}$) if $A^T \vecnot{y} = \vecnot{0}$ (i.e., $\vecnot{y}\in \mathcal{N}(A^T)$).
\item In the second case, $\vecnot{y} \in \mathcal{N}(A^T)$ implies $\vecnot{y} \perp \mathcal{R}(A)$.
So, $\vecnot{y}^T \vecnot{b} \neq 0$ implies $\vecnot{b} \notin \mathcal{R}(A)$. \qedhere
\end{itemize}
\end{proof}

%\visible<3->{Choosing $m=n$, one can also add ``for all $\vecnot{b}\in W$'' to the first statement and ``there exists $\vecnot{b}\in W$'' to the second statement.}

\end{frame}


\begin{frame}{8: Eigenvalue Decomposition}

\begin{definition}<1->
Let $V$ be a vector space over $F$ and let $T\colon V \to V$ be a linear operator.
An \textcolor{blue}{eigenvalue} of $T$ is a scalar $\lambda \in F$ such that there exists a non-zero vector $\vecnot{v} \in V$ with $T \vecnot{v} = \lambda \vecnot{v}$.
Any vector $\vecnot{v}$ such that $T \vecnot{v} = \lambda \vecnot{v}$ is called an \textcolor{blue}{eigenvector} of $T$ associated with the eigenvalue value $\lambda$.
\end{definition}

\begin{definition}<2->
The square matrix $B$ is \textcolor{blue}{diagonalizable} if there is an invertible matrix $S$ (whose columns are eigenvectors) such that $S^{-1} B S = \Lambda$ is diagonal.
\end{definition}

%\begin{lemma}
%Let $A$ be a Hermitian matrix (i.e., $A^H = A$).
%Then, all eigenvalues of $A$ are real and eigenvectors with different eigenvalues are orthogonal.
%\end{lemma}

%\begin{proof}
%First, we notice that $A = A^H$ implies $\vecnot{v}^H A \vecnot{v}$ is real because \vspace{-1mm}
%\[ \overline{s} = \left( \vecnot{v}^H A \vecnot{v} \right)^H = \vecnot{v}^H A^H \vecnot{v} = \vecnot{v}^H A \vecnot{v} = s. \vspace{-1mm} \]
%If $A \vecnot{v} = \lambda_1 \vecnot{v}$, left multiplication by $\vecnot{v}^H$ shows $\vecnot{v}^H A \vecnot{v} = \lambda_1 \vecnot{v}^H \vecnot{v} = \lambda_1 \| \vecnot{v} \|$.
%Thus, $\lambda_1$ is real.
%Next, assume that $A \vecnot{w} = \lambda_2 \vecnot{w}$ and $\lambda_2 \neq \lambda_1$ so that \vspace{-1mm}
%\[ \lambda_1 \lambda_2 \vecnot{w}^H \vecnot{v} = \vecnot{w}^H A^H A \vecnot{v} = \vecnot{w} A^2 \vecnot{v} = \lambda_1^2 \vecnot{w}^H \vecnot{v}. \]
%We also assume, without loss of generality, that $\lambda_1 \neq 0$.
%Therefore, if $\lambda_2 \neq \lambda_1$, then $\vecnot{w}^H \vecnot{v} = 0$ and the eigenvectors are orthogonal.
%\end{proof}

\begin{theorem}<3->
Any Hermitian matrix $B$ can be diagonalized by a unitary matrix $U$ so that $U^H B U = \Lambda$ is a real-valued diagonal matrix.
\end{theorem}

\vspace{2mm}

\visible<4->{
Matrices $A^H A$ and $A A^H$ are always Hermitian and positive semidefinite}

\end{frame}


\begin{frame}{9: Singular Value Decomposition (SVD)}

Idea is to find orthonormal bases for $\RealNumbers^n$ and $\RealNumbers^m$ in which $A$ is diagonal

\begin{itemize}
\setlength\itemsep{3mm}

\item<1-> Let $\vecnot{v}_1, \ldots, \vecnot{v}_r$ be orthonormal eigenvectors of $A^H A$ with positive eigenvalues $\sigma_1^2,\ldots,\sigma_r^2$.
Then, \[\|A \vecnot{v}_i\| ^2 = \vecnot{v}_i^H (A^H A \vecnot{v}_i) =  \vecnot{v}_i^H (\sigma_i^2\vecnot{v}_i) = \sigma_i^2 \]

\item<2-> This implies that $\|A \vecnot{v}_i\| = \sigma_i$.
So $\vecnot{u}_i = \frac{1}{\sigma_i} A\vecnot{v}_i$ has $\| \vecnot{u}_i \| \!=\! 1$ and \vspace{-1mm}
\begin{align*}
A A^H \vecnot{u}_i &= \frac{1}{\sigma_i} A A^H A \vecnot{v}_i = \frac{1}{\sigma_i} \sigma_i^2 A \vecnot{v}_i = \vecnot{u}_i
\\
\vecnot{u}_j^H \vecnot{u}_i &= \left(\frac{1}{\sigma_j} A \vecnot{v}_j \right)^H \left(\frac{1}{\sigma_i} A \vecnot{v}_i \right) =\frac{1}{\sigma_i \sigma_j} \vecnot{v}_j^H (A^H A) \vecnot{v}_i = \delta_{i,j}  \vspace{-1mm}
\end{align*}


\item<3-> For $U_1 = [\vecnot{u}_1,\ldots,\vecnot{u}_r]
$ and $V_1 = [\vecnot{v}_1,\ldots,\vecnot{v}_r]$, this gives $A V_1 = U_1 \Sigma_1$ where $\Sigma_1$ is a $r \times r$ diagonal matrix with diagonal entries $\sigma_1,\ldots,\sigma_r$

\item<4-> Solving for $A$ gives the \textcolor{blue}{compact SVD} \vspace{-1mm} \[ \color{blue} A = U_1 \Sigma V_1^H, \vspace{-1mm}\]
where the \textcolor{blue}{columns of $U_1, V_1$ are orthonormal bases for $\mathcal{R}(A), \mathcal{R}(A^H)$}

\end{itemize}


\end{frame}

\begin{frame}{9: The Four Fundamental Subspaces: Orthogonal Bases}

\vspace{-1mm}
\hspace*{-1mm}
\includegraphics[width=102mm]{strang_fig3}

\vspace{-3mm}
For $V = [\vecnot{v}_1,\ldots,\vecnot{v}_n]
$ and $U = [\vecnot{u}_1,\ldots,\vecnot{u}_m]$, $A V = U \Sigma$ where $\Sigma \in \RealNumbers^{m \times n}$ has diagonal $\sigma_1,\ldots,\sigma_r$.  Thus, $A = U \Sigma V^H$.


\let\thefootnote\relax\footnotetext{\hspace*{-4mm} {\tiny Figure from ``The Fundamental Theorem of Linear Algebra'' by Gilbert Strang, The American Mathematical Monthly, Nov. 1993 }}

\end{frame}

\begin{frame}{6.5: The Four Fundamental Subspaces: Pseudo-Inverse}

\hspace*{-4mm}
\includegraphics[width=114mm]{strang_fig4}

\let\thefootnote\relax\footnotetext{\hspace*{-4mm} {\tiny Figure from ``The Fundamental Theorem of Linear Algebra'' by Gilbert Strang, The American Mathematical Monthly, Nov. 1993 }}

\end{frame}

\begin{frame}{9.2: Singular Value Decomposition Example}

\visible<1->{%
Consider the matrix 
\[ A = \left[ \begin{array}{cc}
1 & 1 \\
5 & -1 \\
-1 & 5 
\end{array} \right]. \]}

\visible<2->{%
An eigenvalue decomposition of $A^H A$ is given by
\[ \!\!\!\!\!\!\!\!\!\!\!\! A^H A \!=\! \left[ \begin{array}{cc}
27 & -9 \\
-9 & 27
\end{array} \right]
\!=\! V \Lambda V^H
\!=\! \left( \frac{1}{\sqrt{2}}\left[ \begin{array}{cc}
-1 & 1 \\
1 & 1 
\end{array} \right] \right)
\left[ \begin{array}{cc}
36 & 0 \\
0 & 18
\end{array} \right]
\left( \frac{1}{\sqrt{2}} \left[ \begin{array}{cc}
-1 & 1 \\
1 & 1 
\end{array} \right] \right)
\]}

\visible<3->{%
This implies $\Sigma_1 = \Lambda^{1/2}$ and $V_1 = V$.
Thus, we find $U_1 = A V_1 \Sigma_1^{-1}$ with
\[ U_1 = \left[ \begin{array}{cc}
1 & 1 \\
5 & -1 \\
-1 & 5 
\end{array} \right]
\left( \frac{1}{\sqrt{2}} \left[ \begin{array}{cc}
-1 & 1 \\
1 & 1
\end{array} \right] \right)
\left[ \begin{array}{cc}
\frac{1}{\sqrt{36}} & 0 \\
0 & \frac{1}{\sqrt{18}}
\end{array} \right]
= \left[ \begin{array}{cc}
0 & \frac{1}{3} \\
\frac{1}{\sqrt{2}} & \frac{2}{3}  \\
-\frac{1}{\sqrt{2}} & \frac{2}{3} 
\end{array} \right] \]}

\vspace{-1mm}

\visible<4->{%
Putting this all together, we have the compressed SVD
\[ A = U_1 \Sigma_1 V_1^H 
= \left[ \begin{array}{cc}
0 & \frac{1}{3} \\
\frac{1}{\sqrt{2}} & \frac{2}{3}  \\
-\frac{1}{\sqrt{2}} & \frac{2}{3} 
\end{array} \right]
\left[ \begin{array}{cc}
\sqrt{36} & 0 \\
0 & \sqrt{18}
\end{array} \right]
\left( \frac{1}{\sqrt{2}} \left[ \begin{array}{cc}
-1 & 1 \\
1 & 1 
\end{array} \right] \right). \]}

\end{frame}

\begin{frame}{Moore--Penrose Pseudo Inverse}

For a matrix $A \in \ComplexNumbers^{m\times n}$, the matrix $A^+ \in \ComplexNumbers^{n\times m}$ is the pseudo-inverse iff:
\begin{enumerate}
\item $A A^+ A = A$ (implies $A A^+$ is idempotent)
\item $A^+ A A^+ = A^+$ (implies $A^+ A$ is idempotent)
\item $(A A^+)^H = A A^+$ (implies $A A^+$ is Hermitian)
\item $(A^+ A)^H = A^+ A$ (implies $A^+ A$ is Hermitian)
\end{enumerate}

\begin{lemma}
From the compact SVD $A = U_1 \Sigma_1 V_1^H$, one finds that $A^+ = V_1
\Sigma_1^{-1} U_1^H$.
\end{lemma}

\begin{proof}<2->
\begin{enumerate}
\item $A A^+ A = U_1 {\color{blue}(\Sigma_1 V_1^H  V_1
\Sigma_1^{-1} U_1^H  U_1)} \Sigma_1 V_1^H = A $
\item $A^+ A A^+ =  V_1
{\color{blue}(\Sigma_1^{-1} U_1^H U_1 \Sigma_1 V_1^H  V_1)}
\Sigma_1^{-1} U_1^H  = A^+$
\item $(A A^+)^H = \big(U_1 {\color{blue}(\Sigma_1 V_1^H V_1
\Sigma_1^{-1})} U_1^H \big)^H  = {U_1 U_1^H} = A A^+$
\item $(A^+ A)^H = \big(V_1 {\color{blue}(\Sigma_1^{-1} U_1^H U_1
\Sigma_1)} V_1^H \big)^H  = {V_1 V_1^H} = A^+ A$ \qedhere
\end{enumerate}
\end{proof}

\visible<2->{Thus, $A A^+$ and $A^+ A$ are projection matrices onto $\mathcal{R}(A)$ and $\mathcal{R}(A^H)$.}

\end{frame}

\backupbegin

%\begin{frame}
%\frametitle{Backup Slides}
%\begin{itemize}
%\item Slide numbers not included in denominator!
%\end{itemize}
%\end{frame}

%\begin{frame}[allowframebreaks]
%\frametitle{References}
%\bibliographystyle{alpha}
%\footnotesize
%\bibliography{IEEEabrv,WCLabrv,WCLbib,WCLnewbib}
%\end{frame}

\backupend

\end{document}
