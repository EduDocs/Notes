\documentclass[10pt,english,aspectratio=169]{beamer}
% Use notes or hide notes or show only notes or handout

\usetheme{default}

\usepackage{xstring}
\usepackage{pgfpages}
%\makeatletter
%\IfSubStr{\@classoptionslist}{handout}
%  {\pgfpagesuselayout{2 on 1}[letterpaper,border shrink=5mm]}
%  {}
%\makeatother

\usepackage{amsmath,amssymb,amsthm}
\usepackage{stmaryrd}
\usepackage{enumerate}
\usepackage{stfloats}
\usepackage{bbm}
\usepackage{pdfpages}
\usepackage{framed}

\usepackage[most]{tcolorbox}
\tcbset{highlight math style={enhanced,
  colframe=white,colback=yellow!15,arc=8pt,boxrule=1pt,
  }}
  
\usepackage{tikz,pgf,pgfplots}
\usepackage{algorithm,algorithmic}
\usepgflibrary{shapes}
\usetikzlibrary{%
  arrows,%
  arrows.meta,
  backgrounds,
  shapes.misc,% wg. rounded rectangle
  shapes.arrows,%
  shapes,%
  calc,%
  chains,%
  matrix,%
  positioning,% wg. " of "
  scopes,%
  decorations.pathmorphing,% /pgf/decoration/random steps | erste Graphik
  shadows,%
  backgrounds,%
  fit,%
  petri,%
  quotes
}

\tikzset{background rectangle/.style={
    fill=white,
  },
  use background/.style={    
    show background rectangle
  }
}

\setbeamersize{text margin left=10mm,text margin right=35mm}

\pgfplotsset{compat=1.12}

%\usetheme{Frankfurt}
%\usecolortheme{ldpc}
\useinnertheme{rounded}
\usecolortheme{whale}
\usecolortheme{orchid}

\newcommand{\ul}[1]{\underline{#1}}
\renewcommand{\Pr}{\mathbb{P}}

%% Setup slides and notes
\makeatletter
\IfSubStr{\@classoptionslist}{notes} { \IfSubStr{\@classoptionslist}{hide} {}{\IfSubStr{\@classoptionslist}{only} {}{\setbeameroption{show notes on second screen=right}}} }{}
\makeatother
%\setbeamertemplate{note page}{\pagecolor{yellow!5}\vfill\insertnote\vfill}

\newcommand{\getpdfpages}[2]{\begingroup
  \setbeamercolor{background canvas}{bg=}
  \addtocounter{framenumber}{1}
  \includepdf[pages={#1},%
  pagecommand={%
    \expandafter\def\expandafter\insertshorttitle\expandafter{%
      \insertshorttitle\hfill\insertframenumber\,/\,\inserttotalframenumber}}%
  ]{#2}
  \endgroup}

\newcommand{\backupbegin}{
   \newcounter{finalframe}
   \setcounter{finalframe}{\value{framenumber}}
}
\newcommand{\backupend}{
   \setcounter{framenumber}{\value{finalframe}}
}

 \setbeamercolor{bibliography entry author}{fg=black}
 \setbeamercolor{bibliography entry title}{fg=black}
 \setbeamercolor{bibliography entry location}{fg=black}
 \setbeamercolor{bibliography entry note}{fg=black}
 
 \setbeamerfont{bibliography item}{size=\footnotesize}
 \setbeamerfont{bibliography entry author}{size=\footnotesize}
 \setbeamerfont{bibliography entry title}{size=\footnotesize}
 \setbeamerfont{bibliography entry location}{size=\footnotesize}
 \setbeamerfont{bibliography entry note}{size=\footnotesize}
 \setbeamertemplate{bibliography item}{\insertbiblabel}
 
\newlength\tikzwidth
\newlength\tikzheight


\newcommand{\mc}[1]{\mathcal{#1}}
\newcommand{\mbb}[1]{\mathbb{#1}}
%\newcommand{\expt}{\mbb{E}}
%\newcommand{\dd}{\mathrm{d}}
\newcommand{\Interior}[1]{\ensuremath{{#1}^{\circ}}}
\newcommand{\Closure}[1]{\ensuremath{\overline{#1}}}
\newcommand{\Complement}[1]{\ensuremath{{#1}^{c}}}

\newcommand{\Expect}{\ensuremath{\mathrm{E}}}
\newcommand{\vecnot}{\underline}
\newcommand{\RealNumbers}{\ensuremath{\mathbb{R}}}
\newcommand{\RationalNumbers}{\mathbb{Q}}
\newcommand{\ComplexNumbers}{\mathbb{C}}
\newcommand{\Real}{\mathrm{Re}}
\newcommand{\Span}{\mathrm{span}}
\newcommand{\Rank}{\mathrm{rank}}
\newcommand{\Nullity}{\mathrm{nullity}}
\newcommand{\Trace}{\mathrm{tr}}
\newcommand{\Diag}{\mathrm{diag}}
\newcommand{\dd}{\mathrm{d}}
\DeclareMathOperator*{\esssup}{ess\,sup}

% Use < , > inner product
\newcommand{\inner}[2]{{\left\langle #1 \mskip2mu , #2 \right\rangle}}
\newcommand{\tinner}[2]{{\langle #1 \mskip1mu , #2 \rangle}}

% Use < | > inner product
%\newcommand{\inner}[2]{{\left\langle #1 \mskip2mu \middle| \mskip2mu #2 \right\rangle}}
%\newcommand{\tinner}[2]{{\langle #1 \mskip1mu | \mskip1mu  #2 \rangle}}




\def\checkmark{\tikz\fill[scale=0.4](0,.35) -- (.25,0) -- (1,.7) -- (.25,.15) -- cycle;}
\def\greencheck{{\color{green}\checkmark}}
\def\scalecheck{\resizebox{\widthof{\checkmark}*\ratio{\widthof{x}}{\widthof{\normalsize x}}}{!}{\checkmark}}
\def\xmark{\tikz [x=1.4ex,y=1.4ex,line width=.2ex, red] \draw (0,0) -- (1,1) (0,1) -- (1,0);}
\def\redx{{\color{red}\xmark}}

\renewcommand{\footnotesep}{-2pt}


\begin{document}

\title{ECE 586: Vector Space Methods \\ Lecture 9 Flip Video: Fields and Matrices}
\author{Henry D. Pfister \\ Duke University}
\date{}
%\date{August 20th, 2020}
%\maketitle

\setbeamertemplate{navigation symbols}{}

\begin{frame}[plain]
	\titlepage
	
	\note{
		\vspace{8mm}
		\begin{enumerate}
			\setlength\itemsep{3mm}
			\color{red}
			\item Welcome to the 9th video lecture for ECE 586, Vector Space Methods. \\[2mm]
			Today, we'll discuss fields and matrices.
		\end{enumerate}
	}
\end{frame}

\addtocounter{framenumber}{-1}
\setbeamertemplate{navigation symbols}{\textcolor{blue}{\footnotesize \insertframenumber ~/ \inserttotalframenumber}}


\begin{frame}<1-2> \frametitle{3.1: Linear Algebra and Abstraction}

\begin{itemize}
\setlength\itemsep{5mm}
\item<1-> Linear algebra is the foundation of engineering mathematics \vspace{1mm}
  \begin{itemize}
  \setlength\itemsep{1.5mm}
  \item This chapter will review and extend your knowledge of linear algebra
  \item Abstraction allows the application of similar ideas to many different areas
  \item Your prior experience with real vector spaces makes this possible 
  \end{itemize}
\item<2-> Abstraction \vspace{1mm}
  \begin{itemize}
  \setlength\itemsep{1.5mm}
  \item Metric spaces provide abstract notions of distance and closeness
  \item Fields provide abstract notions for arithmetic of scalars ($+,-,\times,/$)
  \item Vector spaces abstract the additive properties of real vectors
  \end{itemize}
\end{itemize}

\note{
	\vspace{2mm}
	\begin{enumerate}[<alert@+>]
		\footnotesize
		\setlength\itemsep{3mm}
		\item Read.
		\item Abstraction is an important part of this class. Read. Building the abstraction up slowly allows one to discover what elements are important for each result.
	\end{enumerate}
}

\end{frame}

\begin{frame}<1-2> \frametitle{3.1: Fields}

Consider a set $F$ with binary operations: ``addition" and ``multiplication"
\begin{itemize}
\item For every pair of elements $s, t \in F$, we require $(s + t) \in F$
\item For every pair of elements $s, t \in F$, we require $s t \in F$
\end{itemize}
\vspace{1.5mm}
\uncover<2->{%
$F$ forms a \textcolor{blue}{field} if the two operations satisfy:
\begin{enumerate}
\item there is a unique element $0 \in F$ such that $s + 0 = s \;\;\; \forall s \in F$
\item addition is commutative: $s + t = t + s \;\;\; \forall s, t \in F$
\item addition is associative: $r + (s + t) = (r + s) + t \;\;\; \forall r, s, t \in F$
\item for $s \in F$ there is a unique element $(-s) \in F$ such that $s + (-s) = 0$
\item multiplication is commutative: $st = ts \;\;\; \forall s,t \in F$
\item multiplication is associative: $r(st) = (rs)t \;\;\; \forall r, s, t \in F$
\item there is a unique non-zero element $1 \in F$ such that $s1 = s \;\;\; \forall s \in F$
\item for $s \in F \setminus\{0\}$ there is a unique element $s^{-1} \in F$ such that $s s^{-1} = 1$
\item mult. distributes over addition: $r (s + t) = rs + rt \;\;\; \forall r, s, t \in F$.
\end{enumerate}
}

\note{
	\vspace{2mm}
	\begin{enumerate}[<alert@+>]
		\footnotesize
		\setlength\itemsep{3mm}
		\item Read.  Multiplication is denoted by placing element next to each other.
		\item A field satisfies all the standard multiplicative and additive properties of the real numbers.  They are listed here for your review but you do not need to memorize them.
	\end{enumerate}
}

\end{frame}

\begin{frame}<1> \frametitle{3.1: Examples of Arithmetic Fields}

\begin{example}
The rational numbers with standard addition and multiplication form a field.
\end{example}

\begin{example}
The real numbers with standard addition and multiplication form a field.
\end{example}

\begin{example}
The complex numbers with standard addition and multiplication form a field.
\end{example}

\begin{example}
The set of integers with standard addition and multiplication is not a field.
\end{example}

\begin{example}
For prime $p$, set $\{0,1,\ldots,p-1\}$ with add. and mult. modulo $p$ is a field.
\end{example}

\note{
	\vspace{2mm}
	\begin{enumerate}[<alert@+>]
		\footnotesize
		\setlength\itemsep{3mm}
		\item Read.  The last example is a field with a finite number of elements.  Such fields are be useful in communications and computing. 
	\end{enumerate}
}

\end{frame}

\begin{frame}<1-2> \frametitle{3.2: Matrices}

Consider finding $n$ scalars $x_1, \ldots, x_n \in F$ that satisfy these conditions \vspace{-0.5mm}
\begin{equation*} \begin{array}{ccccccccc}
a_{11} x_1 & + & a_{12} x_2 & + & \cdots & + & a_{1n} x_n & = & y_1 \\
a_{21} x_1 & + & a_{22} x_2 & + & \cdots & + & a_{2n} x_n & = & y_2 \\
\scalebox{0.75}{\vdots} & & \scalebox{0.75}{\vdots}  & & & & \scalebox{0.75}{\vdots} & & \scalebox{0.75}{\vdots} \\
a_{m1} x_1 & + & a_{m2} x_2 & + & \cdots & + & a_{mn} x_n & = & y_m
\end{array}
\end{equation*}
%where $y_j \in F$ and $a_{ij} \in F$.

\uncover<2->{%
They define a \textcolor{blue}{system of $m$ linear equations in $n$ unknowns $A \vecnot{x} = \vecnot{y},$}
where
$\vecnot{x} = (x_1, \ldots, x_n)^T$, 
$\vecnot{y} = (y_1, \ldots, y_m)^T$, and
$A$ is the matrix given by
\begin{equation*}
A = \begin{bmatrix}
a_{11} & a_{12} & \cdots & a_{1n} \\
a_{21} & a_{22} & \cdots & a_{2n} \\
\scalebox{0.75}{\vdots} & \scalebox{0.75}{\vdots} & \scalebox{0.75}{$\ddots$} & \scalebox{0.75}{\vdots} \\
a_{m1} & a_{m2} & \cdots & a_{mn}
\end{bmatrix}.
\end{equation*}
\vspace{-2mm}
}

\note{
	\vspace{2mm}
	\begin{enumerate}[<alert@+>]
		\footnotesize
		\setlength\itemsep{3mm}
		\item Read.
		\item Read.  Thus, matrices can be used to define systems of linear equations.  They are also used to represent linear transformations.
	\end{enumerate}
}

\end{frame}

\begin{frame}<1-2> \frametitle{3.2: Matrix Multiplication}

\vspace{-1mm}

\begin{definition}<1->
Let $A$ be an $m \times n$ matrix over $F$ (i.e., $A \in F^{m\times n}$) and $B$ be an $n \times p$ matrix over $F$.
The \textcolor{blue}{matrix product} $AB$ is the $m \times p$ matrix $C$ whose $i,j$ entry is \vspace{-3mm}
\begin{equation*}
c_{ij} = \sum_{r = 1}^n a_{ir} b_{rj}. \vspace{-0.25mm}
\end{equation*}
\end{definition}


\begin{example}<2->
When $j$ is fixed, one can eliminate $i$ by grouping the elements of $C$ and $A$ into column vectors $\vecnot{c}_1,\ldots,\vecnot{c}_{p}$ and $\vecnot{a}_1,\ldots,\vecnot{a}_n$. In this case, we see that the $j$-th column of $C$ is a linear combination of the columns of $A$, \vspace{-3mm}
\[ \vecnot{c}_{j} = \sum_{r = 1}^n \vecnot{a}_r b_{rj}, \vspace*{-2.5mm} \]

Likewise, grouping $C$ and $B$ into row vectors  $\vecnot{c}_1,\ldots,\vecnot{c}_{m}$ and $\vecnot{b}_1,\ldots,\vecnot{b}_n$ gives \vspace{-3mm}
\[ \vecnot{c}_{i} = \sum_{r = 1}^n a_{ir} \vecnot{b}_r. \vspace{-0.25mm} \]
% with coefficients $b_{i,1},b_{i,2},\ldots,b_{i,n}$.
\end{example}

%\only<3->{\begin{tikzpicture}[overlay,remember picture]
%\fill[opacity=0] (current page.south east) 
%rectangle (current page.north west) 
%node[align=center,pos=0.18,opacity=1,fill=yellow!20,scale=0.95]
%{Example\\[1mm]$\mathbf{V} = \begin{bmatrix}
%1 & 1 & 0 & 1 & 1 & 0 & 0 & 1 \\
%1 & 1 & 0 & 0 & 1 & 0 & 1 & 0 \\
%1 & 1 & 0 & 0 & 0 & 1 & 0 & 0 \\
%1 & 0 & 1 & 0 & 0 & 0 & 0 & 0 \\
%\end{bmatrix}$
%};
%\end{tikzpicture}}

\note{
	\vspace{2mm}
	\begin{enumerate}[<alert@+>]
		\footnotesize
		\setlength\itemsep{3mm}
		\item Read.  This is standard matrix multiplication, which you should have all seen before.
		\item Thus, if we interpret matrices as collection of row (or column) vectors, then matrix multiplication also has other interpretations.
	\end{enumerate}
}

\end{frame}

% Add example!!!

\begin{frame}<1-2> \frametitle{3.2: Echelon Forms}

\begin{definition}<1->
A matrix $A \in F^{m\times n}$ is in \textcolor{blue}{row echelon form} if:
\begin{enumerate}
\item Any all-zero row is below all rows with non-zero entries, and
\item For all rows with non-zero entries, the leading coefficient (i.e., the first non-zero element from the left) is strictly to the right of the leading coefficient of the row above it.
\end{enumerate}
Thus, the entries below the leading coefficient in a column are zero. \\
$A$ is in \textcolor{blue}{column echelon form} if $A^T$ is in row echelon form.
\end{definition}

\begin{definition}<2->
A matrix $B \in F^{m\times n}$ is in \textcolor{blue}{reduced row echelon form} (RREF) if it is:
\begin{enumerate}
\item in row echelon form with every leading coefficient equal to 1, and
\item every leading coefficient is the only non-zero element in its column.
\end{enumerate}
$B$ is \textcolor{blue}{reduced column echelon form} if $B^T$ is in reduced row echelon form.
\end{definition}

\uncover<1->{\begin{tikzpicture}[overlay,remember picture,shift={(13.3,3.6)}]
\node[opacity=1,fill=brown!30,align=center,rounded corners] at (0,0)%
{$A \!=\! \begin{bmatrix}
1 & 0 & 2 & 0 \\
0 & 0 & \!\!-2 & 1 \\
0 & 0 & 0 & 0 \\
\end{bmatrix}\!\!\!$
};
\end{tikzpicture}}

\uncover<2->{\begin{tikzpicture}[overlay,remember picture,shift={(13.3,1.20)}]
\node[opacity=1,fill=brown!30,align=center,rounded corners] at (0,0)%
{$B \!=\! \begin{bmatrix}
1 & 0 & 0 & 1 \\
0 & 0 & 1 & \!\!\!-\frac{1}{2} \\
0 & 0 & 0 & 0 \\
\end{bmatrix}\!\!\!$
};
\end{tikzpicture}}


\note{
	\vspace{2mm}
	\begin{enumerate}[<alert@+>]
		\footnotesize
		\setlength\itemsep{3mm}
		\item Read.
		\item Read.  This form allows one to easily find vectors in the null space of the matrix.
	\end{enumerate}
}

\end{frame}

\begin{frame}<1-3> \frametitle{3.2: Elementary Row Operations}

\vspace{-1.5mm}

\begin{definition}<1->
An \textcolor{blue}{elementary row operation} on an $m \times n$ matrix consists of
\begin{enumerate}
\setlength\itemsep{0.5mm}
\item multiplying a row by a non-zero scalar,
\item swapping two rows, or
\item adding a scalar multiple of one row to another row.
\end{enumerate}
An \textcolor{blue}{elementary column operation} is the same but applied to the columns.
\end{definition}


\begin{definition}<2->
A square matrix $A \in F^{m\times m}$ is \textcolor{blue}{invertible} if there is a square matrix $B \in F^{m\times m}$ such that $AB=BA=I$, where $I$ is the $m\!\times\! m$ identity.  In that case, $A^{-1}=B$.
\end{definition}

\vspace{1mm}
\begin{itemize}
  \item<3-> Elementary row operations on $A\in F^{m\times n}$ \vspace{1mm}
  \begin{itemize}
    \setlength{\itemsep}{1.5mm}
    \item <3-> Are realized, for an invertible matrix $E \in F^{m\times m}$, by matrix mult as $EA$
    \item <3-> Can be used in a sequence to induce RREF (e.g., Gaussian elimination)
  \end{itemize}
\end{itemize}

\note{
	\vspace{2mm}
	\begin{enumerate}[<alert@+>]
		\footnotesize
		\setlength\itemsep{3mm}
		\item Read.
		\item Read.
		\item Read.
	\end{enumerate}
}

\end{frame}


\begin{frame}<1-5> \frametitle{3.2: Elementary Row Operations}


\begin{lemma}<1->
For any $m\times n$ matrix $A$ over $F$, there is an invertible $m \times m$ matrix $P$ over $F$ such that $R=PA$ is in reduced row echelon form.
\end{lemma}

\vspace{3mm}

\begin{proof}[Sketch of Proof]<2->
\begin{itemize}
\item<2-> For $A\in F^{m \times n}$, define the augmented matrix $A' = [A \;\; I]$
\item<3-> Use elementary row operations to transform $A'$ into RREF as $R'$
\item<4-> Elementary row operations imply that $R' = P A'$ for some invertible $P$
\item<5-> Then, $R' = [R \;\; P]$ where $R=PA$ is in RREF \qedhere
\end{itemize}
\end{proof}

\note{
	\vspace{2mm}
	\begin{enumerate}[<alert@+>]
		\footnotesize
		\setlength\itemsep{3mm}
		\item Read.
		\item Read.
		\item Read.
	\end{enumerate}
}

\end{frame}



\begin{frame}<1-4> \frametitle{3.2: Foundation of Dimension}

\begin{lemma}<1->
Let $A$ be an $m \times n$ matrix over $F$ with $m<n$.
Then, there exists a length-$n$ column vector $\vecnot{x} \neq \vecnot{0}$ (over $F$) such that $A \vecnot{x} = \vecnot{0}$.
\end{lemma}

\begin{proof}<2->
\begin{itemize}
\item<2-> Use row reduction to compute the reduced row echelon form $R=PA$ of $A$, where $P$ is invertible by previous lemma.

\item<3-> Observe that the columns of $R$ containing leading elements can be combined in a linear combination to cancel any other column of $R$.

\item<4-> Thus, one can construct a vector $\vecnot{x}$ satisfying $R\vecnot{x}=\vecnot{0}$ and observe that $A \vecnot{x} = P^{-1} R \vecnot{x} = \vecnot{0}$. \hfill \qedhere
\end{itemize}
\end{proof}

\vspace{2mm}

\uncover<4->{A concrete example of this process will be given in class.}


\note{
	\vspace{2mm}
	\begin{enumerate}[<alert@+>]
		\footnotesize
		\setlength\itemsep{3mm}
		\item Read.
		\item Read.
		\item Read.
		\item Read.
		\item Read.
	\end{enumerate}
}

\end{frame}

\begin{frame} \frametitle{Next Steps}

\begin{itemize}
\setlength\itemsep{5mm}
\item To continue studying after this video -- \vspace{2mm}

\begin{itemize}
 \setlength\itemsep{3mm}
 \item Try the required reading: Course Notes EF 3.1 - 3.3
 \item Or the recommended reading: LADR Ch. 1, Ch. 2, Ch. 3ABC
 \item Also, look at the problems in Assignment 4
\end{itemize}
\end{itemize}

\note{
	\vspace{8mm}
	\begin{enumerate}
		\setlength\itemsep{3mm}
		\color{red}
		\item Here are some options to continue learning this material. (read) \\ [2mm]  That's it for today.  So, I'll see you next time.
	\end{enumerate}
}

\end{frame}

\end{document}

