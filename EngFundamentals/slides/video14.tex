\documentclass[10pt,english,aspectratio=169]{beamer}
% Use notes or hide notes or show only notes or handout

\usetheme{default}

\usepackage{xstring}
\usepackage{pgfpages}
%\makeatletter
%\IfSubStr{\@classoptionslist}{handout}
%  {\pgfpagesuselayout{2 on 1}[letterpaper,border shrink=5mm]}
%  {}
%\makeatother

\usepackage{amsmath,amssymb,amsthm}
\usepackage{stmaryrd}
\usepackage{enumerate}
\usepackage{stfloats}
\usepackage{bbm}
\usepackage{pdfpages}
\usepackage{framed}

\usepackage[most]{tcolorbox}
\tcbset{highlight math style={enhanced,
  colframe=white,colback=yellow!15,arc=8pt,boxrule=1pt,
  }}
  
\usepackage{tikz,pgf,pgfplots}
\usepackage{algorithm,algorithmic}
\usepgflibrary{shapes}
\usetikzlibrary{%
  arrows,%
  arrows.meta,
  backgrounds,
  shapes.misc,% wg. rounded rectangle
  shapes.arrows,%
  shapes,%
  calc,%
  chains,%
  matrix,%
  positioning,% wg. " of "
  scopes,%
  decorations.pathmorphing,% /pgf/decoration/random steps | erste Graphik
  shadows,%
  backgrounds,%
  fit,%
  petri,%
  quotes
}

\tikzset{background rectangle/.style={
    fill=white,
  },
  use background/.style={    
    show background rectangle
  }
}

\setbeamersize{text margin left=10mm,text margin right=35mm}

\pgfplotsset{compat=1.12}

%\usetheme{Frankfurt}
%\usecolortheme{ldpc}
\useinnertheme{rounded}
\usecolortheme{whale}
\usecolortheme{orchid}

\newcommand{\ul}[1]{\underline{#1}}
\renewcommand{\Pr}{\mathbb{P}}

%% Setup slides and notes
\makeatletter
\IfSubStr{\@classoptionslist}{notes} { \IfSubStr{\@classoptionslist}{hide} {}{\IfSubStr{\@classoptionslist}{only} {}{\setbeameroption{show notes on second screen=right}}} }{}
\makeatother
%\setbeamertemplate{note page}{\pagecolor{yellow!5}\vfill\insertnote\vfill}

\newcommand{\getpdfpages}[2]{\begingroup
  \setbeamercolor{background canvas}{bg=}
  \addtocounter{framenumber}{1}
  \includepdf[pages={#1},%
  pagecommand={%
    \expandafter\def\expandafter\insertshorttitle\expandafter{%
      \insertshorttitle\hfill\insertframenumber\,/\,\inserttotalframenumber}}%
  ]{#2}
  \endgroup}

\newcommand{\backupbegin}{
   \newcounter{finalframe}
   \setcounter{finalframe}{\value{framenumber}}
}
\newcommand{\backupend}{
   \setcounter{framenumber}{\value{finalframe}}
}

 \setbeamercolor{bibliography entry author}{fg=black}
 \setbeamercolor{bibliography entry title}{fg=black}
 \setbeamercolor{bibliography entry location}{fg=black}
 \setbeamercolor{bibliography entry note}{fg=black}
 
 \setbeamerfont{bibliography item}{size=\footnotesize}
 \setbeamerfont{bibliography entry author}{size=\footnotesize}
 \setbeamerfont{bibliography entry title}{size=\footnotesize}
 \setbeamerfont{bibliography entry location}{size=\footnotesize}
 \setbeamerfont{bibliography entry note}{size=\footnotesize}
 \setbeamertemplate{bibliography item}{\insertbiblabel}
 
\newlength\tikzwidth
\newlength\tikzheight


\newcommand{\mc}[1]{\mathcal{#1}}
\newcommand{\mbb}[1]{\mathbb{#1}}
%\newcommand{\expt}{\mbb{E}}
%\newcommand{\dd}{\mathrm{d}}
\newcommand{\Interior}[1]{\ensuremath{{#1}^{\circ}}}
\newcommand{\Closure}[1]{\ensuremath{\overline{#1}}}
\newcommand{\Complement}[1]{\ensuremath{{#1}^{c}}}

\newcommand{\Expect}{\ensuremath{\mathrm{E}}}
\newcommand{\vecnot}{\underline}
\newcommand{\RealNumbers}{\ensuremath{\mathbb{R}}}
\newcommand{\RationalNumbers}{\mathbb{Q}}
\newcommand{\ComplexNumbers}{\mathbb{C}}
\newcommand{\Real}{\mathrm{Re}}
\newcommand{\Span}{\mathrm{span}}
\newcommand{\Rank}{\mathrm{rank}}
\newcommand{\Nullity}{\mathrm{nullity}}
\newcommand{\Trace}{\mathrm{tr}}
\newcommand{\Diag}{\mathrm{diag}}
\newcommand{\dd}{\mathrm{d}}
\DeclareMathOperator*{\esssup}{ess\,sup}

% Use < , > inner product
\newcommand{\inner}[2]{{\left\langle #1 \mskip2mu , #2 \right\rangle}}
\newcommand{\tinner}[2]{{\langle #1 \mskip1mu , #2 \rangle}}

% Use < | > inner product
%\newcommand{\inner}[2]{{\left\langle #1 \mskip2mu \middle| \mskip2mu #2 \right\rangle}}
%\newcommand{\tinner}[2]{{\langle #1 \mskip1mu | \mskip1mu  #2 \rangle}}




\def\checkmark{\tikz\fill[scale=0.4](0,.35) -- (.25,0) -- (1,.7) -- (.25,.15) -- cycle;}
\def\greencheck{{\color{green}\checkmark}}
\def\scalecheck{\resizebox{\widthof{\checkmark}*\ratio{\widthof{x}}{\widthof{\normalsize x}}}{!}{\checkmark}}
\def\xmark{\tikz [x=1.4ex,y=1.4ex,line width=.2ex, red] \draw (0,0) -- (1,1) (0,1) -- (1,0);}
\def\redx{{\color{red}\xmark}}

\renewcommand{\footnotesep}{-2pt}


\begin{document}

\title{ECE 586: Vector Space Methods \\ Lecture 14 Flip Video: Matrix and Operator Norms}
\author{Henry D. Pfister \\ Duke University}
\date{}
%\date{August 20th, 2020}
%\maketitle

\setbeamertemplate{navigation symbols}{}

\begin{frame}[plain]
	\titlepage
	
	\note{
		\vspace{8mm}
		\begin{enumerate}
			\setlength\itemsep{3mm}
			\color{red}
			\item Welcome to the 11th video lecture for ECE 586, Vector Space Methods. \\[2mm]
			Today, we'll finish our discussion of subspaces and bases and then move on to linear transforms.
		\end{enumerate}
	}
\end{frame}

\addtocounter{framenumber}{-1}
\setbeamertemplate{navigation symbols}{\textcolor{blue}{\footnotesize \insertframenumber ~/ \inserttotalframenumber}}






\begin{frame}<1-3> \frametitle{6.1/6.3: Vector Space of Linear Transforms and Norms}

\begin{definition}<1->
Let $L(V,W)$ denote the vector space of all linear transforms from $V$ into $W$, where $V$ and $W$ are vector spaces over a field $F$.
\end{definition}

An operator norm is a norm on a vector space of linear transforms.

\begin{definition}[Induced Operator Norm]<2->
Let $V$ and $W$ be two normed vector spaces and let $T \colon V \rightarrow W$ be a linear transformation.
The induced \textcolor{blue}{operator norm} of $T$ is defined to
\begin{equation*}
\left\| T \right\|_{\mathrm{op}}
= \sup_{ \vecnot{v} \in V - \left\{ \vecnot{0} \right\} }
\frac{ \left\| T \vecnot{v} \right\|_W }{ \left\| \vecnot{v} \right\|_V }
= \sup_{\vecnot{v} \in V, \left\| \vecnot{v} \right\|_V = 1 }
\left\| T \vecnot{v} \right\|_W .
\end{equation*}
\end{definition}

\vspace{1.5mm}

\visible<3->{A common question about the operator norm is, ``How do I know the two expressions give the same result?".
To see this, we can write
\begin{equation*}
\sup_{ \vecnot{v} \in V - \left\{ \vecnot{0} \right\} }
\frac{ \left\| T \vecnot{v} \right\|_W }{ \left\| \vecnot{v} \right\|_V }
= \sup_{ \vecnot{v} \in V - \left\{ \vecnot{0} \right\} }
\left\| T \frac{\vecnot{v}}{\| \vecnot{v} \|_V} \right\|_W
= \sup_{\vecnot{u} \in V, \left\| \vecnot{u} \right\|_V = 1 }
\left\| T \vecnot{u} \right\|_W .
\end{equation*}}

\end{frame}

\begin{frame}<1-3> \frametitle{6.3: Operator Norms}

%To verify the triangle inequality, we can write
%\begin{align*}
%\| T + U \| &= \sup_{\vecnot{v} \in V, \left\| \vecnot{v} \right\| = 1 }
%\left\| (T+U) \vecnot{v} \right\| \\
%&\leq  \sup_{\vecnot{v} \in V, \left\| \vecnot{v} \right\| = 1 }
%\left(  \left\| T \vecnot{v} \right\| + \left\| U \vecnot{v} \right\| \right) \\
%& \leq \sup_{\vecnot{v} \in V, \left\| \vecnot{v} \right\| = 1 }
%\left\| T \vecnot{v} \right\| + \sup_{\vecnot{v} \in V, \left\| \vecnot{v} \right\| = 1 }
%\left\| U \vecnot{v} \right\| \\
%&= \|T\| + \|U\|.
%\end{align*}


The induced operator norm has another property:
% that follows naturally from its definition.
%Notice that 
\[ \| T \| = \sup_{ \vecnot{v} \in V - \left\{ \vecnot{0} \right\} }
\frac{ \left\| T \vecnot{v} \right\| }{ \left\| \vecnot{v} \right\| } \geq \frac{ \left\| T \vecnot{u} \right\| }{ \left\| \vecnot{u} \right\| }, \]
for non-zero $\vecnot{u} \in V$.
This implies that $\| T \vecnot{u} \| \leq \| T \| \| \vecnot{u} \|$ for all non-zero $\vecnot{u}\in V$.
By checking $\vecnot{u}=\vecnot{0}$ separately, one can see it holds for all $\vecnot{u}\in V$.
\vspace{3mm}

\begin{definition}<2->
For the space $L(V,V)$ of linear operators on $V$, a norm is called \textcolor{blue}{submultiplicative} if $\| T U \| \leq \|T\| \|U\|$ for all $T,U \in L(V,V)$.
\end{definition}

\vspace{4mm}

\visible<3->{Using the above inequality, induced operator norms are submultiplicative:
\begin{equation*}
\left\| U T \vecnot{v} \right\| \leq \left\| U \right\| \left\| T \vecnot{v} \right\| \leq \left\| U \right\| \left\| T \right\| \left\| \vecnot{v} \right\| .
\end{equation*}}


\end{frame}


\begin{frame} \frametitle{Continuity of Linear Transforms}

\vspace{-1mm}

\begin{definition}<1->
A linear transform is called \textcolor{blue}{bounded} if its induced operator norm is finite.
\end{definition}

\vspace{-1mm}

\begin{theorem}<2->
A linear transformation $T \colon V \rightarrow W$ is bounded if and only if it is continuous.
\end{theorem}

\vspace{-1mm}

\begin{proof}<3->
Suppose that $T$ is bounded; that is, there exists $M$ such that  $\left\| T \vecnot{v} \right\| \leq M \left\| \vecnot{v} \right\|$ for all $\vecnot{v} \in V$.
Let $\vecnot{v}_1, \vecnot{v}_2, \ldots$ be a convergent sequence in $V$, then \vspace{-2mm}
\begin{equation*}
\left\| T \vecnot{v}_i - T \vecnot{v}_j \right\|
= \left\| T \left( \vecnot{v}_i - \vecnot{v}_j \right) \right\|
\leq  M \left\| \vecnot{v}_i - \vecnot{v}_j \right\| . \vspace{-1mm}
\end{equation*}
This implies $T \vecnot{v}_1, T \vecnot{v}_2, \ldots$ is convergent in $W$ and, thus, $T$ is continuous. \vspace{1mm}

Conversely, assume $T$ is continuous and notice that $T \vecnot{0} = \vecnot{0}$.
Then, for any $\epsilon > 0$, there is a $\delta>0$ such that $\| T \vecnot{v} \| < \epsilon$ for all $\| \vecnot{v} \| < \delta$.
It follows that \vspace{-2mm}
\begin{equation*}
\left\| T \vecnot{u} \right\|
= \left\| T \frac{ \delta \vecnot{u} }{2 \left\| \vecnot{u} \right\| } \right\|
\frac{ 2 \left\| \vecnot{u} \right\| }{ \delta }
< \frac{2 \epsilon}{ \delta } \left\| \vecnot{u} \right\|. \vspace{-3mm}
\end{equation*}
Thus, $M = \frac{2 \epsilon}{\delta}$ serves as an upper bound on $\|T\|$.
\vspace{-1mm}
\end{proof}

\end{frame}


\begin{frame}<1-2> \frametitle{6.3.3: Matrix Norms}

A norm on a vector space of matrices is called a \textcolor{blue}{matrix norm}.

\begin{definition}
For $A\in F^{m\times n}$, the \textcolor{blue}{matrix norm} induced by the $\ell^p$ vector norm $\|\cdot\|_p$,  is:
\[ \left\| A \right\|_p \triangleq \max_{\left\| \vecnot{v} \right\|_p = 1} \left\| A \vecnot{v} \right\|_p. \]
\end{definition}

\uncover<2->{%
In special cases, there are exact formulae:
\begin{align*}
\left\| A \right\|_{\infty} &=
%\max_{\left\| \vecnot{v} \right\|_{\infty} = 1} \left\| A \vecnot{v} \right\|_{\infty}=
\max_{i} \sum_{j} |a_{ij}| \\
\left\| A \right\|_{1} &= 
%\max_{\left\| \vecnot{v} \right\|_{1} = 1} \left\| A \vecnot{v} \right\|_{1} =
\max_{j} \sum_{i} |a_{ij}| \\
\left\| A \right\|_{2} &= 
%\max_{\left\| \vecnot{v} \right\|_{2} = 1} \left\| A \vecnot{v} \right\|_{2} =
\sqrt{\rho (A^H A)} \, ,
\end{align*}
where the $\rho(B)$ is the maximum absolute value of all eigenvalues.

\vspace{1.5mm}

\textcolor{blue}{Examples in live session.}
}

\end{frame}

\newcommand*{\vertbar}{\rule[-1ex]{0.5pt}{2.5ex}}

\begin{frame}<1-3> \frametitle{Sneak Peek: Eigenvalue Decomposition}

\begin{definition}<1->
Let $W$ be a vector space over $F$ and let $T\colon W \to W$ be a linear operator.
A scalar $\lambda \in F$ is an \textcolor{blue}{eigenvalue} of $T$ if there exists a non-zero vector $\vecnot{v} \in W$ with $T \vecnot{v} = \lambda \vecnot{v}$.
Any vector $\vecnot{v}$ such that $T \vecnot{v} = \lambda \vecnot{v}$ is called an \textcolor{blue}{eigenvector} of $T$ associated with the eigenvalue value $\lambda$.
\end{definition}

\begin{definition}<2->
A matrix $A \in F^{n \times n}$ is \textcolor{blue}{diagonalizable} if there is an invertible matrix $V \in F^{n \times n}$ (whose columns are eigenvectors) such that $V^{-1} A V = \Lambda$ is diagonal.
\end{definition}

\uncover<3->{%
\[ A \underbrace{\begin{bmatrix} \vertbar & \vertbar & & \vertbar \\ \vecnot{v}_1 & \vecnot{v}_2 & \cdots & \vecnot{v}_n \\ \vertbar & \vertbar & & \vertbar \end{bmatrix}}_{V} = \begin{bmatrix} \vertbar & \vertbar & & \vertbar \\ \vecnot{v}_1 & \vecnot{v}_2 & \cdots & \vecnot{v}_n \\ \vertbar & \vertbar & & \vertbar \end{bmatrix} \underbrace{\begin{bmatrix} \lambda_1 & & & \\ & \lambda_2 & & \\ & & \ddots & \\ & & & \lambda_n \end{bmatrix}}_{\Lambda} \]

\textcolor{blue}{Example of $A^m$ in live session.}
}


\end{frame}



\begin{frame}<1-7> \frametitle{6.3.2: Neumann Expansion}

\vspace{-1mm}

\begin{theorem}<1->
Let $\| \cdot \|$ be a submultiplicative operator norm and $T \colon V \rightarrow V$ be a linear operator with $\left\| T \right\| < 1$.
Then, $(I-T)^{-1}$ exists and \vspace{-2mm}
\[ (I-T)^{-1} = \sum_{i=0}^{\infty} T^i. \]
\end{theorem}
\begin{proof}<2->
\begin{itemize}
\item<2-> Observe that $\sum_{i=0}^{\infty} \| T^i \| \leq \sum_{i=0}^{\infty} \| T \|^i = \frac{1}{1-\|T\|}$ because $\|T\|<1$ \vspace{1mm}

\item<3-> Recall an infinite sum $\,\sum_{i=0}^{\infty} T^i$ converges if $\sum_{i=0}^{\infty} \| T^i \|$ converges \vspace{1mm}

\item<4-> Next, observe that $(I - T) \left( I + T + T^2 + \cdots + T^{n-1} \right) = I - T^n $ \vspace{1mm}

\item<5-> $T^n \to 0$ because $ \left\| T^n \right\| \leq \left\| T \right\|^n \rightarrow 0$ since $\left\| T \right\| < 1$ \vspace{1mm}

\item<6-> Thus, $\sum_{i=0}^{\infty} T^i$ is a right inverse for $I-T$ \vspace{1mm}

\item<7-> Same argument works on the left, so we're done. \hfill \qedhere

\end{itemize}
\end{proof}

\end{frame}

\begin{frame} \frametitle{Next Steps}

\begin{itemize}
\setlength\itemsep{5mm}
\item To continue studying after this video -- \vspace{2mm}

\begin{itemize}
 \setlength\itemsep{3mm}
 \item Try the required reading: Course Notes EF 3.5, 6.3
 \item Or the recommended reading: MMA 4.1-4.2
 \item Also, look at the problems in Assignment 5
\end{itemize}
\end{itemize}

\note{
	\vspace{8mm}
	\begin{enumerate}
		\setlength\itemsep{3mm}
		\color{red}
		\item Here are some options to continue learning this material. (read) \\ [2mm]  That's it for today.  So, I'll see you next time.
	\end{enumerate}
}

\end{frame}


\end{document}


