\documentclass[10pt,english,aspectratio=169]{beamer}
% Use notes or hide notes or show only notes or handout

\usetheme{default}

\usepackage{xstring}
\usepackage{pgfpages}
%\makeatletter
%\IfSubStr{\@classoptionslist}{handout}
%  {\pgfpagesuselayout{2 on 1}[letterpaper,border shrink=5mm]}
%  {}
%\makeatother

\usepackage{amsmath,amssymb,amsthm}
\usepackage{stmaryrd}
\usepackage{enumerate}
\usepackage{stfloats}
\usepackage{bbm}
\usepackage{pdfpages}
\usepackage{framed}

\usepackage[most]{tcolorbox}
\tcbset{highlight math style={enhanced,
  colframe=white,colback=yellow!15,arc=8pt,boxrule=1pt,
  }}
  
\usepackage{tikz,pgf,pgfplots}
\usepackage{algorithm,algorithmic}
\usepgflibrary{shapes}
\usetikzlibrary{%
  arrows,%
  arrows.meta,
  backgrounds,
  shapes.misc,% wg. rounded rectangle
  shapes.arrows,%
  shapes,%
  calc,%
  chains,%
  matrix,%
  positioning,% wg. " of "
  scopes,%
  decorations.pathmorphing,% /pgf/decoration/random steps | erste Graphik
  shadows,%
  backgrounds,%
  fit,%
  petri,%
  quotes
}

\tikzset{background rectangle/.style={
    fill=white,
  },
  use background/.style={    
    show background rectangle
  }
}

\setbeamersize{text margin left=10mm,text margin right=35mm}

\pgfplotsset{compat=1.12}

%\usetheme{Frankfurt}
%\usecolortheme{ldpc}
\useinnertheme{rounded}
\usecolortheme{whale}
\usecolortheme{orchid}

\newcommand{\ul}[1]{\underline{#1}}
\renewcommand{\Pr}{\mathbb{P}}

%% Setup slides and notes
\makeatletter
\IfSubStr{\@classoptionslist}{notes} { \IfSubStr{\@classoptionslist}{hide} {}{\IfSubStr{\@classoptionslist}{only} {}{\setbeameroption{show notes on second screen=right}}} }{}
\makeatother

\newcommand{\getpdfpages}[2]{\begingroup
  \setbeamercolor{background canvas}{bg=}
  \addtocounter{framenumber}{1}
  \includepdf[pages={#1},%
  pagecommand={%
    \expandafter\def\expandafter\insertshorttitle\expandafter{%
      \insertshorttitle\hfill\insertframenumber\,/\,\inserttotalframenumber}}%
  ]{#2}
  \endgroup}

\newcommand{\backupbegin}{
   \newcounter{finalframe}
   \setcounter{finalframe}{\value{framenumber}}
}
\newcommand{\backupend}{
   \setcounter{framenumber}{\value{finalframe}}
}

 \setbeamercolor{bibliography entry author}{fg=black}
 \setbeamercolor{bibliography entry title}{fg=black}
 \setbeamercolor{bibliography entry location}{fg=black}
 \setbeamercolor{bibliography entry note}{fg=black}
 
 \setbeamerfont{bibliography item}{size=\footnotesize}
 \setbeamerfont{bibliography entry author}{size=\footnotesize}
 \setbeamerfont{bibliography entry title}{size=\footnotesize}
 \setbeamerfont{bibliography entry location}{size=\footnotesize}
 \setbeamerfont{bibliography entry note}{size=\footnotesize}
 \setbeamertemplate{bibliography item}{\insertbiblabel}
 
\newlength\tikzwidth
\newlength\tikzheight


\newcommand{\mc}[1]{\mathcal{#1}}
\newcommand{\mbb}[1]{\mathbb{#1}}
%\newcommand{\expt}{\mbb{E}}
%\newcommand{\dd}{\mathrm{d}}
\newcommand{\Interior}[1]{\ensuremath{{#1}^{\circ}}}
\newcommand{\Closure}[1]{\ensuremath{\overline{#1}}}
\newcommand{\Complement}[1]{\ensuremath{{#1}^{c}}}

\newcommand{\Expect}{\ensuremath{\mathrm{E}}}
\newcommand{\vecnot}{\underline}
\newcommand{\RealNumbers}{\ensuremath{\mathbb{R}}}
\newcommand{\RationalNumbers}{\mathbb{Q}}
\newcommand{\ComplexNumbers}{\mathbb{C}}
\newcommand{\Real}{\mathrm{Re}}
\newcommand{\Span}{\mathrm{span}}
\newcommand{\Rank}{\mathrm{rank}}
\newcommand{\Nullity}{\mathrm{nullity}}
\newcommand{\Trace}{\mathrm{tr}}
\newcommand{\Diag}{\mathrm{diag}}
\newcommand{\dd}{\mathrm{d}}
\DeclareMathOperator*{\esssup}{ess\,sup}

% Use < , > inner product
\newcommand{\inner}[2]{{\left\langle #1 \mskip2mu , #2 \right\rangle}}
\newcommand{\tinner}[2]{{\langle #1 \mskip1mu , #2 \rangle}}

% Use < | > inner product
%\newcommand{\inner}[2]{{\left\langle #1 \mskip2mu \middle| \mskip2mu #2 \right\rangle}}
%\newcommand{\tinner}[2]{{\langle #1 \mskip1mu | \mskip1mu  #2 \rangle}}




\def\checkmark{\tikz\fill[scale=0.4](0,.35) -- (.25,0) -- (1,.7) -- (.25,.15) -- cycle;}
\def\greencheck{{\color{green}\checkmark}}
\def\scalecheck{\resizebox{\widthof{\checkmark}*\ratio{\widthof{x}}{\widthof{\normalsize x}}}{!}{\checkmark}}
\def\xmark{\tikz [x=1.4ex,y=1.4ex,line width=.2ex, red] \draw (0,0) -- (1,1) (0,1) -- (1,0);}
\def\redx{{\color{red}\xmark}}

\renewcommand{\footnotesep}{-2pt}


\begin{document}



\title{ECE 586: Vector Space Methods \\ Lecture 7 Flip Video: Compactness}
\author{Henry D. Pfister \\ Duke University}
\date{}
%\date{August 20th, 2020}
%\maketitle

\setbeamertemplate{navigation symbols}{}

\begin{frame}[plain]
	\titlepage
	
	\note{
		\vspace{8mm}
		\begin{enumerate}
			\setlength\itemsep{3mm}
			\color{red}
			\item Welcome to the seventh video lecture for ECE 586, Vector Space Methods. \\[2mm]
			Today, we'll discuss the idea of compactness and its applications.
		\end{enumerate}
	}
\end{frame}

\addtocounter{framenumber}{-1}
\setbeamertemplate{navigation symbols}{\textcolor{blue}{\footnotesize \insertframenumber ~/ \inserttotalframenumber}}


\begin{frame}<1-4>
\frametitle{2.1.4: Compactness}

\begin{definition}<1->
A metric space $(X,d)$ is \textcolor{blue}{totally bounded} if, for any $\epsilon > 0$, there exists a finite set of radius-$\epsilon$ balls that cover $X$ (i.e., $\cup_{x\in S} B_d (x,\epsilon) = X$).
\end{definition}

\begin{definition}<2->
A metric space is \textcolor{blue}{compact} if it is complete~and~totally~bounded.
\end{definition}

\begin{itemize}
\setlength\itemsep{3mm}
\item<3-> Examples \vspace{1mm}
\begin{itemize} 
  \setlength\itemsep{1.5mm}
  \item The closed real interval $[0,1]\subset \RealNumbers$ is compact
  \item A subset of Euclidean $\RealNumbers^n$ is compact iff it is closed and bounded
  \item But, the standard metric space of real numbers is not compact because it is not totally bounded.
\end{itemize}

\end{itemize}

\begin{theorem}<4>
A closed subset $A$ of a compact space $X$ is itself a compact space.
\end{theorem}

\note{
	\vspace{2mm}
	\begin{enumerate}[<alert@+>]
		\footnotesize
		\setlength\itemsep{3mm}
		\item Compactness is a very useful property in analysis that identifies when an arbitrary metric space behaves (in some sense) like a closed and bounded subset of real numbers. \\[3mm]
		Read
		\item Read.
		\item Read.  This is because $N$ balls of radius $\epsilon$ can cover a width of at most $N\epsilon$.
		\item Read.
	\end{enumerate}
}

\end{frame}

\begin{frame}<1-3>
\frametitle{2.1.4: Compactness and Sequences}

\begin{definition}<1->
Let $x_1,x_2,\ldots \in X$ be a sequence and $n_1,n_2,\ldots\in \mathbb{N}$ be
a strictly increasing sequence.
Then, $x_{n_1},x_{n_2},\ldots$ is called \textcolor{blue}{subsequence}.
\end{definition}

\begin{theorem}<2->
A sequence in a compact metric space has a subsequence that converges.
\end{theorem}

\begin{example}<3->
For the compact metric space $X=[-2,2]\subset \mathbb{R}$ with absolute distance, let $x_n = (-1)^n + \frac{1}{n}$.
Then, subsequence $x_2,x_4,x_6,\ldots$ converges to 1.
\end{example}

\begin{itemize}
\item<3-> Sketch proof on whiteboard in pictures
\end{itemize}

\note{
	\vspace{2mm}
	\begin{enumerate}[<alert@+>]
		\footnotesize
		\setlength\itemsep{3mm}
		\item Read.
		\item Read. This property is used somewhat regularly in analysis.
		\item Read. This is because $-1$ to an even power equals 1 and $1/n$ vanishes.  Now, I want you to think the subsequence $x_1,x_3,x_5,\ldots$.  Does this converge? \\[2mm] In the live session, we will sketch a proof of the theorem.
	\end{enumerate}
}

\end{frame}

\begin{frame}<1-4>
\frametitle{2.1.4: Properties of Real Numbers}

\begin{itemize}
\setlength\itemsep{3mm}
\item<1-> Let us include the natural boundary values for the real numbers \vspace{1mm}

\begin{itemize} 
  \setlength\itemsep{1.5mm}
  \item Extended Real Numbers: $\overline{\RealNumbers} \triangleq \RealNumbers \cup \{ \infty,-\infty\}$
  \item This is a \textcolor{blue}{compact metric space} with metric $d_{\overline{\RealNumbers}} (x,y) \triangleq |\frac{x}{1+|x|} - \frac{y}{1+|y|}|$
  \item ``$x_n \to \infty$'' equivalent to ``$\forall M>0, \, \exists N\in \mathbb{N}, \, \forall n>N, \, x_n > M$''
\end{itemize}
\end{itemize}

\begin{definition}<2->
The \textcolor{blue}{supremum} (or least upper bound) of $X\subseteq \RealNumbers$ is denoted \textcolor{blue}{$\sup X$} and equals the smallest extended real number $M \in \overline{\RealNumbers}$ such that $x\leq M$ for all $x\in X$.
%It is always well-defined and equals $-\infty$ if $X=\emptyset$.
\end{definition}

\begin{lemma}[supremum sequence]<3->
Let $X$ be a metric space and $f \colon X\rightarrow \RealNumbers$ be a function mapping $X$ to the real numbers.
Let $M = \sup f(A)$ for some non-empty $A \subseteq X$.
Then, there exists a sequence $x_1,x_2,\ldots \in A$ such that $\lim_n f(x_n) = M$.
\end{lemma}

\begin{itemize}
\item<4-> Sketch proof on whiteboard
\end{itemize}

\note{
	\vspace{2mm}
	\begin{enumerate}[<alert@+>]
		\footnotesize
		\setlength\itemsep{3mm}
		\item Read.  With this metric, the extended real numbers are the completion of the standard real numbers.
		\item Read. 
		\item Read. This is an important concept in analysis that allows one to exploit compactness.
		\item In the live session, we will sketch a proof of this.
	\end{enumerate}
}

\end{frame}


\begin{frame}<1-4>
\frametitle{2.1.4: More Properties of Real Numbers}

\begin{definition}<1->
The \textcolor{blue}{maximum} of $X\subseteq \RealNumbers$, denoted \textcolor{blue}{$\max X$}, is the largest value contained in the set.
It equals the supremum if $\sup X \in X$ and it is \textcolor{red}{undefined} otherwise.
\end{definition}

\begin{example}<2->
$X = [1,2) \subset \mathbb{R}$ has $\sup X = 2$ and $\max X$ undefined. \\[1mm]
For $f(x)=\frac{1}{2-x}$, $f(X) = [1,\infty)$ and $\sup f(X) = \infty$.
\end{example}

\begin{itemize}
\item<3-> \textcolor{blue}{infimum}: $\inf X = - \left(\sup \, -\!X\right)$, where $-\!X = \{x\in \mathbb{R}\,|-\!x\in X\}$
\item<3-> \textcolor{blue}{minimum}: $\min X = - \left(\max \, -\!X \right)$, if it exists
\item<3-> supremum and infimum always well-defined but may equal $\pm \infty$
\end{itemize}

\begin{theorem}<4->
A bounded non-decreasing sequence of real numbers converges to its supremum.
\end{theorem}

\note{
	\vspace{2mm}
	\begin{enumerate}[<alert@+>]
		\footnotesize
		\setlength\itemsep{3mm}
		\item Read.
		\item Read.
		\item The infimum and minimum are the natural minimal quantities associated with the supremum and maximum. \\[2mm] Read. 
		\item Read. This result says that a non-decreasing sequence must converge if it is bounded.
	\end{enumerate}
}

\end{frame}  


\begin{frame}<1-3>
\frametitle{2.1.5: Sequences of Functions}

Let $(X,d_X)$ and $(Y,d_Y)$ be metric spaces and
$f_n \colon X \to Y$ for $n\in \mathbb{N}$ be a sequence of functions mapping $X$ to $Y$.

\begin{definition}<1->
The sequence $f_n$ \textcolor{blue}{converges pointwise} to $f\colon X \to Y$ if, for all $x \in X$, \vspace{-1.5mm}
\[ \lim_{n\to \infty} f_n (x) = f(x) \vspace*{-0.5mm} \]

\end{definition}

\begin{definition}<2->
The sequence $f_n$ \textcolor{blue}{converges uniformly} to $f \colon X \to Y$ if \vspace{-1.5mm}
\[ \forall \epsilon > 0, \, \exists N\in \mathbb{N}, \, \forall n>N, \, \forall x\in X, \, d_Y \left( f_n(x),f(x) \right) < \epsilon. \vspace*{-0.5mm} \]
\end{definition}

\begin{theorem}<3->
If each $f_n$ is continuous and $f_n$ converges uniformly to $f \colon X \to Y$, then $f$ is continuous.
\end{theorem}

\note{
	\vspace{2mm}
	\begin{enumerate}[<alert@+>]
		\footnotesize
		\setlength\itemsep{3mm}
		\item Read.  This is the standard notion of convergence for sequences of functions.
		\item Read. This is a stronger notion of convergence that can be quite useful. 
		\item In particular, we have the following theorem... Read. \\[2mm]  This is the standard mechanism by which a sequence of continuous functions converges to a continuous function.
	\end{enumerate}
}

\end{frame}

\begin{frame}<1-3>
\frametitle{2.1.5: Two Important Results}

\begin{theorem}<1->
Let $X$ be a metric space and $f \colon X\rightarrow \RealNumbers$ be a continuous function from $X$ to $\mathbb{R}$.
If $A$ is a compact subset of $X$, then there exists $x\in A$ such that $f(x)=\sup f(A)$ (i.e., $f$ achieves a maximum on $A$).
\end{theorem}

\vspace{3mm}

\begin{theorem}<2->
Let $(X,d)$ be a compact metric space and $C(X)$ be the set of continuous functions mapping $X$ to $\mathbb{R}$.
This set of functions, with the metric \vspace{-1.5mm} \[d_\infty (f,g) \triangleq \sup_{x\in X} |f(x)-g(x)| = \max_{x\in X} |f(x)-g(x)|, \vspace*{-1.5mm}\] defines a complete metric space.
\end{theorem}

\uncover<3->{Note: For $f,g\in C(X)$, $d_\infty$ metrizes uniform convergence because \vspace{-1.5mm} \[ \text{``}\max_{x\in X} |f(x)-g(x)| < \epsilon \text{''} \Leftrightarrow \text{``} \forall x\in X, |f(x) - g(x)| < \epsilon \text{''}. \]}

\note{
	\vspace{2mm}
	\begin{enumerate}[<alert@+>]
		\footnotesize
		\setlength\itemsep{3mm}
		\item Read. This result is often summarized as ``A continuous function on a compact set achieves a maximum value''.
		\item The previous result allows one to define a complete metric space of functions... Read.
		\item We say that a metric ``metrizes'' a type of convergence if convergence under that metric is equivalent to that type of convergence.
	\end{enumerate}
}

\end{frame}


\begin{frame} \frametitle{Next Steps}

\begin{itemize}
\setlength\itemsep{5mm}
\item To continue studying after this video -- \vspace{2mm}

\begin{itemize}
 \setlength\itemsep{3mm}
 \item Try the suggested reading: Course Notes EF 2.1.4 - 2.1.5
 \item Or the optional reading: MMA 2.1
 \item Also, look at the problems in Assignment 3
\end{itemize}
\end{itemize}

\note{
	\vspace{8mm}
	\begin{enumerate}
		\setlength\itemsep{3mm}
		\color{red}
		\item Here are some options to continue learning this material. (read) \\ [2mm]  That's it for today.  So, I'll see you next time.
	\end{enumerate}
}

\end{frame}


\end{document}

  
\backupbegin

%\begin{frame}
%\frametitle{Backup Slides}
%\begin{itemize}
%\item Slide numbers not included in denominator!
%\end{itemize}
%\end{frame}

%\begin{frame}[allowframebreaks]
%\frametitle{References}
%\bibliographystyle{alpha}
%\footnotesize
%\bibliography{IEEEabrv,WCLabrv,WCLbib,WCLnewbib}
%\end{frame}

\backupend

\end{document}
