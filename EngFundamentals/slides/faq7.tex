\documentclass[10pt,english]{article}

\usepackage{amsfonts,url}
\usepackage{amsmath}
\usepackage{amssymb}
\usepackage{hyperref}

% Paper setup
\evensidemargin=0in
\oddsidemargin=0in
\textwidth=6.25in
\topmargin=-0.5in
\headheight=0.0in
\headsep=0.5in
\textheight=9.0in
\footskip=0.5in

\newcommand{\Interior}[1]{\ensuremath{{#1}^{\circ}}}
\newcommand{\Closure}[1]{\ensuremath{\overline{#1}}}
\newcommand{\Complement}[1]{\ensuremath{{#1}^{c}}}

\newcommand{\Expect}{\ensuremath{\mathrm{E}}}
\newcommand{\vecnot}{\underline}
\newcommand{\RealNumbers}{\ensuremath{\mathbb{R}}}
\newcommand{\RationalNumbers}{\mathbb{Q}}
\newcommand{\ComplexNumbers}{\mathbb{C}}
\newcommand{\Real}{\mathrm{Re}}
\newcommand{\Span}{\mathrm{span}}
\newcommand{\Rank}{\mathrm{rank}}
\newcommand{\Nullity}{\mathrm{nullity}}
\newcommand{\Trace}{\mathrm{tr}}
\newcommand{\Diag}{\mathrm{diag}}
\newcommand{\dd}{\mathrm{d}}
\DeclareMathOperator*{\esssup}{ess\,sup}

% Use < , > inner product
\newcommand{\inner}[2]{{\left\langle #1 \mskip2mu , #2 \right\rangle}}
\newcommand{\tinner}[2]{{\langle #1 \mskip1mu , #2 \rangle}}

% Use < | > inner product
%\newcommand{\inner}[2]{{\left\langle #1 \mskip2mu \middle| \mskip2mu #2 \right\rangle}}
%\newcommand{\tinner}[2]{{\langle #1 \mskip1mu | \mskip1mu  #2 \rangle}}




\begin{document}

\title{ECE 586: Vector Space Methods \\ Lecture 7: Frequently Asked Questions}
\author{Henry D. Pfister \\ Duke University}
\date{September 15th, 2020}

\maketitle

Here we give a list of questions, and their answers, that were submitted by students after watching the flip video.

\section{Complete Metric Spaces}

\paragraph{I'm confused about the proof of Example 2.1.12 in the notes. Could you elaborate it more during the lecture?}

I have added more information about this to the Lecture Notebook for September 3rd which is posted on Sakai.


\paragraph{Could you expand more on isometry?}

An isometry is simply a map between two spaces that preserves distances.
Namely, for two metric spaces $(X,d_X)$ and $(Y,d_Y)$, an isometry is a map $f: X \rightarrow Y$ that satisfies $\forall a,b \in X, d_X(a,b) = d_Y(f(a),f(b))$.

\paragraph{For page 4, what is $\boldsymbol{f(y)}$ in the example? and I am also confused about what does this example illustrate? And for page 6, where does these examples apply the contraction mapping theorem?  I haven't understood the idea of this theorem. Thank you!}

The example is meant to illustrate the meaning of a contraction. Intuitively, a contraction is a mapping that "shrinks" distances. More specifically, consider a mapping $f: X \rightarrow X$ in a complete metric space $(X,d_X)$ such that, for all $a,b \in X$, we have $d_X(f(a),f(b)) \leq \gamma d(a,b)$ where $\gamma \in [0,1)$.
Suppose now you construct a sequence $x_{n+1} = f(x_n)$ where $f$ is a contraction (i.e. if the first point in the sequence is $x_1$, then the $n$-th point is $f(...f(f(x_1)))$ where $f$ is applied $n-1$ times). The contraction mapping theorem states that such a sequence converges to a fixed point $x^*$ (defined as a point $x^*$ such that $x^* = f(x^*)$, i.e. a point ``unchanged" by $f$). Moreover, the contraction mapping theorem states that the contraction has a unique fixed point $x^*$.



\paragraph{In the definition of a dense subset, you mention that a dense subset A is equivalent to the closure A (bar) being equal to X. However, I recall that X is an open set while a closure is a closed set. How can this equivalence hold between open and closed sets? Or did I perceive this statement incorrectly?}

Let me start by relaying an unfortunate truth -- for some reason, in topology/analysis, the definitions of ``open" and ``closed" are not opposites. A set $X$ being open just means that for every element $x$ in the set, you can find an $\epsilon$-ball around $x$ that is entirely inside $X$. A closed set is a set that contains all of its limit points.
Clearly, these definitions are not opposites of each other (case in point -- you can find sets that are both open \textit{and} closed, e.g. $\emptyset$ or $\mathbb{R}$).
However, the concepts of open and closed are related to each other. Specifically, a closed set is a set whose \textit{complement} is an open set.

The other thing is, in the metric space $(X,d)$, $X$ always both open and closed.

\paragraph{I don't understand how Cauchy sequence of elements in X can help complete a metric space. Could you give an example?}

Suppose we take the rationals for the example. We know that the rationals are not complete, since we can find a Cauchy sequence of rationals that does not converge to a rational (e.g. $x_n = (1+\frac{1}{n})^n$ is a rational sequence that is Cauchy, but it converges to $e$, which is not rational -- hence $\mathbb{Q}$ is not complete).
To ``complete" $\mathbb{Q}$, we would need to find all the Cauchy sequences in $\mathbb{Q}$, and add their limits to $\mathbb{Q}$ which would complete it.
In particular, the completion would have a new element $x$ for each irrational number and $x$ would be associated with all Cauchy sequences of rationals such that converge to $x$.

\paragraph{Is the closure of the set of all rational numbers equal to the set of real numbers?}
Yes, every real number is a limit point of rational numbers.
So, the closure of $\mathbb{Q}$ (i.e. $\mathbb{Q}$ + its limit points) is equivalent to $\mathbb{R}$.

\paragraph{You read the $L^2$ norm as ``energy". Is it supposed to resemble the formula signal energy?} 
Yes, they are the same.

\paragraph{What is meant by "converge to a continuous function"? Since I have only seen the definition of converging to a point.}

Good question. You can think of functions as \textit{points} in a metric space (i.e. you can define a distance between each of these points, e.g. the L2 distance). The concept of convergence requires you to define a metric. Convergence just means getting closer and closer to some point w.r.t a metric $d$. Hence, you can think that a sequence of functions can indeed converge to a function w.r.t., say, the L2 metric.



\paragraph{Can you clarify the difference between range, image and codomain?}
You can think of the domain as the set of possible inputs to $f$, and the codomain as the set in which $f$ generates outputs -- it is important to note that the domain and codomain are sets that are largely a matter of \textit{choice} (e.g. I could very well choose a function, say $f(x) = x^2$ and restrict its domain to be $\mathbb{Q}$ and its codomain to be e.g. $\mathbb{R}$).

The \textit{image} of a set $A$ by a function $f$ is the set (denoted $f(A)$) of elements obtained by mapping all elements of $A$ through $f$. More precisely, $f(A) \triangleq \{f(x) | x\in A\}$.
The \textit{range} of a function is simply the image of its domain $f(X)$ or equivalently the subset of the codomain achieved by $f$.

\paragraph{I cannot understand the example on page 3.}

The example on page 3 is a counterexample to show that $C[-1,1]$ is not a complete metric space.
It does so by providing an example of a Cauchy sequence (of functions in $C[-1,1]$) that does not converge to a function in $C[-1,1]$.

More specifically, the functions converge pointwise to the function
\[ f(t) = \begin{cases} 0,~~ \text{for}~t<0 \\ 1, ~~ \text{for}~t> 0 ,\end{cases} \] which is not continuous and therefore not in $C[-1,1]$.
Since we have found an example of a Cauchy sequence in $C[-1,1]$ that does not converge (to a point in $C[-1,1]$), this means that $C[-1,1]$ is not complete (by definition of completeness).\\

\paragraph{What does it mean to be "$y$-Jacobian" invertible? Also on your last slide, what is the pink spiraling line supposed to represent?}
For a function $g \colon \mathbb{R}^n \to \mathbb{R}^n$, the Jacobian is the matrix of partial derivatives of output values with respect to input values.
Consider a function mapping $\mathbb{R}^m \times \mathbb{R}^n$ to $\mathbb{R}^n$ defined by $z = f (x,y)$ with $z_i = f_i (x,y)$.
Then, the $y$-Jacobian is an $n\times n$ matrix $J (x,y)$ with entries $J_{ij} (x,y) = \frac{\partial f_i (x,y)}{\partial y_j}$.
If $J(x,y)$ is invertible, then $f$ is $y$-Jacobian invertible at $(x,y)$.

%with reJacobian-invertible means that the function $f$ has a Jacobian (a matrix, and the high-dimensional extension of a first derivative for multi-input multi-output functions) which is an invertible matrix.

The pink spiraling line on the last slide is meant to show the path by which the sequence defined by $x_{n+1} = f(x_n)$ converges to the fixed point.


\paragraph{What are some signal processing and ML applications of metric space theory?}

See part IV of the optional text Mathematical Methods and Algorithms for Signal Processing.
The URL is in the Sakai resources.

\paragraph{There are so many new terms in this section such as ZF set theory. I tend to forget their definitions and need to google them in order to follow along. Do you have any suggestions on how to improve?}

Yes, it's true that learning all these terms is challenging.
For the exam, the key is to learn the terms that you see in homework or on the practice exam.
In the lectures and notes, other terms are also used to explain things but you do not need to memorize them.
My suggestion is to be patient.
Googling and rereading the definitions will eventually give your brain what it needs to recall the essence (if not the exact definition).




\paragraph{(Slide 2) Why do we define an isometry?  Why is the completion is unique up to isometry? Does the remark suggest to use the limits of Cauchy sequence to complete the metric space? }

The term ``dense'' is used regularly in mathematical discussions.
We define isometry primarily to define dense precisely.
After you know the term, you will also realize that the term isometry is used somewhat regularly as well.


\paragraph{On the last slide, I am not sure how we obtain unique fixed point $\boldsymbol{x^* \approx 0.739}$.}

One can numerically solve the fixed-point equation $\cos(x)-x=0$ for an $x\in[0,1]$.

\paragraph{In the explanation of the contraction mapping theorem, is the colored translucent circle meant to represent the point $\boldsymbol{x^*}$ to which other points converge?}

No, the colored circles are used to represent a blob in the set that is mapped through $f$.
For example, the initial set $A$ and the blue blob were chosen by me.
But, the purple set if defined by $f$ and $A$.
Also, the purple blob is supposed to represent the image of the blue blob.
Finally, the orange set is defined by $f$ and $f(A)$.
The orange blob is supposed to represent the image of the purple blob.




\end{document}
