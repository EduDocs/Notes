\documentclass[10pt,english,aspectratio=169]{beamer}
% Use notes or hide notes or show only notes or handout

\usetheme{default}

\usepackage{xstring}
\usepackage{pgfpages}
%\makeatletter
%\IfSubStr{\@classoptionslist}{handout}
%  {\pgfpagesuselayout{2 on 1}[letterpaper,border shrink=5mm]}
%  {}
%\makeatother

\usepackage{amsmath,amssymb,amsthm}
\usepackage{stmaryrd}
\usepackage{enumerate}
\usepackage{stfloats}
\usepackage{bbm}
\usepackage{pdfpages}
\usepackage{framed}

\usepackage[most]{tcolorbox}
\tcbset{highlight math style={enhanced,
  colframe=white,colback=yellow!15,arc=8pt,boxrule=1pt,
  }}
  
\usepackage{tikz,pgf,pgfplots}
\usepackage{algorithm,algorithmic}
\usepgflibrary{shapes}
\usetikzlibrary{%
  arrows,%
  arrows.meta,
  backgrounds,
  shapes.misc,% wg. rounded rectangle
  shapes.arrows,%
  shapes,%
  calc,%
  chains,%
  matrix,%
  positioning,% wg. " of "
  scopes,%
  decorations.pathmorphing,% /pgf/decoration/random steps | erste Graphik
  shadows,%
  backgrounds,%
  fit,%
  petri,%
  quotes
}

\tikzset{background rectangle/.style={
    fill=white,
  },
  use background/.style={    
    show background rectangle
  }
}

\setbeamersize{text margin left=10mm,text margin right=35mm}

\pgfplotsset{compat=1.12}

%\usetheme{Frankfurt}
%\usecolortheme{ldpc}
\useinnertheme{rounded}
\usecolortheme{whale}
\usecolortheme{orchid}

\newcommand{\ul}[1]{\underline{#1}}
\renewcommand{\Pr}{\mathbb{P}}

%% Setup slides and notes
\makeatletter
\IfSubStr{\@classoptionslist}{notes} { \IfSubStr{\@classoptionslist}{hide} {}{\IfSubStr{\@classoptionslist}{only} {}{\setbeameroption{show notes on second screen=right}}} }{}
\makeatother
%\setbeamertemplate{note page}{\pagecolor{yellow!5}\vfill\insertnote\vfill}

\newcommand{\getpdfpages}[2]{\begingroup
  \setbeamercolor{background canvas}{bg=}
  \addtocounter{framenumber}{1}
  \includepdf[pages={#1},%
  pagecommand={%
    \expandafter\def\expandafter\insertshorttitle\expandafter{%
      \insertshorttitle\hfill\insertframenumber\,/\,\inserttotalframenumber}}%
  ]{#2}
  \endgroup}

\newcommand{\backupbegin}{
   \newcounter{finalframe}
   \setcounter{finalframe}{\value{framenumber}}
}
\newcommand{\backupend}{
   \setcounter{framenumber}{\value{finalframe}}
}

 \setbeamercolor{bibliography entry author}{fg=black}
 \setbeamercolor{bibliography entry title}{fg=black}
 \setbeamercolor{bibliography entry location}{fg=black}
 \setbeamercolor{bibliography entry note}{fg=black}
 
 \setbeamerfont{bibliography item}{size=\footnotesize}
 \setbeamerfont{bibliography entry author}{size=\footnotesize}
 \setbeamerfont{bibliography entry title}{size=\footnotesize}
 \setbeamerfont{bibliography entry location}{size=\footnotesize}
 \setbeamerfont{bibliography entry note}{size=\footnotesize}
 \setbeamertemplate{bibliography item}{\insertbiblabel}
 
\newlength\tikzwidth
\newlength\tikzheight


\newcommand{\mc}[1]{\mathcal{#1}}
\newcommand{\mbb}[1]{\mathbb{#1}}
%\newcommand{\expt}{\mbb{E}}
%\newcommand{\dd}{\mathrm{d}}
%\usepackage{fixltx2e}

\newcommand{\Interior}[1]{\ensuremath{{#1}^{\circ}}}
\newcommand{\Closure}[1]{\ensuremath{\overline{#1}}}
\newcommand{\Complement}[1]{\ensuremath{{#1}^{c}}}

\newcommand{\Expect}{\ensuremath{\mathrm{E}}}
\newcommand{\vecnot}{\underline}
\newcommand{\RealNumbers}{\mathbb{R}}
\newcommand{\RationalNumbers}{\mathbb{Q}}
\newcommand{\ComplexNumbers}{\mathbb{C}}
\newcommand{\Real}{\mathrm{Re}}
\newcommand{\Span}{\mathrm{span}}
\newcommand{\Rank}{\mathrm{rank}}
\newcommand{\Nullity}{\mathrm{nullity}}
\newcommand{\Trace}{\mathrm{tr}}
\newcommand{\Diag}{\mathrm{diag}}
\DeclareMathOperator*{\esssup}{ess\,sup}
\newcommand{\dd}{\mathrm{d}}



\def\checkmark{\tikz\fill[scale=0.4](0,.35) -- (.25,0) -- (1,.7) -- (.25,.15) -- cycle;}
\def\greencheck{{\color{green}\checkmark}}
\def\scalecheck{\resizebox{\widthof{\checkmark}*\ratio{\widthof{x}}{\widthof{\normalsize x}}}{!}{\checkmark}}
\def\xmark{\tikz [x=1.4ex,y=1.4ex,line width=.2ex, red] \draw (0,0) -- (1,1) (0,1) -- (1,0);}
\def\redx{{\color{red}\xmark}}

\renewcommand{\footnotesep}{-2pt}


\begin{document}

\title{ECE 586: Vector Space Methods \\ Lecture 13 Flip Video: Normed Vector Spaces}
\author{Henry D. Pfister \\ Duke University}
\date{}
%\date{August 20th, 2020}
%\maketitle

\setbeamertemplate{navigation symbols}{}

\begin{frame}[plain]
	\titlepage
	
	\note{
		\vspace{8mm}
		\begin{enumerate}
			\setlength\itemsep{3mm}
			\color{red}
			\item Welcome to the 11th video lecture for ECE 586, Vector Space Methods. \\[2mm]
			Today, we'll finish our discussion of subspaces and bases and then move on to linear transforms.
		\end{enumerate}
	}
\end{frame}

\addtocounter{framenumber}{-1}
\setbeamertemplate{navigation symbols}{\textcolor{blue}{\footnotesize \insertframenumber ~/ \inserttotalframenumber}}




\begin{frame}{3.5: Normed Vector Spaces}

Let $V$ be a vector space over the real numbers or the complex numbers.

\begin{definition}
A \textcolor{blue}{norm} on vector space $V$ is a real-valued function $\left\| \cdot \right\| \colon V \rightarrow \RealNumbers$ that satisfies the following properties.
\begin{enumerate}
\item $\left\| \vecnot{v} \right\| \geq 0 \quad \forall \vecnot{v} \in V$;
equality holds if and only if $\vecnot{v} = \vecnot{0}$
\item $\left\| s \vecnot{v} \right\| = |s| \left\| \vecnot{v} \right\| \quad \forall \vecnot{v} \in V, s \in F$
\item $\left\| \vecnot{v} + \vecnot{w} \right\| \leq
\left\| \vecnot{v} \right\| + \left\| \vecnot{w} \right\| \quad \forall \vecnot{v}, \vecnot{w} \in V$.
\end{enumerate}
\end{definition}

\vspace{3mm}

The concept of a norm is closely related to that of a metric.
For instance, a metric can be defined from any norm.\\[2mm]
Let $\left\| \vecnot{v} \right\|$ be a norm on vector space $V$, then the \textcolor{blue}{induced metric} is
\begin{equation*}
d \left( \vecnot{v}, \vecnot{w} \right)
= \left\| \vecnot{v} - \vecnot{w} \right\|.
\end{equation*}

\note{
	\vspace{2mm}
	\begin{enumerate}[<alert@+>]
		\footnotesize
		\setlength\itemsep{3mm}
		\item Sums and Direct Sums allow one to combine subspaces.  Read.
		\item Read.  This is the smallest intersection possible because all subspaces contain the zero vector.
		\item Read.
		\item Read.
	\end{enumerate}
}

\end{frame}

\begin{frame}{3.5: Examples of Normed Vector Spaces}

\begin{example}[Standard Norms for Real/Complex Vector Spaces]
The following functions are examples of norms for $\mathbb{R}^n$ and $\mathbb{C}^n$:
\begin{enumerate}
\item the $l^1$ norm: $\left\| \vecnot{v} \right\|_1 = \sum_{i=1}^n |v_i|$
\item the $l^p$ norm: $\left\| \vecnot{v} \right\|_p = \left( \sum_{i=1}^n |v_i|^p \right)^{\frac{1}{p}}, \quad p \in (1,\infty)$
\item the $l^{\infty}$ norm: $\left\| \vecnot{v} \right\|_{\infty} = \max_{1,\ldots, n} \{ |v_i| \}$
\end{enumerate}
\end{example}

\begin{example}[Standard Norms for Real/Complex Function Spaces]
Similarly, for the vector space of functions from $[a, b]$ to $\RealNumbers$ (or $\ComplexNumbers$):
\begin{enumerate}
\item the $L^1$ norm: $\left\| f(t) \right\|_1 = \int_a^b |f(t)| dt$
\item the $L^p$ norm: $\left\| f(t) \right\|_p = \left( \int_a^b |f(t)|^p dt \right)^{\frac{1}{p}}, \quad p \in (1,\infty)$
\item the $L^{\infty}$ norm: $\left\| f(t) \right\|_{\infty} = \esssup_{[a,b]} \{ | f(t) | \}$
\end{enumerate}
\end{example}

For infinite dimensional spaces, only vectors with finite norm are included.

\end{frame}

\begin{frame}{3.5: Norms Versus Metrics}

\begin{example}
Consider vectors in $\RealNumbers^n$ with the euclidean metric
\begin{equation*}
d \left( \vecnot{v}, \vecnot{w} \right)
= \sqrt{ (v_1 - w_1)^2 + \cdots + (v_n - w_n)^2 }.
\end{equation*}
Recall the bounded metric given by
\begin{equation*}
\bar{d} \left( \vecnot{v}, \vecnot{w} \right)
= \min \left\{ d \left( \vecnot{v}, \vecnot{w} \right), 1 \right\}.
\end{equation*}
\vspace{3mm}
Define $f \colon \RealNumbers^n \rightarrow \RealNumbers$ by
$f \left( \vecnot{v} \right) = \bar{d} \left( \vecnot{v}, \vecnot{0} \right)$.
Is the function $f$ a norm?

By the properties of $\overline{d}$, we have
\begin{enumerate}
\item $\bar{d} \left( \vecnot{v}, \vecnot{0} \right) \geq 0 \quad \forall \vecnot{v} \in V$; equality holds if and only if $\vecnot{v} = \vecnot{0}$
\item $\bar{d} \left( \vecnot{v} + \vecnot{w}, \vecnot{0}\right) = \bar{d} \left( \vecnot{v}, -\vecnot{w} \right) \leq \bar{d} \left( \vecnot{v}, \vecnot{0} \right) + \bar{d} \left( \vecnot{w}, \vecnot{0} \right) \quad \forall \vecnot{v}, \vecnot{w} \in V$.
\end{enumerate}

\vspace{2mm}

However, $\bar{d} \left( s \vecnot{v}, \vecnot{0} \right)$ is not always equal to $s \bar{d} \left( \vecnot{v}, \vecnot{0} \right)$.
For instance,
$\bar{d} \left( 2 \vecnot{e}_1, \vecnot{0} \right) = 1 < 2 \bar{d} \left( \vecnot{e}_1, \vecnot{0} \right)$.
Thus, the $f \left( \vecnot{v} \right) = \bar{d} \left( \vecnot{v}, \vecnot{0} \right)$
is not a norm.
\end{example}

\end{frame}

\begin{frame}{3.5: Complete Normed Spaces}

\begin{definition}<1->
A vector $\vecnot{v} \in V$ is called \textcolor{blue}{normalized} if $\left\| \vecnot{v} \right\| = 1$.
For any $\vecnot{v} \neq \vecnot{0}$, consider \vspace{-2mm}
\begin{equation*}
\vecnot{u} = \vecnot{v}\, / \left\| \vecnot{v} \right\|
\end{equation*}
with norm $\left\| \vecnot{u} \right\| = 1$.
A normalized vector is called a \textcolor{blue}{unit vector}.
\end{definition}

\begin{definition}<2->
A complete normed vector space is called a \textcolor{blue}{Banach space}.
\end{definition}

\begin{example}<3->
Vector spaces $\mathbb{R}^n$ (or $\mathbb{C}^n$) with any well-defined norm are Banach spaces.
\end{example}

\begin{example}<4->
The vector space of continuous functions $f \colon [a,b] \to \mathbb{R}$ is a Banach space under the norm \vspace{-2.5mm}
\[ \left\| f(t) \right\| = \sup_{t\in [a,b]} f(t). \vspace*{-0.5mm}\]
\end{example}

\end{frame}

\begin{frame}{3.5: Schauder Basis}

\begin{definition}
A Banach space $V$ has a \textcolor{blue}{Schauder basis}, $\vecnot{v}_1, \vecnot{v}_2, \ldots$, if every $\vecnot{v} \in V$ can be written uniquely as
\[ \vecnot{v} = \sum_{i=1}^\infty s_i \vecnot{v}_i, \]
where convergence is determined by the norm topology.
\end{definition}

\begin{example}
Let $V = \mathbb{R}^\infty$ be the vector space of semi-infinite real sequences.
The \textcolor{blue}{standard Schauder basis} is the countably infinite extension $\{\vecnot{e}_1 ,\vecnot{e}_2, \ldots \}$ of the standard basis.
\end{example}


\end{frame}

\begin{frame}{3.5: Convergence of Sums}


Banach space convergence via the induced metric $d(\vecnot{v}, \vecnot{w}) = \| \vecnot{v}-\vecnot{w} \|$.

\newcommand{\vtp}{\hspace{-0.02em}}
\begin{lemma}
If $\,\sum_{i=1}^\infty \| \vecnot{v}_i \| \vtp = \vtp a \vtp <\vtp \infty$, then $\vecnot{u}_n \vtp = \vtp \sum_{i=1}^n \vecnot{v}_i$ satisfies $\vecnot{u}_n \vtp \to \vtp \vecnot{u}$ with $\| \vecnot{u} \| \vtp \leq \vtp a$.
\end{lemma}
\vspace{-0.5mm}

\begin{proof}
\begin{itemize}
\item Let $a_n = \sum_{i=1}^n \| \vecnot{v}_i \|$ and observe that, for $n\!>\!m$, \vspace{-2mm}
\[ |a_n - a_m| = \left| \sum_{i=1}^n \| \vecnot{v}_i \| - \sum_{i=1}^m \| \vecnot{v}_i \|  \right| = \sum_{i=m+1}^n \| \vecnot{v}_i \| \] 
\vspace{-2mm}
\[ \|\vecnot{u}_n - \vecnot{u}_m \| = \left\| \sum_{i=1}^n \vecnot{v}_i - \sum_{i=1}^m \vecnot{v}_i  \right\| = \left\| \sum_{i=m+1}^n \vecnot{v}_i \right\| \leq \sum_{i=m+1}^n \| \vecnot{v}_i \| \] 

\item Since $\sum_{i=1}^\infty \| \vecnot{v}_i \|$ converges in $\RealNumbers$, $a_n$ must be a Cauchy sequence \vspace{1mm}

\item Since $\|\vecnot{u}_n - \vecnot{u}_m\| \leq |a_n - a_m|$, $\vecnot{u}_n$ is also a Cauchy sequence \vspace{1mm}

\item Once $\vecnot{u}_n$ converges, the norm bound given by the triangle inequality \hfill \qedhere

\end{itemize}

\end{proof}

\end{frame}

\begin{frame}{3.5: Open and Closed Subspaces}

\begin{definition}<1->
A \textcolor{blue}{closed subspace} of a Banach space is a subspace that is a closed set in the topology generated by the norm.
\end{definition}
\vspace{-0.5mm}

\begin{theorem}<2->
All finite dimensional subspaces of a Banach space are closed.
\end{theorem}
\vspace{-0.5mm}

\begin{example}<3->
Let $W = \{\vecnot{w}_1 ,\vecnot{w}_2 , \ldots \}$ be a linearly independent sequence of normalized vectors in a Banach space.
The span of $W$ only includes finite linear combinations.
However, a sequence of finite linear combinations, like \vspace{-2mm}
\[ \vecnot{u}_n = \sum_{i=1}^n \frac{1}{i^2} \vecnot{w}_i , \vspace{-1.5mm}\]  converges to $\lim_{n\to \infty} \vecnot{u}_n$ if it exists.
Thus, the span of $W$ is not closed.
\end{example}
\vspace{-0.5mm}

\visible<3->{Show convergence in live session.}

\end{frame}

\begin{frame} \frametitle{Next Steps}

\begin{itemize}
\setlength\itemsep{5mm}
\item To continue studying after this video -- \vspace{2mm}

\begin{itemize}
 \setlength\itemsep{3mm}
 \item Try the required reading: Course Notes EF 3.5
 \item Or the recommended reading: LADR Ch. 6A
 \item Also, look at the problems in Assignment 5
\end{itemize}
\end{itemize}

\note{
	\vspace{8mm}
	\begin{enumerate}
		\setlength\itemsep{3mm}
		\color{red}
		\item Here are some options to continue learning this material. (read) \\ [2mm]  That's it for today.  So, I'll see you next time.
	\end{enumerate}
}

\end{frame}


\end{document}


