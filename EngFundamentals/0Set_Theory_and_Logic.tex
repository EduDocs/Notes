\chapter{Set Theory and Logic}

In mathematics, a \emph{proof} is a demonstration that, assuming certain \emph{axioms}, some statement is necessarily true.
That is, a proof is a logical argument, not an empirical one.
One must demonstrate that a proposition is true in all cases before it is considered a theorem of mathematics.
An unproven proposition for which there is some sort of empirical evidence is known as a \emph{conjecture}.
Mathematical logic is the framework upon which rigorous proofs are built.
It is the study of the principles and criteria of valid inference and demonstrations.

Logicians have analyzed set theory in great details, formulating a collection of axioms that affords a broad enough and strong enough foundation to mathematical reasoning.
The standard form of axiomatic set theory is the Zermelo-Fraenkel set theory, together with the axiom of choice.
Each of the axioms included in this theory expresses a property of sets that is widely accepted by mathematicians.
It is unfortunately true that careless use of set theory can lead to contradictions.
Avoiding such contradictions was one of the original motivations for the axiomatization of set theory.

A rigorous analysis of set theory belongs to the foundations of mathematics and mathematical logic.
The study of these topics is, in itself, a formidable task.
For our purposes, it will suffice to approach basic logical concepts informally.
That is, we adopt a naive point of view regarding set theory and assume that the meaning of a set as a collection of objects is intuitively clear.
While informal logic is not itself rigorous, it provides the underpinning for rigorous proofs.
The rules we follow in dealing with sets are derived from established axioms.
At some point of your academic career, you may wish to study set theory and logic in greater detail.
Our main purpose here is to learn how to state mathematical results clearly and how to prove them.


\section{Statements}

A proof in mathematics demonstrates the truth of certain \emph{statements}.
It is therefore natural to begin with a brief discussion of statements.
A statement, or \emph{proposition}, is the content of an assertion.
It is either true or false, but cannot be both true and false at the same time.
For example, the expression ``There are no classes at Texas A\&M University today'' is a statement since it is either true or false.
The expression ``Do not cheat and do not tolerate those who do'' is not a statement.
Note that an expression being a statement does not depend on whether we personally can verify its validity.
The expression ``The base of the natural logarithm, denoted $e$, is an irrational number'' is a statement that most of us cannot prove.

Statements on their own are fairly uninteresting.
What brings value to logic is the fact that there are a number of ways to form new statements from old ones.
In this section, we present five ways to form new statements from old ones.
They correspond to the English expressions: and; or; not; if, then; if and only if.
In the discussion below, $P$ and $Q$ represent two abstract statements.

A logical \emph{conjunction} is an operation on two logical propositions that produces a value of true if both statements are true, and is false otherwise.
The \emph{conjunction} of $P$ and $Q$, denoted by $P \wedge Q$, is precisely defined by
\begin{center}
\begin{tabular}{|c|c|c|}
\hline
$P$ & $Q$ & $P \wedge Q$ \\
\hline
T & T & T \\
T & F & F \\
F & T & F \\
F & F & F \\
\hline
\end{tabular}
\end{center}

Similarly, a logical \emph{disjunction} is an operator on two logical propositions that is true if either statements is true or both are true, and is false otherwise.
The disjunction of $P$ and $Q$, denoted $P \vee Q$, is defined by
\begin{center}
\begin{tabular}{|c|c|c|}
\hline
$P$ & $Q$ & $P \vee Q$ \\
\hline
T & T & T \\
T & F & T \\
F & T & T \\
F & F & F \\
\hline
\end{tabular}
\end{center}

In mathematics, a \emph{negation} is an operator on the logical value of a proposition that sends true to false and false to true.
The negation of $P$, denoted $\neg P$, is given by
\begin{center}
\begin{tabular}{|c|c|}
\hline
$P$ & $\neg P$ \\
\hline
T & F \\
F & T \\
\hline
\end{tabular}
\end{center}

The next method of combining mathematical statements is slightly more subtle than the preceding ones.
It is connected to the notion of logical implication.
The \emph{conditional} from $P$ to $Q$, denoted $P \rightarrow Q$, is mathematically true if it is not the case that $P$ is true and $Q$ is false.
The precise definition of $P \rightarrow Q$ is given in the truth table
\begin{center}
\begin{tabular}{|c|c|c|}
\hline
$P$ & $Q$ & $P \rightarrow Q$ \\
\hline
T & T & T \\
T & F & F \\
F & T & T \\
F & F & T \\
\hline
\end{tabular}
\end{center}
Although it may seem strange at first glance, this truth table is universally accepted by mathematicians.
It is important to get used to it.
Logicians draw a firm distinction between the \emph{conditional connective} and the \emph{implication relation}.
These logicians use the phrase ``if $P$ then $Q$'' for the conditional connective and the phrase ``$P$ implies $Q$'' for the implication relation.
They explain the difference between these two forms by saying that the conditional is the contemplated relation, while the implication is the asserted relation.
We will discuss this distinction in the Section~\ref{section:Relations}, where we formally study relations between statements.
The importance and soundness of the conditional form $P \rightarrow Q$ will become clearer then.

Finally, the logical \emph{biconditional} is an operator connecting two logical propositions that is true if the statements are both true or both false, and it is false otherwise.
The \emph{biconditional} from $P$ to $Q$, denoted $P \leftrightarrow Q$, is precisely defined by
\begin{center}
\begin{tabular}{|c|c|c|}
\hline
$P$ & $Q$ & $P \leftrightarrow Q$ \\
\hline
T & T & T \\
T & F & F \\
F & T & F \\
F & F & T \\
\hline
\end{tabular}
\end{center}
We read $P \leftrightarrow Q$ as ``$P$ if and only if $Q$.''
The phrase ``if and only if'' is often abbreviated as ``iff.''

Using the five basic operations defined above, it is possible to form more complicated compound statements.
We sometime need parentheses to avoid ambiguity in writing compound statements.
We use the convention that $\neg$ takes precedence over the other four operations, but none of these operations takes precedence over the others.
For example, let $P$, $Q$ and $R$ be three propositions.
We wish to make a truth table for the following statement,
\begin{equation} \label{equation:LogicStatement}
(P \rightarrow R) \wedge (Q \vee \neg R) .
\end{equation}
We can form the true table for this statement, using simple steps, as follows
\begin{center}
\begin{tabular}{|c|c|c|ccccccc|}
\hline
$P$ & $Q$ & $R$
& $(P$ & $\rightarrow$ & $R)$ & $\wedge$ & $(Q$ & $\vee$ & $\neg R)$ \\
\hline
T & T & T & T & T & T & T & T & T & F \\
T & T & F & T & F & F & F & T & T & T \\
T & F & T & T & T & T & F & F & F & F \\
T & F & F & T & F & F & F & F & T & T \\
F & T & T & F & T & T & T & T & T & F \\
F & T & F & F & T & F & T & T & T & T \\
F & F & T & F & T & T & F & F & F & F \\
F & F & F & F & T & F & T & F & T & T \\
& & & 1 & 5 & 2 & 7 & 3 & 6 & 4 \\
\hline
\end{tabular}
\end{center}
%\end{example}

We conclude this section with a brief mention of two important concepts.
A \emph{tautology} is a statement that is true in every valuation of its propositional variables, independent of the truth values assigned to these variables.
The proverbial tautology is $P \vee \neg P$,
\begin{center}
\begin{tabular}{|c|ccc|}
\hline
$P$ & $P$ & $\vee$ & $\neg P$ \\
\hline
T & T & T & F \\
F & F & T & T \\
& 1 & 3 & 2 \\
\hline
\end{tabular}
\end{center}
For instance, the statement ``The Aggies won their last football game or the Aggies did not win their last football game'' is true regardless of whether the Aggies actually defeated their latest opponent.

The negation of a tautology is a \emph{contradiction}, a statement that is necessarily false regardless of the truth values of its propositional variables.
The statement $P \wedge \neg P$ is a contradiction, and its truth table is
\begin{center}
\begin{tabular}{|c|ccc|}
\hline
$P$ & $P$ & $\wedge$ & $\neg P$ \\
\hline
T & T & F & F \\
F & F & F & T \\
& 1 & 3 & 2 \\
\hline
\end{tabular} .
\end{center}

Of course, most statement we encounter are neither tautologies nor contradictions.
For example, \eqref{equation:LogicStatement} is not necessarily either true or false.
Its truth value depends on the values of $P$, $Q$ and $R$.
Try to see whether the statement
\begin{equation*}
((P \wedge Q) \rightarrow R) \rightarrow (P \rightarrow (Q \rightarrow R))
\end{equation*}
is a tautology, a contradiction, or neither.


\section{Relations between Statements}
\label{section:Relations}

Strictly speaking, relations between statements are not formal statements themselves.
They are \emph{meta-statements} about some propositions.
We study two types of relations between statements, \emph{implication} and \emph{equivalence}.
An example of an implication meta-statement is the observation that ``if the statement `Robert graduated from Texas A\&M University' is true, then it implies that the statement `Robert is an Aggie' is also true.''
Another example of a meta-statement is ``the statement `Fred is an Aggie and Fred is honest' being true is equivalent to the statement `Fred is honest and Fred is an Aggie' being true.''
These two examples illustrate how meta-statements describe the relationship between statements.
It is also instructive to note that implications and equivalences are the meta-statements analogs of conditionals and biconditionals.

The logical implication can be intuitively described as $P$ implies $Q$ if necessarily $Q$ is true whenever $P$ is true.
That is, $Q$ cannot be false if $P$ is true.
Necessity is the key aspect of this sentence, the fact that $P$ and $Q$ both happen to be true cannot be coincidental.
To have $P$ implies $Q$, we need the conditional $P \rightarrow Q$ to be true under all possible circumstances.

The notion of implication can be rigorously defined as follows, $P$ implies $Q$ if the statement $P \rightarrow Q$ is a tautology.
We abbreviate $P$ implies $Q$ by writing $P \Rightarrow Q$.
It is important to understand the difference between ``$P \rightarrow Q$'' and ``$P \Rightarrow Q$.''
The former, $P \rightarrow Q$, is a compound statement that may or may not be true.
On the other hand, $P \Rightarrow Q$ is a relation stating that the compound statement $P \rightarrow Q$ is true under all instances of $P$ and $Q$.

While the distinction between implication and conditional may seem extraneous, we will soon see that meta-statements become extremely useful in building valid arguments.
In particular, the following implications are used extensively in constructing proofs.
\begin{fact}
Let $P$, $Q$, $R$ and $S$ be statements.
\begin{enumerate}
\item $(P \rightarrow Q) \wedge P \Rightarrow Q$.
\item $(P \rightarrow Q) \wedge \neg Q \Rightarrow \neg P$.
\item $P \wedge Q \Rightarrow P$.
\item $(P \vee Q) \wedge \neg P \Rightarrow Q$.
\item $P \leftrightarrow Q \Rightarrow P \rightarrow Q$.
\item $(P \rightarrow Q) \wedge (Q \rightarrow P) \Rightarrow P \rightarrow Q$.
\item $(P \rightarrow Q) \wedge (Q \rightarrow R) \Rightarrow P \rightarrow R$
\item $(P \rightarrow Q) \wedge (R \rightarrow S) \wedge (P \vee R) \Rightarrow Q \vee S$.
\end{enumerate}
\end{fact}

As an illustrative example, we show that $(P \rightarrow Q) \wedge (Q \rightarrow R)$ implies $P \rightarrow R$.
To demonstrate this assertion, we need to show that
\begin{equation} \label{equation:HypotheticalSyllogism}
((P \rightarrow Q) \wedge (Q \rightarrow R)) \rightarrow (P \rightarrow R)
\end{equation}
is a tautology.
This is accomplished in the truth table below
\begin{center}
\begin{tabular}{|c|c|c|ccccccccccc|}
\hline
$P$ & $Q$ & $R$
& $((P$ & $\rightarrow$ & $Q)$ & $\wedge$ & $(Q$ & $\rightarrow$ & $R))$ & $\rightarrow$ & $(P$ & $\rightarrow$ & $R)$ \\
\hline
T & T & T & T & T & T & T & T & T & T & T & T & T & T \\
T & T & F & T & T & T & F & T & F & F & T & T & F & F \\
T & F & T & T & F & F & F & F & T & T & T & T & T & T \\
T & F & F & T & F & F & F & F & T & F & T & T & F & F \\
F & T & T & F & T & T & T & T & T & T & T & F & T & T \\
F & T & F & F & T & T & F & T & F & F & T & F & T & F \\
F & F & T & F & T & F & T & F & T & T & T & F & T & T \\
F & F & F & F & T & F & T & F & T & F & T & F & T & F \\
& & & 1 & 7 & 2 & 10 & 3 & 8 & 4 & 11 & 5 & 9 & 6 \\
\hline
\end{tabular} .
\end{center}
Column~11 has the truth values for statement \eqref{equation:HypotheticalSyllogism}.
Since \eqref{equation:HypotheticalSyllogism} is true under all circumstances, it is a tautology and the implication holds.
Showing that the other relations are valid is left to the reader as an exercise.

Logical implications are not always reversible.
For instance, although $(P \rightarrow Q) \wedge (Q \rightarrow P)$ implies $P \rightarrow Q$, the converse is not true.
It can easily be shown from columns~9~\&~10 above that
\begin{equation*}
(P \rightarrow R) \rightarrow ((P \rightarrow Q) \wedge (Q \rightarrow R))
\end{equation*}
is not a tautology.
That is, $P \rightarrow R$ certainly does not imply $(P \rightarrow Q) \wedge (Q \rightarrow R)$.

A logical implication that is reversible is called an equivalence.
More precisely, $P$ is equivalent to $Q$ if the statement $P \leftrightarrow Q$ is a tautology.
We denote the sentence ``$P$ is equivalent to $Q$'' by simply writing ``$P \Leftrightarrow Q$.''
It is easily seen that $P \Leftrightarrow Q$ holds if and only if $P \Rightarrow Q$ and $Q \Rightarrow P$ are both true.
Being able to recognize that two statements are equivalent will become handy.
It is sometime possible to demonstrate a result by finding an alternative, equivalent form of the statement that is easier to prove than the original form.
A list of important equivalences appears below.

\begin{fact}
Let $P$, $Q$ and $R$ be statements.
\begin{enumerate}
\item $\neg (\neg P) \Leftrightarrow P$.
\item $P \vee Q \Leftrightarrow Q \vee P$.
\item $P \wedge Q \Leftrightarrow Q \wedge P$.
\item $(P \vee Q) \vee R \Leftrightarrow P \vee (Q \vee R)$.
\item $(P \wedge Q) \wedge R \Leftrightarrow P \wedge (Q \wedge R)$.
\item $P \wedge (Q \vee R) \Leftrightarrow (P \wedge Q) \vee (P \wedge R)$.
\item $P \vee (Q \wedge R) \Leftrightarrow (P \vee Q) \wedge (P \vee R)$.
\item $P \rightarrow Q \Leftrightarrow \neg P \vee Q$.
\item $P \rightarrow Q \Leftrightarrow \neg Q \rightarrow \neg P$ (Contrapositive).
\item $P \leftrightarrow Q \Leftrightarrow (P \rightarrow Q) \wedge (Q \rightarrow P)$.
\item $\neg (P \wedge Q) \Leftrightarrow \neg P \vee \neg Q$ (De Morgan's Law).
\item $\neg (P \vee Q) \Leftrightarrow \neg P \wedge \neg Q$ (De Morgan's Law).
\end{enumerate}
\end{fact}

The first equivalence in this list, $\neg (\neg P) \Leftrightarrow P$, may appear trivial.
However, from the point of view of constructing mathematical proofs, this equivalence is frequently employed.
Indeed, one method to prove that statement $P$ is true is to hypothesize that $\neg P$ is true and then derive a contradiction.
It then follows that $\neg P$ is false, which implies that $P$ is true.
This popular technique is called \emph{proof by contradiction}.
It is illustrated below through a classic example.

The equivalence $P \leftarrow Q \Leftrightarrow \neg Q \leftarrow \neg P$ is also frequently used in constructing mathematical proofs.
Given a conditional statement of the form $P \rightarrow Q$, we call $\neg Q \rightarrow \neg P$ the \emph{contrapositive} of the original statement.


\subsection{Fallacious Arguments}

A \emph{fallacy} is a component of an argument that is demonstrably flawed in its logic or form, thus rendering the argument invalid.
Recognizing fallacies in mathematical proofs may be difficult since arguments are often structured using convoluted patterns that obscure the logical connections between assertions.
We give below examples for three types of fallacies that are often found in attempted mathematical proofs.

\paragraph{Affirming the Consequent:}
If the Indian cricket team wins a test match, then all the players will drink tea together.
All the players drank tea together.
Therefore the Indian cricket team won a test match.

\paragraph{Denying the Antecedent:}
If Diego Maradona drinks coffee, then he will be fidgety.
Diego Maradona did not drink coffee.
Therefore, he is not fidgety.

\paragraph{Unwarranted Assumptions:}
If Yao Ming gets close to the basket, then he scores a lot of points.
Therefore, Yao Ming scores a lot of points.


\subsection{Quantifiers}

Quantifiers are of paramount importance in rigorous proofs.
They are employed to make statements about collections of elements.
Universal quantification is used to formalize the notion that a statement is true for all possible values of a collection.
The \emph{universal quantifier} is typically denoted by $\forall$ and it is informally read ``for all.''
Let $U$ be a specific collection of elements, and let $P(x)$ be a statement that applies to $x$.
Then the statement $\forall x \in U, P(x)$ is true if $P(x)$ is true for all values of $x$ in $U$.
The other type of quantifier often encountered in mathematical proofs is the \emph{existential quantifier}, denoted $\exists$.
The statement $\exists x \in U, P(x)$ is true if $P(x)$ is true for at least one value of $x$ in $U$.

In mathematics, a \emph{free variable} is a notation for a place or places in an expression into which some definite substitution may take place, or with respect to which some operation (e.g. quantification) may take place. 
A \emph{bound variable} is a variable for which we have no ability to choose the value.


\section{Strategies for Proofs}

The relation between intuition and formal rigor is not a trivial matter.
Intuition tells us what is important, what might be true, and what mathematical tools may be used to prove it.
Rigorous proofs are used to verify that a given statement that appears intuitively true is indeed true.
Ultimately, a mathematical proof is a convincing argument that starts from some premises, and logically deduces the desired conclusion.
Most proofs do not mention the logical rules of inference used in the derivation.
Rather, they focus on the mathematical justification of each step, leaving to the reader the task of filling the logical gaps.
The mathematics is the major issue.
Yet, it is essential that you understand the underlying logic behind the derivation as to not get confused while reading or writing a proof.

True statements in mathematics have different names.
They can be called theorems, propositions, lemmas, corollaries and exercises.
A \emph{theorem} is a statement that can be proved on the basis of explicitly stated or previously agreed assumptions.
A \emph{proposition} is a statement not associated with any particular theorem; this term sometimes connotes a statement with a simple proof.
A \emph{lemma} is a proven proposition which is used as a stepping stone to a larger result rather than an independent statement in itself.
A \emph{corollary} is a mathematical statement which follows easily from a previously proven statement, typically a mathematical theorem. 
The distinction between these names and their definitions is somewhat arbitrary.
Ultimately, they are all synonymous to a true statement.

A proof should be written in grammatically correct English.
Complete sentences should be used, with full punctuation.
In particular, every sentence should end with a period, even if the sentence ends in a displayed equation.
Mathematical formulas and symbols are parts of sentences, and are treated no differently than words.
One way to learn to construct proofs is to read a lot of well written proofs, to write progressively more difficult proofs, and to get detailed feedback on the proofs you write.


\paragraph{Direct Proof:}

The simplest form of proof for an implication $P \rightarrow Q$ is the \emph{direct proof}.
First assume that $P$ is true.
Produce a series of steps, each one following from the previous ones, that eventually leads to conclusion $Q$.
It warrants the name "direct proof" only to distinguish it from other, more intricate, methods of proof.

\paragraph{Proof by Contrapositive:}
A proof by contrapositive takes advantage of the mathematical equivalence $P \rightarrow Q \Leftrightarrow \neg Q \rightarrow \neg P$.
That is, a proof by contrapositive begins by assuming that $\neg Q$ is true.
A series of implications then leads to the conclusion that $\neg P$ must also be true.
It follows that if $P$ is true then $\neg Q$ is false, which implies that $Q$ must be true.

\paragraph{Proof by Contradiction:}
A proof by contradiction is based on the mathematical equivalence $\neg (P \rightarrow Q) \Leftrightarrow P \wedge \neg Q$.
In a proof by contradiction, it is shown that if some statement were false, a logical contradiction occurs, hence the statement must be true.

\begin{example} \label{example:SquareRoot2}
We wish to show that $\sqrt{2}$ is an irrational number.

First, assume that $\sqrt{2}$ is a rational number.
This assumption implies that there exist integers $p$ and $q$ with $q \neq 0$ such that $p/q = \sqrt{2}$.
In fact, we can further assume that the fraction $p/q$ is irreducible.
That is, $p$ and $q$ are coprime integers (they have no common factor greater than 1).
From $p/q = \sqrt{2}$, it follows that $p = \sqrt{2} q$, and so $p^2 = 2 q^2$.
Thus $p^2$ is an even number, which implies that $p$ itself is even (only even numbers have even squares).
Because $p$ is even, there exists an integer $r$ satisfying $p = 2r$.
We then obtain the equation $(2r)^2 = 2q^2$, which is equivalent to $2r^2 = q^2$ after simplification.
Because $2r^2$ is even, it follows that $q^2$ is even, which means that $q$ is also even.
We conclude that $p$ and $q$ are both even.
This contradicts the fact that $p/q$ is irreducible.
Hence, the initial assumption that $\sqrt{2}$ is a rational number must be false.
That is to say, $\sqrt{2}$ is irrational.
\end{example}

\begin{example}
Consider the following statement, which is related to Example~\ref{example:SquareRoot2}.
``If $\sqrt{2}$ is rational, then it can be expressed as an irreducible fraction.''
The contrapositive of this statement is ``If $\sqrt{2}$ cannot be expressed as an irreducible fraction, then it is not rational.''
Above, we proved that $\sqrt{2}$ cannot be expressed as an irreducible fraction and therefore $\sqrt{2}$ is not a rational number.
\end{example}


\newpage
\section{Other Stuff}


ECEN 601 is a transition course that intends to bridges the computational concepts typical of an undergraduate curriculum in electrical and computer engineering and the rigorous mathematical arguments used in theoretical research.
Its purpose is to make advance mathematics accessible to students in electrical and computer engineering, and to provide them with a solid basis upon which to build their graduate work.

The course is divided into three parts:

Both engineering intuition and facility with writing proofs can be developed with practice.

Proofs are used to validate intuitive ideas about engineering systems.

Like music and art, theoretical engineering is learned by doing, not just by reading texts and listening to lectures.

Doing the exercises is the best way to get a feel for the material, test your understanding and see what needs further study.
It is important that you try to do the exercises by yourself prior to discussing with other students or your instructor.
Once solved, problems often appear much easier than they really are.
Testing your knowledge during exams is a dangerous proposition.

In this book we decided to take a good thing and make it better, adding the symbol $\triangle$ to the end of a definition, the symbol $\diamond$ for the end of an example.

Physical sloppiness is often a sign of either laziness or disrespect, and sloppiness in writing style is often a mask for sloppy thinking.


Functional knowledge,
but can you prove it?

A statement is something that can be verified, it has to be either true or false.
We will be making two assumptions when dealing with statements: every statement is either true or false, and no statement is both true and false.
One of the consequences of this law is that if a statement is not false, then it must be true.
Hence, to prove that something is true, it would suffice to show that it is not false.

"Write a computer program."
"Eat a pineapple"

What makes statement valuable for our purposes is that there are a number of ways of forming new statements from old ones.

In literary writing, some measure of ambiguity is often acceptable, and sometime valuable.
In mathematics, by contrast, precision is key; ambiguity is to be avoided at all cost.
When using a mathematical term, always stick to the precise mathematical definition, regardless of any other colloquial usage.

