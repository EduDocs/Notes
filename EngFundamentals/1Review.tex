\chapter{Metric Spaces and Topology}

\section{Metric Spaces}

\index{metric space}
A \textbf{metric space} is a set that has a well-defined ``distance'' between any two elements of the set.
Mathematically, the notion of a metric space abstracts a few basic properties of Euclidean space.
Formally, a metric space $(X,d)$ is a set $X$ and a function $d$ that is a metric on $X$.
\begin{definition}
A \defn{metric space}{metric} on a set $X$ is a function
\begin{equation*}
d: X \times X \rightarrow \RealNumbers
\end{equation*}
that satisfies the following properties,
\begin{enumerate}
\item $d(x,y) \geq 0 \quad \forall x, y \in X$; equality holds if and only if $x = y$
\item $d(x,y) = d(y,x) \quad \forall x, y \in X$
\item $d(x,y) + d(y,z) \geq d(x,z) \quad \forall x, y, z \in X$.
\end{enumerate}
\end{definition}

\begin{example}
Given $\vecnot{x} = (x_1, \ldots, x_n), \vecnot{y} = (y_1, \ldots, y_n) \in \RealNumbers^n$, the \defn{metric space}{euclidean metric} $d$ on $\RealNumbers^n$ is defined by the equation
\begin{equation*}
d \left( \vecnot{x}, \vecnot{y} \right)
= \sqrt{ (x_1 - y_1)^2 + \cdots + (x_n - y_n)^2 }.
\end{equation*}
As implied by its name, the function $d$ defined above is a metric.
\end{example}

\begin{problem}
Let $\vecnot{x} = (x_1, \ldots, x_n), \vecnot{y} = (y_1, \ldots, y_n) \in \RealNumbers^n$ and consider the function $\rho$ given by
\begin{equation*}
\rho \left( \vecnot{x}, \vecnot{y} \right)
= \max \left\{ |x_1 - y_1|, \ldots, |x_n - y_n| \right\}.
\end{equation*}
Show that $\rho$ is a metric.
\end{problem}

\begin{problem} \label{problem:StandardBoundedMetric}
Let $X$ be a metric space with metric $d$.
Define $\bar{d} : X \times X \rightarrow \RealNumbers$ by
\begin{equation*}
\bar{d} (x, y)
= \min \left\{ d (x, y), 1 \right\}.
\end{equation*}
Show that $\bar{d}$ is also a metric.
\end{problem}

Let $(X,d)$ be a metric space.
Then, elements of $X$ are called \defn{metric space}{points} and the number $d(x,y)$ is called the \defn{metric space}{distance} between $x$ and $y$.
Let $\epsilon > 0$ and consider the set $B_d(x,\epsilon) = \left\{ y \in X | d(x,y) < \epsilon \right\}$.
This set is called the \defn{metric space}{$d$-open ball} of radius $\epsilon$ centered at $x$.

\begin{problem}
Suppose $a \in B_d(x, \epsilon)$ with $\epsilon > 0$.
Show that there exists a $d$-open ball centered at $a$ of radius $\delta$ (i.e., $B_d(a, \delta)$) that is contained in $B_d(x, \epsilon)$.
\end{problem}

One of the main benefits of having a metric is that it provides some notion of ``closeness'' between points in a set.
This alows one to discusss limits, convergence, open sets, and closed sets.

\begin{definition}
A \defn{metric space}{sequence} of elements from a set $X$ is an infinite list $x_1,x_2,\ldots$ where $x_i \in X$ for all $i\in \mathbb{N}$.
Formally, a sequence is equivalent to a function $f: \mathbb{N} \rightarrow X$ where $x_i = f(i)$ for all $i\in \mathbb{N}$.
\end{definition}

\begin{definition}
Consider a sequence $x_1, x_2, \ldots$ of points in a metric space $(X,d)$.
This sequence \defn{metric space}{$d$-converges} to $x\in X$ if, for any $\epsilon >0$, there is natural number $N$ such that $d(x,x_n) < \epsilon$ for all $n>N$.
\end{definition}

\begin{definition}
A sequence $x_1,x_2,\ldots$ in $(X,d)$ is a \defn{metric space}{Cauchy sequence} if, for any $\epsilon >0$, there is a natural number $N$ (depending on $\epsilon$) such that, for all $m,n > N$,
\begin{equation*}
d \left( x_m, x_n \right) < \epsilon.
\end{equation*}
\end{definition}

\begin{theorem}
Every $d$-convergent sequence is a Cauchy sequence.
\end{theorem}
\begin{proof}
Since $x_1,x_2,\ldots$ $d$-converges to some $x$, there is an $N$, for any $\epsilon>0$, such that $d(x,x_n)<\epsilon /2$ for all $n>N$.
The triangle inequality for $d(x_m,x_n)$ shows that, for all $m,n>N$,
\[ d(x_m,x_n)\leq d(x_m,x) + d(x,x_n) \leq \epsilon/2 + \epsilon/2 = \epsilon. \]
Therefore, $x_1,x_2,\ldots$ is a Cauchy sequence.
\end{proof}

\begin{definition}
Let $W$ be a subset of a metric space $(X,d)$.
The set $W$ is called \defn{metric space}{$d$-open} if, for every $w\in W$, there is an $\epsilon>0$ such that $B_d (w,\epsilon) \subseteq W$.
\end{definition}

\begin{theorem}
\label{theorem:d_open_closure}
For any metric space $(X,d)$,
\begin{enumerate}
\item $\emptyset$ and $X$ are $d$-open
\item any union of $d$-open sets is $d$-open
\item any finite intersection of $d$-open sets if $d$-open
\end{enumerate}
\end{theorem}
\begin{proof}
This proof is left as an exercise for the reader.
\end{proof}

\begin{definition}
A subset $W$ of a metric space $(X,d)$ is $d$-closed if its complement $W^c = X-W$ is $d$-open.
\end{definition}

\begin{corollary}
For any metric space $(X,d)$,
\begin{enumerate}
\item $\emptyset$ and $X$ are $d$-closed
\item any intersection of $d$-closed sets is $d$-closed
\item any finite union of $d$-closed sets if $d$-closed
\end{enumerate}
\end{corollary}
\begin{proof}[Sketch of proof]
Using the definition of $d$-closed, one can apply De Morgan's Laws to Theorem \ref{theorem:d_open_closure} verify this result.
\end{proof}

\begin{definition}
Suppose $f: X \rightarrow Y$ is a function from the metric space $(X,d_X)$ to the metric space $(Y,d_Y)$.
Then, $f$ is \defn{metric space}{$d$-continuous} at $x_0$ if, for any $\epsilon> 0$, there is a $\delta >0$ such that 
\[ d_Y \left( f(x_0),f(x) \right) < \epsilon\]
for all $x\in X$ such that $d_X(x_0,x)< \delta$.
\end{definition}

\begin{definition}
A function $f : X \rightarrow Y$ is called \textbf{$d$-continuous} if it is $d$-continuous at $x_0$ for all $x_0 \in X$.
\end{definition}

\begin{definition}
A metric space $(X,d)$ is said to be \defn{topology}{complete} if every Cauchy sequence in $X$ converges to a limit $x \in X$.
\end{definition}

\begin{example}
Any closed subset of $\mathbb{R}^n$ (or $\mathbb{C}^n$) is complete.
\end{example}

\begin{example}
Consider the sequence $x_n \in \mathbb{Q}$ defined by $x_n = \left( 1 + \frac{1}{n} \right)^n$.
It is well-known that this sequence converges to $e\in \mathbb{R}$, but this number is not rational.
Therefore, the rational numbers $\mathbb{Q}$ are not complete.
\end{example}

\section{General Topology}

\index{topology}
\begin{definition}
A \textbf{topology} on a set $X$ is a collection $\mathcal{J}$ of subsets of $X$ that satisfies the following properties,
\begin{enumerate}
\item $\emptyset$ and $X$ are in $\mathcal{J}$
\item the union of the elements of any subcollection of $\mathcal{J}$ is in $\mathcal{J}$
\item the intersection of the elements of any finite subcollection of $\mathcal{J}$ is in $\mathcal{J}$.
\end{enumerate}
\end{definition}

A subset $A \subseteq X$ is called an \defn{topology}{open set} of $X$ if $A \in \mathcal{J}$.
Using this terminology, a topological space is a set $X$ together with a collection of subsets of $X$, called \emph{open sets}, such that $\emptyset$ and $X$ are both open and such that arbitrary unions and finite intersections of open sets are open.

\begin{definition}
If $X$ is a set, a \defn{topology}{basis} for a topology on $X$ is a collection $\mathcal{B}$ of subsets of $X$ (called \emph{basis elements}) such that
\begin{enumerate}
\item for each $x \in X$, there exists a basis element $B$ containing $x$
\item if $x \in B_1$ and $x \in B_2$ where $B_1, B_2 \in \mathcal{B}$, then there exists a basis element $B_3$ containing $x$ such that $B_3 \subset B_1 \cap B_2$.
\end{enumerate}
\end{definition}

If $\mathcal{B}$ is a basis for a topology on $X$, the topology $\mathcal{J}$ generated by $\mathcal{B}$ is described as follows.
A subset $A \subseteq X$ is open in $X$ if for each $x \in A$ there exists a basis element $B \in \mathcal{B}$ such that $x \in B$ and $B \subseteq A$. 

Probably the most important and frequently used way of imposing a topology on a set is to define the topology in terms of a metric.
\begin{definition}
If $d$ is a metric on the set $X$, then the collection of all $\epsilon$-balls
\begin{equation*}
\left\{ B_d (x, \epsilon) |  x \in X, \epsilon > 0 \right\}
\end{equation*}
is a basis for a topology on $X$.
This topology is called the \defn{topology}{metric topology} induced by $d$.
\end{definition}

In particular, a set $A$ is open in the metric topology induced by $d$ if and only if for each $a \in A$, there exists a $\delta > 0$ such that $B_d (a, \delta) \subset A$.
Clearly, this condition implies that $A$ is open.
Conversely, if $A$ is open, it contains a basis element $B = B_d (x, \epsilon)$ containing $a$, and $B$ in turn contains a basis element $B_d (a, \delta)$ centered at $a$.

\begin{definition}
Let $X$ be a topological space.
The space $X$ is said to be \defn{topology}{metrizable} if there exists a metric $d$ on the set $X$ that induces the topology of $X$.
A metric space is a space $X$ together with a specific metric $d$ that gives the topology of $X$.
\end{definition}


\subsection{Closed Sets and Limit Points}

\begin{definition}
A subset $A$ of a topological space $X$ is \defn{topology}{closed} if the set
\begin{equation*}
\Complement{A} = X - A = \{ x \in X | x \notin A \}
\end{equation*}
is open.
\end{definition}

Note that a set can be open, closed, both, or neither!
It can be shown that the collection of closed subsets of a space $X$ has properties similar to those satisfied by the collection of open subsets of $X$.

\begin{fact}
Let $X$ be a topological space.
The following conditions hold,
\begin{enumerate}
\item $\emptyset$ and $X$ are closed
\item arbitrary intersections of closed sets are closed
\item finite unions of closed sets are closed.
\end{enumerate}
\end{fact}

\begin{definition}
Given a subset $A$ of a topological space $X$, the \defn{topology}{interior} of $A$ is defined as the union of all open sets contained in $A$.
The \defn{topology}{closure} of $A$ is defined as the intersection of all closed sets containing $A$.
\end{definition}

The interior of $A$ is denoted by $\Interior{A}$ and the closure of $A$ is denoted by $\Closure{A}$.
We note that $\Interior{A}$ is open and $\Closure{A}$ is closed.
Furthermore, $\Interior{A} \subseteq A \subseteq \Closure{A}$.

\begin{theorem} \label{theorem:ClosureConditions}
Let $A$ be a subset of the topological space $X$.
The element $x$ is in $\Closure{A}$ if and only if every open set $B$ containing $x$ intersects $A$.
Furthermore, if the topology of $X$ is given by a basis, then $x \in \Closure{A}$ if and only if every basis element $B$ containing $x$ intersects $A$.
\end{theorem}

\begin{definition}
An open set $O$ containing $x$ is called an \defn{topology}{neighborhood} of $x$.
\end{definition}

\begin{definition}
Suppose $A$ is a subset of the topological space $X$ and let $x$ be an element of $X$.
Then $x$ is a \defn{topology}{limit point} of $A$ if every neighborhood of $x$ intersects $A$ in some point other than $x$ itself.
\end{definition}

In other words, $x \in X$ is a limit point of $A \subset X$ if $x \in \Closure{A - \{ x \}}$, the closure of $A - \{ x \}$.
The point $x$ may or may not be in $A$.

\begin{theorem}
A subset of a topological space is closed if and only if it contains all its limit points.
\end{theorem}


\subsection{Continuity}

\begin{definition}
Let $X$ and $Y$ be topological spaces.
A function $f: X \rightarrow Y$ is \defn{topology}{continuous} if for each open subset $O \subseteq Y$, the set $f^{-1} (O)$ is an open subset of $X$.
\end{definition}

Recall that $f^{-1}(B)$ is the set $\{ x \in X | f(x) \in B \}$.
Continuity of a function depends not only upon the function $f$ itself, but also on the topologies specified for its domain and range!

\begin{theorem} \label{theorem:ContinuityEquivalentConditions}
Let $X$ and $Y$ be topological spaces and consider a function $f: X \rightarrow Y$.
The following are equivalent:
\begin{enumerate}
\item $f$ is continuous
\item for every subset $A \subseteq X$, $f \left( \Closure{A} \right) \subseteq \Closure{ f(A) }$
\item for every closed set $C \subset Y$, the set $f^{-1} (C)$ is closed in $X$.
\end{enumerate}
\end{theorem}
\begin{proof}
$(1 \Rightarrow 2)$.
Assume $f$ is a continuous function.
Suppose $x \in \Closure{A}$, where $A$ is a subset of $X$.
Let $O$ be a neighborhood of $f(x)$.
Then $f^{-1}(O)$ is an open set of $X$ containing $x$; it must intersect with $A$ in some point $x'$.
It follows that $O$ intersects $f(A)$ in the point $f(x')$.
Hence $f(x) \in \Closure{ f(A) }$.

$(2 \Rightarrow 3)$.
Suppose that for every subset $A \subseteq X$, $f \left( \Closure{A} \right) \subseteq \Closure{ f(A) }$.
Let $C \subseteq Y$ be a closed set and let $A = f^{-1} (C)$.
By elementary set theory, we have $f(A) \subseteq C$.
If $x \in \Closure{A}$, then
\begin{equation*}
f(x) \in f \left( \Closure{A} \right) \subseteq \Closure{f (A)} \subseteq \Closure{C} = C.
\end{equation*}
So that $x \in f^{-1} (C) = A$ and, as a consequence, $\Closure{A} \subseteq A$.

$(3 \Rightarrow 1)$.
Let $O$ be an open set in $Y$.
Let $\Complement{O} = Y - O$; then $\Complement{O}$ is closed in $Y$.
By assumption, $f^{-1} (\Complement{O})$ is closed in $X$.
Using elementary set theory,
\begin{equation*}
f^{-1} (O) = f^{-1} (Y - \Complement{O}) = f^{-1}(Y) - f^{-1}(\Complement{O}) = X - f^{-1}(\Complement{O}).
\end{equation*}
That is, $f^{-1} (O)$ is open.
\end{proof}

\begin{theorem}
Suppose $X$ and $Y$ are two metrizable spaces with metrics $d_X$ and $d_Y$.
Consider a function $f : X \rightarrow Y$.
The function $f$ is continuous if and only if it $d$-continuous with these metrics.
\end{theorem}
\begin{proof}
Suppose that $f$ is continuous.
For any $x_1 \in X$ and $\epsilon > 0$, let $O_y =  B_{d_Y} (f(x_1), \epsilon)$ and consider the set
\begin{equation*}
O_x = f^{-1} \left( O_y \right)
\end{equation*}
which is open in $X$ and contains the point $x_1$.
Since $O_x$ is open and $x_1 \in O_x$, there exists a $\delta$-ball $B_{d_X} (x_1, \delta)$ centered at $x_1$ such that $B_{d_X} (x_1, \delta) \subset O_x$.
We also see that $f(x_2) \in O_y$ for any $x_2 \in B_{d_X} (x_1, \delta)$ because $A \subseteq O_x$ implies $f(A) \subseteq O_y$.
It follows that $d_Y \left( f(x_1), f(x_2) \right) < \epsilon$ for all $x_2 \in B_{d_X} (x_1, \delta)$.

Conversely, let $O_y$ be an open set in $Y$ and suppose that the $\epsilon$-$\delta$ condition is satisfied.
For any $x \in f^{-1} (O_y)$, $f(x) \in O_y$ and there exists an $\epsilon$-ball $B_{d_Y} (f(x), \epsilon)$ centered at $f(x)$ contained in $O_y$.
By the $\epsilon$-$\delta$ condition for continuity, there exits a $\delta$-ball $B_{d_X} (x, \delta)$ centered at $x$ such that $f \left( B_{d_X} (x,\delta) \right) \subset B_{d_Y} \left( f(x), \epsilon \right)$.
Therefore, every $x \in f^{-1}(O_y)$ has a neighborhood in the same set, and that implies $f^{-1} (O_y)$ is open.
\end{proof}

\begin{definition} \label{definition:SequenceConvergence}
A sequence $x_1, x_2, \ldots$ of points in $X$ is said to \defn{topology}{converge} to $x \in X$ if for every neighborhood $O$ of $x$ there exists a positive integer $N$ such that $x_i \in O$ for all $i \geq N$.
\end{definition}

A sequence need not converge at all.
However, if it converges in a metrizable space, then it converges to only one element.

\begin{theorem}
Suppose that $X$ is a metrizable space, and let $A \subseteq X$.
There exists a sequence of points of $A$ converging to $x$ if and only if $x \in \Closure{A}$.
\end{theorem}
\begin{proof}
Suppose $x_n \rightarrow x$, where $x_n \in A$.
By Theorem~\ref{theorem:ClosureConditions}, every neighborhood $O$ of $x$ contains a point of $A$, so $x \in \Closure{A}$.
Let $d$ be a metric for the topology of $X$.
For each positive integer $n$, consider the neighborhood $B_d \left( x, \frac{1}{n} \right)$.
Choose $x_n$ to be a point of $A \cap B_d \left( x, \frac{1}{n} \right)$,
then the sequence $x_1, x_2, \ldots$ converges to $x$.
\end{proof}

\begin{theorem}
Let $f: X \rightarrow Y$ where $X$ is a metrizable space.
The function $f$ is continuous if and only if for every convergent sequence $x_n \rightarrow x$ in $X$, the sequence $f(x_n)$ converges to $f(x)$.
\end{theorem}
\begin{proof}
Suppose that $f$ is continuous.
Let $O$ be a neighborhood of $f(x)$.
Then $f^{-1}(O)$ is a neighborhood of $x$, and so there exists an integer $N$ such that $x_n \in f^{-1}(O)$ for $n \geq N$.
Thus, $f(x_n) \in O$ for all $n \geq N$ and $f(x_n) \rightarrow f(x)$.

To prove the converse, assume that the convergent sequence condition is true.
Let $A \subseteq X$.
If $x \in \Closure{A}$, then there exists a sequence $x_1, x_2, \ldots$ of points of $A$ converging to $x$.
By assumption, $f(x_n) \rightarrow f(x)$.
Since $f(x_n) \in f(A)$, the preceding theorem implies that $f(x) \in \Closure{f(A)}$.
Hence $f \left( \Closure{A} \right) \subseteq \Closure{f(A)}$ and $f$ is continuous.
\end{proof}


\subsection{Completeness}

Suppose $X$ is a metrizable space.
From Definition~\ref{definition:SequenceConvergence}, we know that a sequence $x_1, x_2, \ldots$ of points in $X$ converges to $x \in X$ if for every neighborhood $A$ of $x$ there exists a positive integer $N$ such that $x_i \in A$ for all $i \geq N$.

It is possible for a sequence in a metrizable space $X$ to satisfy the Cauchy criterion, but not to converge in $X$.

\begin{example}
Let $C[-1,1]$ be the vector space of continuous functions on the interval $[-1,1]$ and consider the $L_2$ norm
\begin{equation*}
\left\| f(t) \right\|_2 = \left( \int_{-1}^1 |f(t)|^2 dt \right)^{\frac{1}{2}}.
\end{equation*}
Define the sequence of functions $f_n(t)$ given by
\begin{equation*}
f_n(t) = \left\{ \begin{array}{ll}
0 & t \in \left[ -1, -\frac{1}{n} \right] \\
\frac{nt}{2} + \frac{1}{2} & t \in \left( -\frac{1}{n}, \frac{1}{n} \right) \\
1 & t \in \left[ \frac{1}{n}, 1 \right]
\end{array} \right. .
\end{equation*}
Assuming that $m \geq n$, we get
\begin{equation*}
d(f_n, f_m) = \left\| f_n(t) - f_m(t) \right\|_2
= \left( \int_{-1}^1 |f_n(t) - f_m(t)|^2 dt \right)^{\frac{1}{2}}
= \frac{(m-n)^2}{6m^2n}.
\end{equation*}
This sequence satisfies the Cauchy criterion, but it does not converge to a continuous function in $C[-1,1]$.
\end{example}

\begin{definition}
Two Cauchy sequences $x_1, x_2, \ldots$ and $y_1, y_2, \ldots$ are equivalent if for every $\epsilon >0$ there exists an integer $N$ such that $d (x_k, y_k) \leq \epsilon$ for all $k \geq N$.
\end{definition}

Cauchy sequences have many applications in analysis and signal processing.
They may be used to construct the set of real numbers from the set of rational numbers.

\begin{definition}
Let $\mathcal{C}$ denote the set of all Cauchy sequence $q_1, q_2, \ldots$ of rational numbers, and let $\RealNumbers$ denote the set of equivalence classes of elements of $\mathcal{C}$.
The elements $r \in \RealNumbers$, which are equivalence classes of Cauchy sequences of rationals, are called the real numbers.
\end{definition}

\begin{definition}
Let $X$ be a metrizable space.
The space $X$ is said to be \defn{topology}{complete} if every Cauchy sequence in $X$ converges to a limit $x \in X$.
\end{definition}

\begin{definition}
Let $A$ be a subset of a metric space $(X,d)$ and $f: X \rightarrow X$ be a function.
The function $f$ is a \defn{metric space}{contraction} on $A$ if $d \left( f(x),f(y) \right) \leq \gamma  d(x,y)$ for all $x,y\in X$ and some $\gamma \in [0,1)$.
\end{definition}

\begin{theorem}[Contraction Mapping Theorem]
Let $(X,d)$ be a complete metric space and $f$ be contraction on a closed subset $A \subseteq X$.
Then, $f$ has a unique fixed point $x^*$ on $A$ such that $f(x^*) = x^*$.
Moreover, the sequence $x_{n+1} = f(x_n)$ converges to $x^*$ if $x_1 \in A$ and $f(x_1) \in A$.
\end{theorem}
\begin{proof}
Suppose $f$ has two fixed points $y,z\in A$.
Then, $d(y,z) = d \left(f(y),f(z) \right) \leq \gamma  d(y,z)$ and $d(y,z) = 0$ because $\gamma \in [0,1)$.
This shows that $y=z$ and any two fixed points in $A$ must be identical.

Using induction, it is easy to see that $d(x_n,x_{n+1}) \leq \gamma^{n-1} d(x_1,x_2)$.
From this, we can bound the distance between $x_m$ and $x_n$ (for $m<n$) with
\begin{align*}
d(x_m,x_n)
& \leq d(x_m,x_{m+1}) + d(x_{m+1},x_n) \\
& \leq \sum_{i=m}^{n-1} d(x_i,x_{i+1}) \\
& \leq \sum_{i=m}^{n-1} \gamma^{i-1} d(x_1,x_2) \\
& \leq \sum_{i=m}^\infty \gamma^{i-1} d(x_1,x_2) \\
& \leq \frac{\gamma^{m-1}}{1-\gamma} d(x_1,x_2).
\end{align*}
The sequence $x_n$ is Cauchy because $d(x_m,x_n)$ can be made arbitrarily small (for all $n>m$) by increasing $m$.
Since $(X,d)$ is complete, it must converge to a fixed point and $x^*$ is the unique fixed point in $A$.
\end{proof}


