\chapter{Mathematical Review}

\section{Topology}

\begin{definition}
A \emph{topology} on a set $X$ is a collection $\mathcal{J}$ of subsets of $X$ that satisfies the following properties,
\begin{enumerate}
\item $\emptyset$ and $X$ are in $\mathcal{J}$
\item the union of the elements of any subcollection of $\mathcal{J}$ is in $\mathcal{J}$
\item the intersection of the elements of any finite subcollection of $\mathcal{J}$ is in $\mathcal{J}$.
\end{enumerate}
\end{definition}

A subset $A \subset X$ is called an \emph{open set} of $X$ if $A \in \mathcal{J}$.
Using this terminology, a topological space is a set $X$ together with a collection of subsets of $X$, called \emph{open sets}, such that $\emptyset$ and $X$ are both open and such that arbitrary unions and finite intersections of open sets are open.

\begin{definition}
If $X$ is a set, a \emph{basis} for a topology on $X$ is a collection $\mathcal{B}$ of subsets of $X$ (called \emph{basis elements}) such that
\begin{enumerate}
\item for each $x \in X$, there exists a basis element $B$ containing $x$
\item if $x \in B_1$ and $x \in B_2$ where $B_1, B_2 \in \mathcal{B}$, then there exists a basis element $B_3$ containing $x$ such that $B_3 \subset B_1 \cap B_2$.
\end{enumerate}
\end{definition}

If $\mathcal{B}$ is a basis for a topology on $X$, the topology $\mathcal{J}$ generated by $\mathcal{B}$ is described as follows.
A subset $A \subset X$ is open in $X$ if for each $x \in A$ there exists a basis element $B \in \mathcal{B}$ such that $x \in B$ and $B \subset A$. 


\subsection{Closed Sets and Limit Points}

\begin{definition}
A subset $A$ of a topological space $X$ is \emph{closed} if the set
\begin{equation*}
\Complement{A} = X - A = \{ x \in X | x \notin A \}
\end{equation*}
is open.
\end{definition}

Note that a set can be open, closed, both, or neither!
It can be shown that the collection of closed subsets of a space $X$ has properties similar to those satisfied by the collection of open subsets of $X$.

\begin{fact}
Let $X$ be a topological space.
The following conditions hold,
\begin{enumerate}
\item $\emptyset$ and $X$ are closed
\item arbitrary intersections of closed sets are closed
\item finite unions of closed sets are closed.
\end{enumerate}
\end{fact}

\begin{definition}
Given a subset $A$ of a topological space $X$, the \emph{interior} of $A$ is defined as the union of all open sets contained in $A$.
The \emph{closure} of $A$ is defined as the intersection of all closed sets containing $A$.
\end{definition}

The interior of $A$ is denoted by $\Interior{A}$ and the closure of $A$ is denoted by $\Closure{A}$.
We note that $\Interior{A}$ is open and $\Closure{A}$ is closed.
Furthermore, $\Interior{A} \subset A \subset \Closure{A}$.

\begin{theorem} \label{theorem:ClosureConditions}
Let $A$ be a subset of the topological space $X$.
The element $x$ is in $\Closure{A}$ if and only if every open set $B$ containing $x$ intersects $A$.
Furthermore, if the topology of $X$ is given by a basis, then $x \in \Closure{A}$ if and only if every basis element $B$ containing $x$ intersects $A$.
\end{theorem}

An open set $O$ containing $x$ is sometimes referred to as a \emph{neighborhood} of $x$.

\begin{definition}
Suppose $A$ is a subset of the topological space $X$ and let $x$ be an element of $X$.
Then $x$ is a \emph{limit point} of $A$ if every neighborhood of $x$ intersects $A$ in some point other than $x$ itself.
\end{definition}

In other words, $x \in X$ is a limit point of $A \subset X$ if $x \in \Closure{A - \{ x \}}$, the closure of $A - \{ x \}$.
The point $x$ may or may not be in $A$.

\begin{theorem}
A subset of a topological space is closed if and only if it contains all its limit points.
\end{theorem}


\section{The Metric Topology}

Probably the most important and frequently used way of imposing a topology on a set is to define the topology in terms of a metric.

\begin{definition}
A \emph{metric} on a set $X$ is a function
\begin{equation*}
d: X \times X \rightarrow \RealNumbers
\end{equation*}
that satisfies the following properties,
\begin{enumerate}
\item $d(x,y) \geq 0 \quad \forall x, y \in X$; equality holds if and only if $x = y$
\item $d(x,y) = d(y,x) \quad \forall x, y \in X$
\item $d(x,y) + d(y,z) \geq d(x,z) \quad \forall x, y, z \in X$.
\end{enumerate}
\end{definition}

\begin{example}
Given $\vecnot{x} = (x_1, \ldots, x_n), \vecnot{y} = (y_1, \ldots, y_n) \in \RealNumbers^n$, the \emph{euclidean metric} $d$ on $\RealNumbers^n$ is defined by the equation
\begin{equation*}
d \left( \vecnot{x}, \vecnot{y} \right)
= \sqrt{ (x_1 - y_1)^2 + \cdots + (x_n - y_n)^2 }.
\end{equation*}
As implied by its name, the function $d$ defined above is a metric.
\end{example}

Let $d$ be a metric on $X$.
The number $d(x,y)$ is called the \emph{distance} between $x$ and $y$ in the metric $d$.
Let $\epsilon > 0$ and consider the set $B_d(x,\epsilon) = \left\{ y \in X | d(x,y) < \epsilon \right\}$.
This set is called the \emph{$\epsilon$-ball centered at $x$}.

\begin{definition}
If $d$ is a metric on the set $X$, then the collection of all $\epsilon$-balls
\begin{equation*}
\left\{ B_d (x, \epsilon) |  x \in X, \epsilon > 0 \right\}
\end{equation*}
is a basis for a topology on $X$.
This topology is called the \emph{metric topology} induced by $d$.
\end{definition}

In particular, a set $A$ is open in the metric topology induced by $d$ if and only if for each $a \in A$, there exists a $\delta > 0$ such that $B_d (a, \delta) \subset A$.
Clearly, this condition implies that $A$ is open.
Conversely, if $A$ is open, it contains a basis element $B = B_d (x, \epsilon)$ containing $a$, and $B$ in turn contains a basis element $B_d (a, \delta)$ centered at $a$.

\begin{problem}
Suppose $a \in B_d(x, \epsilon)$ with $\epsilon > 0$.
Show that there exists an $\epsilon$-ball $B_d(a, \delta)$ centered at $a$ that is contained in $B_d(x, \epsilon)$.
\end{problem}

\begin{definition}
Let $X$ be a topological space.
The space $X$ is said to be \emph{metrizable} if there exists a metric $d$ on the set $X$ that induces the topology of $X$.
A \emph{metric space} is a space $X$ together with a specific metric $d$ that gives the topology of $X$.
\end{definition}

\begin{problem}
Let $\vecnot{x} = (x_1, \ldots, x_n), \vecnot{y} = (y_1, \ldots, y_n) \in \RealNumbers^n$ and consider the function $\rho$ given by
\begin{equation*}
\rho \left( \vecnot{x}, \vecnot{y} \right)
= \max \left\{ |x_1 - y_1|, \ldots, |x_n - y_n| \right\}.
\end{equation*}
Show that $\rho$ is a metric.
\end{problem}

\begin{problem} \label{problem:StandardBoundedMetric}
Let $X$ be a metric space with metric $d$.
Define $\bar{d} : X \times X \rightarrow \RealNumbers$ by
\begin{equation*}
\bar{d} (x, y)
= \min \left\{ d (x, y), 1 \right\}.
\end{equation*}
Show that $\bar{d}$ is also a metric.
\end{problem}


\section{Continuity}

\begin{definition}
Let $X$ and $Y$ be topological spaces.
A function $f: X \rightarrow Y$ is \emph{continuous} if for each open subset $O \subset Y$, the set $f^{-1} (O)$ is an open subset of $X$.
\end{definition}

Recall that $f^{-1}(B)$ is the set $\{ x \in X | f(x) \in B \}$.
Continuity of a function depends not only upon the function $f$ itself, but also on the topologies specified for its domain and range!

\begin{theorem} \label{theorem:ContinuityEquivalentConditions}
Let $X$ and $Y$ be topological spaces and consider a function $f: X \rightarrow Y$.
The following are equivalent:
\begin{enumerate}
\item $f$ is continuous
\item for every subset $A \subset X$, $f \left( \Closure{A} \right) \subset \Closure{ f(A) }$
\item for every closed set $C \subset Y$, the set $f^{-1} (C)$ is closed in $X$.
\end{enumerate}
\end{theorem}
\begin{proof}
$(1 \Rightarrow 2)$.
Assume $f$ is a continuous function.
Suppose $x \in \Closure{A}$, where $A$ is a subset of $X$.
Let $O$ be a neighborhood of $f(x)$.
Then $f^{-1}(O)$ is an open set of $X$ containing $x$; it must intersect with $A$ in some point $x'$.
It follows that $O$ intersects $f(A)$ in the point $f(x')$.
Hence $f(x) \in \Closure{ f(A) }$.

$(2 \Rightarrow 3)$.
Suppose that for every subset $A \subset X$, $f \left( \Closure{A} \right) \subset \Closure{ f(A) }$.
Let $C \subset Y$ be a closed set and let $A = f^{-1} (C)$.
By elementary set theory, we have $f(A) \subset C$.
If $x \in \Closure{A}$, then
\begin{equation*}
f(x) \in f \left( \Closure{A} \right) \subset \Closure{f (A)} \subset \Closure{C} = C.
\end{equation*}
So that $x \in f^{-1} (C) = A$ and, as a consequence, $\Closure{A} \subset A$.

$(3 \Rightarrow 1)$.
Let $O$ be an open set in $Y$.
Let $\Complement{O} = Y - O$; then $\Complement{O}$ is closed in $Y$.
By assumption, $f^{-1} (\Complement{O})$ is closed in $X$.
Using elementary set theory,
\begin{equation*}
f^{-1} (O) = f^{-1} (Y - \Complement{O}) = f^{-1}(Y) - f^{-1}(\Complement{O}) = X - f^{-1}(\Complement{O}).
\end{equation*}
That is, $f^{-1} (O)$ is open.
\end{proof}

\begin{theorem}
Suppose $X$ and $Y$ are two metrizable spaces with metrics $d_X$ and $d_Y$.
Consider a function $f : X \rightarrow Y$.
The function $f$ is continuous if and only if given $x \in X$ and $\epsilon > 0$ there exists $\delta > 0$ such that
\begin{equation*}
d_X(x_1, x_2) < \delta
\quad \text{implies} \quad
d_Y \left( f(x_1), f(x_2) \right) < \epsilon.
\end{equation*}
\end{theorem}
\begin{proof}
Suppose that $f$ is continuous.
Given $x_1$ and $\epsilon > 0$, consider the set
\begin{equation*}
O_x = f^{-1} \left( B_{d_Y} (f(x_1), \epsilon) \right)
\end{equation*}
which is open in $X$ and contains the point $x_1$.
Since $O_x$ is open and $x_1 \in O_x$, there exists a $\delta$-ball $B_{d_X} (x_1, \delta)$ centered at $x_1$ such that $B_{d_X} (x_1, \delta) \subset O_x$.
For any $x_2 \in B_{d_X} (x_1, \delta)$, it follows that $f(x_2) \in B_{d_Y} (f(x_1), \epsilon)$ as desired.

Conversely, suppose that the $\epsilon$-$\delta$ condition is satisfied.
Let $O_y$ be an open set in $Y$.
For any $x \in f^{-1} (O_y)$, $f(x) \in O_y$ and therefore there exists an $\epsilon$-ball $B_{d_Y} (f(x), \epsilon)$ centered at $f(x)$ contained in $O_y$.
By the $\epsilon$-$\delta$ condition for continuity, there exits a $\delta$-ball $B_{d_X} (x, \delta)$ centered at $x$ such that $f \left( B_{d_X} (x,\delta) \right) \subset B_{d_Y} \left( f(x), \epsilon \right)$.
The set $B_{d_X} (x, \delta)$ is a neighborhood of $x$ contained in $f^{-1} (O_y)$, which implies that $f^{-1} (O_y)$ is open.
\end{proof}

Consider the convergent sequence definition of continuity, and note that a \emph{sequence} of points of a set $X$ can be seen as a function from the positive integers to $X$.

\begin{definition} \label{definition:SequenceConvergence}
A sequence $x_1, x_2, \ldots$ of points in $X$ is said to \emph{converge} to $x \in X$ if for every neighborhood $O$ of $x$ there exists a positive integer $N$ such that $x_i \in O$ for all $i \geq N$.
\end{definition}

A sequence need not converge at all.
However, if it converges, then it converges to only one element.

\begin{theorem}
Suppose that $X$ is a metrizable space, and let $A \subset X$.
There exists a sequence of points of $A$ converging to $x$ if and only if $x \in \Closure{A}$.
\end{theorem}
\begin{proof}
Suppose $x_n \rightarrow x$, where $x_n \in A$.
By Theorem~\ref{theorem:ClosureConditions}, every neighborhood $O$ of $x$ contains a point of $A$, so $x \in \Closure{A}$.
Let $d$ be a metric for the topology of $X$.
For each positive integer $n$, consider the neighborhood $B_d \left( x, \frac{1}{n} \right)$.
Choose $x_n$ to be a point of $A \cap B_d \left( x, \frac{1}{n} \right)$,
then the sequence $x_1, x_2, \ldots$ converges to $x$.
\end{proof}

\begin{theorem}
Let $f: X \rightarrow Y$ where $X$ is a metrizable space.
The function $f$ is continuous if and only if for every convergent sequence $x_n \rightarrow x$ in $X$, the sequence $f(x_n)$ converges to $f(x)$.
\end{theorem}
\begin{proof}
Suppose that $f$ is continuous.
Let $O$ be a neighborhood of $f(x)$.
Then $f^{-1}(O)$ is a neighborhood of $x$, and so there exists an integer $N$ such that $x_n \in f^{-1}(O)$ for $n \geq N$.
Thus, $f(x_n) \in O$ for all $n \geq N$ and $f(x_n) \rightarrow f(x)$.

To prove the converse, assume that the convergent sequence condition is true.
Let $A \subset X$.
If $x \in \Closure{A}$, then there exists a sequence $x_1, x_2, \ldots$ of points of $A$ converging to $x$.
By assumption, $f(x_n) \rightarrow f(x)$.
Since $f(x_n) \in f(A)$, the preceding theorem implies that $f(x) \in \Closure{f(A)}$.
Hence $f \left( \Closure{A} \right) \subset \Closure{f(A)}$ and $f$ is continuous.
\end{proof}


\section{Completeness}

Suppose $X$ is a metrizable space.
From Definition~\ref{definition:SequenceConvergence}, we know that a sequence $x_1, x_2, \ldots$ of points in $X$ converges to $x \in X$ if for every neighborhood $A$ of $x$ there exists a positive integer $N$ such that $x_i \in A$ for all $i \geq N$.

\begin{definition}
Consider a sequence $x_1, x_2, \ldots$ in a metrizable space $X$.
This sequence satisfies the \emph{Cauchy criterion} if for any natural number $n$ there exists a natural number $M$ (depending on $n$) such that for all $j \geq M$ and $k \geq M$,
\begin{equation*}
d \left( x_j, x_k \right) \leq \frac{1}{n}.
\end{equation*}
\end{definition}

A sequence that satisfies the Cauchy criterion is called a \emph{Cauchy sequence}.

\begin{problem}
Consider a sequence $x_1, x_2, \ldots$ in a metrizable space $X$.
Show that if this sequence converges to a limit $x \in X$ then it is a Cauchy sequence.
\end{problem}

It is possible for a sequence in a metrizable space $X$ to satisfy the Cauchy criterion, but not to converge in $X$.

\begin{example}
Let $C[-1,1]$ be the vector space of continuous functions on the interval $[-1,1]$ and consider the $L_2$ norm
\begin{equation*}
\left\| f(t) \right\|_2 = \left( \int_{-1}^1 |f(t)|^2 dt \right)^{\frac{1}{2}}.
\end{equation*}
Define the sequence of functions $f_n(t)$ given by
\begin{equation*}
f_n(t) = \left\{ \begin{array}{ll}
0 & t \in \left[ -1, -\frac{1}{n} \right] \\
\frac{nt}{2} + \frac{1}{2} & t \in \left( -\frac{1}{n}, \frac{1}{n} \right) \\
1 & t \in \left[ \frac{1}{n}, 1 \right]
\end{array} \right. .
\end{equation*}
Assuming that $m \geq n$, we get
\begin{equation*}
d(f_n, f_m) = \left\| f_n(t) - f_m(t) \right\|_2
= \left( \int_{-1}^1 |f_n(t) - f_m(t)|^2 dt \right)^{\frac{1}{2}}
= \frac{(m-n)^2}{6m^2n}.
\end{equation*}
This sequence satisfies the Cauchy criterion, but it does not converge to a continuous function in $C[-1,1]$.
\end{example}

\begin{definition}
Two Cauchy sequences $x_1, x_2, \ldots$ and $y_1, y_2, \ldots$ are equivalent if for every $n$ there exists an integer $M$ such that $d (x_k, y_k) \leq \frac{1}{n}$ for all $k \geq M$.
\end{definition}

Cauchy sequences have many applications in analysis and signal processing.
They may be used to construct the real numbers from the rational numbers.

\begin{definition}
Let $\mathcal{C}$ denote the set of all Cauchy sequence $q_1, q_2, \ldots$ of rational numbers, and let $\RealNumbers$ denote the set of equivalence classes of elements of $\mathcal{C}$.
The elements $r \in \RealNumbers$, which are equivalence classes of Cauchy sequences of rationals, are called the real numbers.
\end{definition}

\begin{definition}[Completeness]
Let $X$ be a metrizable space.
The space $X$ is said to be \emph{complete} if every Cauchy sequence in $X$ converges to a limit $x \in X$.
\end{definition}


\section{Dealing with Infinity*}

\subsection{The Axiom of Choice}

The \emph{axiom of choice}, formulated by Zermelo in 1904, is innocent-looking.
However, one can prove theorems with its aid that some mathematicians were originally reluctant to accept in the past.

\begin{definition}[The Axiom of Choice]
Given a collection $\mathcal{X}$ of disjoint nonempty sets, there exists a set $C$ having exactly one element in common with each element of $\mathcal{X}$.
That is, for each $X \in \mathcal{X}$ the set $C \cap X$ contains a single element.
\end{definition}

Most mathematicians today accept the axiom of choice as part of the set theory on which they base their mathematics.
A straightforward consequence of the axiom of choice is the existence of a choice function.

\begin{lemma}[Existence of a Choice Function]
Given a collection $\mathcal{Y}$ of non-empty sets, there exists a function
\begin{equation*}
c: \mathcal{Y}  \rightarrow \bigcup_{Y \in \mathcal{Y}} Y
\end{equation*}
satisfying $c(Y) \in Y$ for every $Y \in \mathcal{Y}$.
\end{lemma}
\begin{proof}
The difference between the axiom of choice and the lemma is that in the latter statement the sets of the collection $\mathcal{Y}$ need not be disjoint.
Given an element $Y \in \mathcal{Y}$, define the set $Y'$ by
\begin{equation*}
Y' = \left\{ \left( Y, y \right) | y \in Y \right\}.
\end{equation*}
That is, $Y'$ is the collection of all ordered pairs where the first coordinate of the ordered pair is the set $Y$, and the second coordinate is an element of $Y$.
Because $Y$ contains at least one element, the set $Y'$ is nonempty.
Furthermore, $Y'$ is a subset of the cartesian product
\begin{equation*}
\mathcal{Y} \times \bigcup_{Y \in \mathcal{Y}} Y.
\end{equation*}
If $Y_1$ and $Y_2$ are two different sets in $\mathcal{Y}$, then the sets $Y_1'$ and $Y_2'$ are disjoint; specifically, the elements of $Y_1'$ and $Y_2'$ differ at least in their first coordinates.

Consider the collection
\begin{equation*}
\mathcal{Z} = \left\{ Y' | Y \in \mathcal{Y} \right\}.
\end{equation*}
This is a collection of disjoint nonempty subsets of
\begin{equation*}
\mathcal{Y} \times \bigcup_{Y \in \mathcal{Y}} Y.
\end{equation*}
By the axiom of choice, there exists a set $Z$ having exactly one element in common with each element of $\mathcal{Z}$.
Define the function
\begin{equation*}
c: \mathcal{Z} \rightarrow
\mathcal{Y} \times \bigcup_{Y \in \mathcal{Y}} Y
\end{equation*}
by $c \left( Y' \right) = Y' \cap Z$.
This function $c$ implicitly provides the rule for a function from $\mathcal{Y}$ to the set $\bigcup_{Y \in \mathcal{Y}} Y$ such that $y$ belongs to $Y$ whenever $\left( Y, y \right) \in Z$.
This rule is the desired choice function.
\end{proof}


\subsection{Well-Ordered Sets}

A \emph{simple order $<$} on a set $X$ is a relation such that, for all $x, y, z \in X$,
\begin{enumerate}
\item if $x \neq y$ then either $x < y$ or $y < x$
\item if $x < y$ then $x \neq y$
\item if $x < y$ and $y < z$ then $x < z$.
\end{enumerate}

\begin{definition}
A set $X$ with an order relation $<$ is said to be \emph{well-ordered} if every nonempty subset of $X$ has a smallest element.
\end{definition}

The set of natural numbers, for example, is well-ordered.
On the other hand, the set of integers is not well-ordered.

\begin{fact}[Well-ordering theorem]
If $X$ is a set, there exists an order relation on $X$ that is a well-ordering.
\end{fact}

This theorem was proved by Zermelo using the axiom of choice.
It startled the mathematical community in 1904 and spurred much controversy about the axiom of choice.
It is given here without a proof.

\begin{corollary}
There exists an uncountable well-ordered set.
\end{corollary}

\begin{definition}
Let $X$ be an ordered set.
Given $x \in X$, the set
\begin{equation*}
Y_x = \left\{ y \in Y | y < x \right\}
\end{equation*}
is called the \emph{section of $X$ by $x$}.
\end{definition}

\begin{corollary}
There exists an uncountable well-ordered set, every section of which is countable.
\end{corollary}

The well-ordering principle is a necessary tool in proofs \emph{by induction} when the set over which the induction process is applied is not a segment of the natural numbers; this is the so-called transfinite induction.


\subsection{The Maximum Principle}

A \emph{strict partial order $\prec$} on a set $X$ is a relation such that for all $x, y, z \in X$
\begin{enumerate}
\item if $x \prec y$ then $x \neq y$
\item if $x \prec y$ and $y \prec z$ then $x \prec z$.
\end{enumerate}

A strict partial order is similar to a simple order, except that it need not be true that for every distinct $x, y \in X$, either $x \prec y$ or $y \prec x$.

\begin{fact}[The maximum principle]
Let $X$ be a set and suppose that $\prec$ is a strict partial order on $X$.
If $Y$ is a subset of $X$ that is simply ordered by $\prec$, then there exists a maximal simply ordered subset $Z$ of $X$ containing $Y$.
\end{fact}

The maximum principle is given here without a proof.
It is interesting to note that the well-ordering theorem and the maximum principle are equivalent; either of them implies the other.
Furthermore, each of them is equivalent to the axiom of choice.

Let $\prec$ be a strict partial order on $X$.
For $x, y \in X$, the relation $x \preceq y$ holds if $x \prec y$ or $x = y$.
The relation $\preceq$ so defined is called a \emph{partial order} on $X$.
For example, the inclusion relation $\subset$ on a collection of sets is a partial order, whereas proper inclusion is a strict partial order.

