\chapter{Optional Topics}

\section{Dealing with Infinity*}

\subsection{The Axiom of Choice}

The \defn{set theory}{axiom of choice}, formulated by Zermelo in 1904, is innocent-looking.
However, one can prove theorems with its aid that some mathematicians were originally reluctant to accept in the past.

\begin{definition}[The Axiom of Choice]
Given a collection $\mathcal{X}$ of disjoint nonempty sets, there exists a set $C$ having exactly one element in common with each element of $\mathcal{X}$.
That is, for each $X \in \mathcal{X}$ the set $C \cap X$ contains a single element.
\end{definition}

Most mathematicians today accept the axiom of choice as part of the set theory on which they base their mathematics.
A straightforward consequence of the axiom of choice is the existence of a choice function.

\begin{lemma}[Existence of a Choice Function]
Given a collection $\mathcal{Y}$ of non-empty sets, there exists a function
\begin{equation*}
c: \mathcal{Y}  \rightarrow \bigcup_{Y \in \mathcal{Y}} Y
\end{equation*}
satisfying $c(Y) \in Y$ for every $Y \in \mathcal{Y}$.
\end{lemma}
\begin{proof}
The difference between the axiom of choice and the lemma is that in the latter statement the sets of the collection $\mathcal{Y}$ need not be disjoint.
Given an element $Y \in \mathcal{Y}$, define the set $Y'$ by
\begin{equation*}
Y' = \left\{ \left( Y, y \right) | y \in Y \right\}.
\end{equation*}
That is, $Y'$ is the collection of all ordered pairs where the first coordinate of the ordered pair is the set $Y$, and the second coordinate is an element of $Y$.
Because $Y$ contains at least one element, the set $Y'$ is nonempty.
Furthermore, $Y'$ is a subset of the cartesian product
\begin{equation*}
\mathcal{Y} \times \bigcup_{Y \in \mathcal{Y}} Y.
\end{equation*}
If $Y_1$ and $Y_2$ are two different sets in $\mathcal{Y}$, then the sets $Y_1'$ and $Y_2'$ are disjoint; specifically, the elements of $Y_1'$ and $Y_2'$ differ at least in their first coordinates.

Consider the collection
\begin{equation*}
\mathcal{Z} = \left\{ Y' | Y \in \mathcal{Y} \right\}.
\end{equation*}
This is a collection of disjoint nonempty subsets of
\begin{equation*}
\mathcal{Y} \times \bigcup_{Y \in \mathcal{Y}} Y.
\end{equation*}
By the axiom of choice, there exists a set $Z$ having exactly one element in common with each element of $\mathcal{Z}$.
Define the function
\begin{equation*}
c: \mathcal{Z} \rightarrow
\mathcal{Y} \times \bigcup_{Y \in \mathcal{Y}} Y
\end{equation*}
by $c \left( Y' \right) = Y' \cap Z$.
This function $c$ implicitly provides the rule for a function from $\mathcal{Y}$ to the set $\bigcup_{Y \in \mathcal{Y}} Y$ such that $y$ belongs to $Y$ whenever $\left( Y, y \right) \in Z$.
This rule is the desired choice function.
\end{proof}


\subsection{Well-Ordered Sets}

A \defn{set theory}{simple order $<$} on a set $X$ is a relation such that, for all $x, y, z \in X$,
\begin{enumerate}
\item if $x \neq y$ then either $x < y$ or $y < x$
\item if $x < y$ then $x \neq y$
\item if $x < y$ and $y < z$ then $x < z$.
\end{enumerate}

\begin{definition}
A set $X$ with an order relation $<$ is said to be \defn{set theory}{well-ordered} if every nonempty subset of $X$ has a smallest element.
\end{definition}

The set of natural numbers, for example, is well-ordered.
On the other hand, the set of integers is not well-ordered.

\begin{fact}[Well-ordering theorem]
If $X$ is a set, there exists an order relation on $X$ that is a well-ordering.
\end{fact}

This theorem was proved by Zermelo using the axiom of choice.
It startled the mathematical community in 1904 and spurred much controversy about the axiom of choice.
It is given here without a proof.

\begin{corollary}
There exists an uncountable well-ordered set.
\end{corollary}

\begin{definition}
Let $X$ be an ordered set.
Given $x \in X$, the set
\begin{equation*}
Y_x = \left\{ y \in Y | y < x \right\}
\end{equation*}
is called the \textbf{section} of $X$ by $x$.
\end{definition}

\begin{corollary}
There exists an uncountable well-ordered set, every section of which is countable.
\end{corollary}

The well-ordering principle is a necessary tool in proofs by induction when the set over which the induction process is applied is not a segment of the natural numbers; this is the so-called transfinite induction.


\subsection{The Maximum Principle}

A \defn{set theory}{strict partial order} $\prec$ on a set $X$ is a relation such that for all $x, y, z \in X$
\begin{enumerate}
\item if $x \prec y$ then $x \neq y$
\item if $x \prec y$ and $y \prec z$ then $x \prec z$.
\end{enumerate}

A strict partial order is similar to a simple order, except that it need not be true that for every distinct $x, y \in X$, either $x \prec y$ or $y \prec x$.

\begin{fact}[The maximum principle]
Let $X$ be a set and suppose that $\prec$ is a strict partial order on $X$.
If $Y$ is a subset of $X$ that is simply ordered by $\prec$, then there exists a maximal simply ordered subset $Z$ of $X$ containing $Y$.
\end{fact}

The maximum principle is given here without a proof.
It is interesting to note that the well-ordering theorem and the maximum principle are equivalent; either of them implies the other.
Furthermore, each of them is equivalent to the axiom of choice.

Let $\prec$ be a strict partial order on $X$.
For $x, y \in X$, the relation $x \preceq y$ holds if $x \prec y$ or $x = y$.
The relation $\preceq$ so defined is called a \defn{set theory}{partial order} on $X$.
For example, the inclusion relation $\subset$ on a collection of sets is a partial order, whereas proper inclusion is a strict partial order.

